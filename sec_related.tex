\section{Related Work}\label{sec:related}

We break this section into three groups: prior work with stream cipher based
FDE, prior work trading off security for other concerns, and prior work on
energy-aware mobile devices.

\mysub{Stream cipher based FDE.} The standard approach to FDE, using AES-XTS,
introduces significant overhead~\cite{AESItself}. Recently, it has been
established that encryption using \emph{stream ciphers} for FDE is faster than
using AES~\cite{StrongBox}, but accomplishing this in practice is non-trivial.
Prior work explores several approaches: a non-deterministic CTR mode
(Freestyle~\cite{Freestyle}), a length-preserving ``tweakable super-pseudorandom
permutation'' (Adiantum~\cite{Adiantum}), and a stream cipher in a binary
additive (XOR) mode leveraging LFS overwrite-averse behavior to prevent
overwrites (StrongBox~\cite{StrongBox}).

Unlike StrongBox and other work, which focuses on optimizing performance despite
re-ciphering due to overwrites, \sys maintains overwrite protections while
abstracting the idea of re-ciphering nuggets out into cipher switching models
and crypts; instead of myopically pursuing a performance win, \sys gives us the
flexibility to pursue energy/battery and security wins as well.

\mysub{Trading security for other concerns.} Further, trading off security for
energy, performance, and other concerns is not a new research
area~\cite{ScalableSecurity, WolterReinecke, ZengChow1, HaleemEtAl,
LiOmiecinski, Merkel4, Merkle3}. Goodman et al. introduced selectively
decreasing the security of some data to save energy~\cite{ScalableSecurity}.
However, their approach is designed for communication and only considered
iteration/round count, thus it did not anticipate the need for \sysA, \sysB, or
\sysC. Wolter and Reinecke study approaches to quantifying security in several
contexts~\cite{WolterReinecke}. This study anticipates the value of dynamically
switching ciphers but proposes no mechanisms to enable this in FDE. Similarly,
companies like LastPass and Google have explored performance-security tradeoffs.
Google's Adiantum uses a reduced-round version of ChaCha in exchange for
performance~\cite{Adiantum}. While not an FDE solution, LastPass has dealt with
scaling the number of iterations of PBKDF\#2, trading security for performance
during login sessions~\cite{LastPass}.

\mysub{Energy-aware mobile devices.} On mobile energy/battery life, prior works
have shown the importance of energy saving mode and bugs that drain
energy~\cite{energy-doctor, power-aware}.
