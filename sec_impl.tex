\section{Implementation}\label{sec:impl}

Our \sys implementation consists of \locTotal lines of C code (excluding
StrongBox components we re-use as is). To ensure high quality code, we also
wrote a \locTest line test suite. Our implementation is available open
source~\footnoteref{ftn:foss}. We deploy \sys on top of the BUSE virtual block
device~\cite{BUSE} as our device controller. BUSE is a thin (200 LoC)
wrapper around the standard Linux Network Block Device (NBD), allowing our
system to transact block layer requests in user space, reducing implementation
complexity.

\begin{table}[t]
    \begin{center}
        \small
        \centering
        \begin{tabular}{l|l}
            \hline
            {\bf Cipher} & {\bf Source} \\
            \hline
            AES-XTS, AES-CTR & OpenSSL~\cite{OpenSSL} \& libsodium~\cite{Sodium}
            \\
            ARM Neon ChaCha & Floodyberry~\cite{Floodyberry} \\
            Freestyle & Babu~\cite{Freestyle} \\
            eSTREAM Ciphers & libestream~\cite{libestream} \\
        \end{tabular}
    \end{center}

    \mycaption{tbl:impl}{Cipher Implementations}{See \cref{sec:impl},
    \cref{subsec:des-crypts}.}
\end{table}

% ------------- floating: \begin{floatingtable}[r]{ % note the open curly
% bracket \begin{tabular}{...} \end{tabular}....} % note the closed bracket here
% \mycaption{fig-pass}{x}{x.} \end{floatingtable}


Among the many ciphers our implementation supports, we focus on five
in this research: ChaCha8, ChaCha20~\cite{ChaCha20}, and
Freestyle~\cite{Freestyle} in three different configurations: a
``fast'' mode with parameters
\texttt{FFast($R_{min}$=$8$,$R_{max}$=$20$,$H_I$=$4$,$I_C$=$8$)}, a
``balanced'' mode with parameters \texttt{FBalanced($R_{min}$=$12$,
  $R_{max}$=$28$,$H_I$=$2$,$I_C$=$10$)}, and a ``strong'' mode with
parameters
\texttt{FStrong($R_{min}$=$20$,$R_{max}$=$36$,$H_I$=$1$,$I_C$=$12$)}.
\hank{Doesn't this contradict the sec\_des\_crypts section?  That
  section says that there are implementations AES, Rabbit, HC-128,
  Sosemanuk...  I am guessing that you implemented the large set to
  test stuff, but those mentioned here are just for the use cases. If
  that is right, then say that so that people don't get confused or
  think the paper is contradicting itself.}


Table \cref{tbl:impl} shows the cipher implementations we use (wrapped into
crypts with \sysB; see \cref{subsec:des-crypts}).

