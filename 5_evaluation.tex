\section{Evaluation}\label{sec:evaluation}

\subsection{Experimental Setup}

We implement SwitchCrypt and our experiments on a Hardkernel Odroid XU3 ARM
big.LITTLE system with Samsung Exynos 5422 A15/A7 heterogeneous multi-processing
quad core CPUs at maximum clock speed, 2 gigabyte LPDDR3 RAM at 933 MHz, and an
eMMC5.0 HS400 backing store running Ubuntu Trusty 14.04.6 LTS, kernel version
3.10.106.

We evaluate SwitchCrypt using a Linux RAM disk (tmpfs). The maximum theoretical
memory bandwidth for this Odroid model is 14.9GB/s\@. Our observed maximum
memory bandwidth is 4.5GB/s. Using a RAM disk focuses the evaluation on the
performance differences due to different ciphers.

In each experiment below, we evaluate SwitchCrypt on two high level workloads:
sequential and random I/O. In both workloads, a number of bytes are written and
then read (either 4KB, 512KB, 5MB, 40MB) 10 times. Each experiment is repeated
three times for a total of 240 tests; 30 results per byte size, 120 results per
workload. Results are accumulated and the median is taken for each byte size.

When evaluating switching strategies, a finer breakdown in workloads is made
using a pre-selected pair of ciphers we call the \emph{primary cipher} and
\emph{secondary cipher} in context, yielding a cipher configuration point given
some switching strategy; a specific \emph{ratio} of those 10 write-read pairs is
performed using the primary cipher and the remainder with secondary other. These
ratio workloads test SwitchCrypt's ability to reach configuration points not
accessible to prior work.

There are three ratios we use to evaluate SwitchCrypt's performance in this
regard: 1:3, 1:1, and 3:1. Respectively, that is 1 whole file write-read
operation in the primary cipher for every 3 whole file write-read operations in
the secondary cipher (1:3), 1 whole file write-read operation in the primary
cipher for every other whole file write-read operation in the secondary cipher
(1:1), and 3 whole file write-read operations in the primary cipher for every 1
whole file write-read operation in the secondary cipher (3:1). The security
score for each ratio point is calculated as $score$=$score_L + ||score_L -
score_H|| * r * 0.25)$ where $score_L$ is the lowest security score
(configuration), $score_H$ is the highest, and $r$ is a number representing the
ratio; 1:3=3, 1:1=2, and 3:1=1.

All experiments are performed with basic Linux I/O commands, bypassing system
caching.

In this section we answer the following questions:

\begin{enumerate}
 \item What shape does the cipher configuration tradeoff space take under
 our workloads? (\cref{subsec:1})
 \item Can SwitchCrypt achieve dynamic security/energy tradeoffs by reaching
 configuration points not accessible with prior work? (\cref{subsec:2})
 \item What is the performance and storage overhead of each cipher switching
 strategy? (\cref{subsec:3})
 \item How does incorporation of cipher switching change the threat analysis
 versus prior work? (\cref{subsec:4})
\end{enumerate}

\subsection{Switching Strategies Under Various Workloads}\label{subsec:1}

\begin{figure}[ht]
  \textbf{Baseline Cipher I/O Performance}\par\medskip
  {\begin{tikzpicture}[baseline]

    \pgfmathsetmacro{\ymax}{1.1} % set the maximum y value
    \pgfmathsetmacro{\ymaxbreak}{1.2} % set the y value at which overflow is drawn

    \begin{groupplot}[
        group style={
            group size=2 by 2,
            xlabels at=edge bottom,
            ylabels at=edge left,
            xticklabels at=edge bottom,
            yticklabels at=edge left,
            vertical sep=25pt,
            horizontal sep=15pt,
        },
        %axis x line*=bottom,
        height=3.5cm,
        width=\linewidth/1.75,
        tick align=outside,
        tick pos=bottom, % make sure ticks only appear at the bottom and left axes
        title style={yshift=-1.5ex},
        tick style={ black },
        y tick label style={ /pgf/number format/fixed, /pgf/number format/precision=0 },
        grid style={ dotted, gray },
        scatter,
        point meta=explicit symbolic,
        scatter/classes={
            c8={mark=square*},
            c20={mark=triangle*},
            ff={mark=diamond*},
            fb={mark=pentagon*},
            fs={mark=otimes}
        },
        %every node near coord/.append style={font=\tiny},
        %
        % magic to make the numbers appear above the overly long bars:
        % visualization depends on={rawy \as \rawy}, % save original y values
        % restrict y to domain*={ % now clip/restrict any y value to ymax
        %     \pgfkeysvalueof{/pgfplots/ymin}:\ymaxbreak
        % },
        % after end axis/.code={ % draw squiggly line indicating break
        %     \draw [semithick, white, decoration={snake,amplitude=0.1mm,segment length=0.75mm,post length=0.375mm}, decorate] (rel axis cs:0,1.01) -- (rel axis cs:1,1.01);
        % },
        % nodes near coords={\color{.!75!black}\pgfmathprintnumber\rawy}, % print the original y values (darkened in case they are too light)...
        % nodes near coords greater equal only=\ymax, % ... but ONLY if they are >= ymax
        % clip=false, % allow clip to protrude beyond ymax
        % Custom stuff to edit per template
        %
        xlabel={\footnotesize Security Score},
        xlabel near ticks,
        %xlabel shift={-1.5mm},
        xmin=0, xmax=4,
        xtick={ 0, 1, 2, 3, 4 },
        xticklabels={ 0,,, 3, \empty },
        %major x tick style=transparent,
        %enlarge x limits=0.2, % add some breathing room along the x axis's sides
        %
        ylabel={\footnotesize Latency (normalized)},
        ylabel near ticks,
        ylabel shift={-1.5mm},
        ymajorgrids=true,
        ymin=0, ymax=\ymax,
        ytick={ 0, 1, \ymax },
        yticklabels={ 0, 1, \empty },
        %yticklabels={ 0, 0.5, 1.5, 2 },
        % extra y ticks={1},
        % extra y tick style={grid=major, grid style={dashed, black}},
        % extra y tick label={\empty},
        %bar width=4.5pt, % change size of bars
        %
        legend cell align=center,
        legend style={ column sep=1ex },
        legend entries={%
            {\scriptsize 4K},
            {\scriptsize 512K},
            {\scriptsize 5M},
            {\scriptsize 40M}
        },
        legend style={
            draw=none,
            legend columns=4,
            at={(1.0,1.35)},
            anchor=south,
        },
    ]
        \nextgroupplot[title={Sequential Reads}]
            \addlegendimage{no markers,red}
            \addlegendimage{no markers,green,dashed}
            \addlegendimage{no markers,blue,dashdotted}
            \addlegendimage{no markers,orange,densely dotted}
            \addplot [thick, red] table [
                meta=cipher,
                x=score,
                y=latency,
                discard if symbol not={iop}{4k-r},
                discard if symbol not={order}{seq},
                col sep=space,
            ] {data/tradeoff-baseline.dat};
            % \label[c8]{fig:tnr:c8}
            % \label[c20]{fig:tnr:c20}
            % \label[ff]{fig:tnr:ff}
            % \label[fb]{fig:tnr:fb}
            % \label[fs]{fig:tnr:fs}
            \addplot [thick, dashed, green] table [
                meta=cipher,
                x=score,
                y=latency,
                discard if symbol not={iop}{512k-r},
                discard if symbol not={order}{seq},
                col sep=space
            ] {data/tradeoff-baseline.dat};
            \addplot [thick, dashdotted, blue] table [
                meta=cipher,
                x=score,
                y=latency,
                discard if symbol not={iop}{5m-r},
                discard if symbol not={order}{seq},
                col sep=space
            ] {data/tradeoff-baseline.dat};
            \addplot [thick, densely dotted, orange] table [
                meta=cipher,
                x=score,
                y=latency,
                discard if symbol not={iop}{40m-r},
                discard if symbol not={order}{seq},
                col sep=space
            ] {data/tradeoff-baseline.dat};
        \nextgroupplot[legend to name={throwaway1}, title={Random Reads}]
            \addplot [thick, red] table [
                meta=cipher,
                x=score,
                y=latency,
                discard if symbol not={iop}{4k-r},
                discard if symbol not={order}{rnd},
                col sep=space,
            ] {data/tradeoff-baseline.dat};
            \addplot [thick, dashed, green] table [
                meta=cipher,
                x=score,
                y=latency,
                discard if symbol not={iop}{512k-r},
                discard if symbol not={order}{rnd},
                col sep=space
            ] {data/tradeoff-baseline.dat};
            \addplot [thick, dashdotted, blue] table [
                meta=cipher,
                x=score,
                y=latency,
                discard if symbol not={iop}{5m-r},
                discard if symbol not={order}{rnd},
                col sep=space
            ] {data/tradeoff-baseline.dat};
            \addplot [thick, densely dotted, orange] table [
                meta=cipher,
                x=score,
                y=latency,
                discard if symbol not={iop}{40m-r},
                discard if symbol not={order}{rnd},
                col sep=space
            ] {data/tradeoff-baseline.dat};
        \nextgroupplot[legend to name={throwaway2}, title={Sequential Writes}]
            \addplot [thick, red] table [
                meta=cipher,
                x=score,
                y=latency,
                discard if symbol not={iop}{4k-w},
                discard if symbol not={order}{seq},
                col sep=space,
            ] {data/tradeoff-baseline.dat};
            \addplot [thick, dashed, green] table [
                meta=cipher,
                x=score,
                y=latency,
                discard if symbol not={iop}{512k-w},
                discard if symbol not={order}{seq},
                col sep=space
            ] {data/tradeoff-baseline.dat};
            \addplot [thick, dashdotted, blue] table [
                meta=cipher,
                x=score,
                y=latency,
                discard if symbol not={iop}{5m-w},
                discard if symbol not={order}{seq},
                col sep=space
            ] {data/tradeoff-baseline.dat};
            \addplot [thick, densely dotted, orange] table [
                meta=cipher,
                x=score,
                y=latency,
                discard if symbol not={iop}{40m-w},
                discard if symbol not={order}{seq},
                col sep=space
            ] {data/tradeoff-baseline.dat};
        \nextgroupplot[legend to name={throwaway3}, title={Random Writes}]
            \addplot [thick, red] table [
                meta=cipher,
                x=score,
                y=latency,
                discard if symbol not={iop}{4k-w},
                discard if symbol not={order}{rnd},
                col sep=space,
            ] {data/tradeoff-baseline.dat};
            \addplot [thick, dashed, green] table [
                meta=cipher,
                x=score,
                y=latency,
                discard if symbol not={iop}{512k-w},
                discard if symbol not={order}{rnd},
                col sep=space
            ] {data/tradeoff-baseline.dat};
            \addplot [thick, dashdotted, blue] table [
                meta=cipher,
                x=score,
                y=latency,
                discard if symbol not={iop}{5m-w},
                discard if symbol not={order}{rnd},
                col sep=space
            ] {data/tradeoff-baseline.dat};
            \addplot [thick, densely dotted, orange] table [
                meta=cipher,
                x=score,
                y=latency,
                discard if symbol not={iop}{40m-w},
                discard if symbol not={order}{rnd},
                col sep=space
            ] {data/tradeoff-baseline.dat};
    \end{groupplot}%
\end{tikzpicture}%
} \caption{Median sequential and random
  read and write performance baseline.}
 \label{fig:tradeoff-no-ratios}
\end{figure}

\figref{tradeoff-no-ratios} shows the security score versus median normalized
latency tradeoff between different stream ciphers when completing our sequential
and random I/O workloads. Trends for median hold when looking at tail latencies
as well. Each line represents one workload. From left to right: 4KB, 512KB, 5MB,
and 40MB respectively. Each symbol represents one of our ciphers. From left to
right: ChaCha8, ChaCha20, Freestyle Fast, Freestyle Balanced, and Freestyle
Secure. As expected, of the ciphers we tested, those with higher security scores
result in higher latency for I/O operations while ciphers with less desirable
security properties result in lower latency. The relationship between these
concerns is not simply linear, however, which exposes a rich security vs
latency/energy tradeoff space.

Besides the 4KB workload, the shape of each workload follows a similar
superlinear latency-vs-security trend, hence we will mostly focus on 40MB and
4KB workloads going forward. Due to the overhead of metadata management and the
fast completion time of the 4KB workloads (\ie{little time for amortization of
overhead}), ChaCha8 and ChaCha20 take longer to complete than the higher scoring
Freestyle Fast. This advantage is not enough to make Freestyle Balanced or
Secure complete faster than the ChaCha variants, however.

Though ChaCha8 is generally faster than ChaCha20, there is some variability in
our timing setup when capturing extremely fast events occurring close together
in time. This is why ChaCha8 sometimes appears with higher latency than
ChaCha20 for normalized 4KB workloads. ChaCha8 is not slower than ChaCha20.

\subsection{Reaching Between Static Configuration Points}\label{subsec:2}

\begin{figure}[ht]
  \textbf{Forward Switching I/O Ratio Performance}\par\medskip
  {\begin{tikzpicture}[baseline]

    \pgfmathsetmacro{\ymax}{1.1} % set the maximum y value
    \pgfmathsetmacro{\ymaxbreak}{1.2} % set the y value at which overflow is drawn

    \begin{groupplot}[
        group style={
            group size=2 by 2,
            xlabels at=edge bottom,
            ylabels at=edge left,
            xticklabels at=edge bottom,
            yticklabels at=edge left,
            vertical sep=25pt,
            horizontal sep=15pt,
        },
        %axis x line*=bottom,
        height=3.5cm,
        width=\linewidth/1.75,
        tick align=outside,
        tick pos=bottom, % make sure ticks only appear at the bottom and left axes
        title style={yshift=-1.5ex},
        tick style={ black },
        y tick label style={ /pgf/number format/fixed, /pgf/number format/precision=0 },
        grid style={ dotted, gray },
        scatter,
        point meta=explicit symbolic,
        scatter/classes={
            c8={mark=square*},
            c20={mark=triangle*},
            ff={mark=diamond*},
            fb={mark=pentagon*},
            fs={mark=otimes}
        },
        %every node near coord/.append style={font=\tiny},
        %
        % magic to make the numbers appear above the overly long bars:
        % visualization depends on={rawy \as \rawy}, % save original y values
        % restrict y to domain*={ % now clip/restrict any y value to ymax
        %     \pgfkeysvalueof{/pgfplots/ymin}:\ymaxbreak
        % },
        % after end axis/.code={ % draw squiggly line indicating break
        %     \draw [semithick, white, decoration={snake,amplitude=0.1mm,segment length=0.75mm,post length=0.375mm}, decorate] (rel axis cs:0,1.01) -- (rel axis cs:1,1.01);
        % },
        % nodes near coords={\color{.!75!black}\pgfmathprintnumber\rawy}, % print the original y values (darkened in case they are too light)...
        % nodes near coords greater equal only=\ymax, % ... but ONLY if they are >= ymax
        % clip=false, % allow clip to protrude beyond ymax
        % Custom stuff to edit per template
        %
        xlabel={\footnotesize R+R (norm)},
        xlabel near ticks,
        %xlabel shift={-1.5mm},
        xmin=0, xmax=4,
        xtick={ 0, 0.5, 1, 2, 3, 4 },
        xticklabels={ ,0,,, 1, \empty },
        %major x tick style=transparent,
        %enlarge x limits=0.2, % add some breathing room along the x axis's sides
        %
        ylabel={\footnotesize Latency},
        ylabel near ticks,
        ylabel shift={-1.5mm},
        ymajorgrids=true,
        ymin=0, ymax=\ymax,
        ytick={ 0, 1, \ymax },
        yticklabels={ 0, 1, \empty },
        %yticklabels={ 0, 0.5, 1.5, 2 },
        % extra y ticks={1},
        % extra y tick style={grid=major, grid style={dashed, black}},
        % extra y tick label={\empty},
        %bar width=4.5pt, % change size of bars
        %
        legend cell align=left,
        legend style={ column sep=1ex },
        legend entries={
            {\scriptsize Baseline},
            {\scriptsize Ratios},
            {\scriptsize },
            {\scriptsize },
            {\scriptsize },
            {\scriptsize C8},
            {\scriptsize C20},
            {\scriptsize FF},
            {\scriptsize FB},
            {\scriptsize FS}
        },
        legend style={
            draw=none,
            legend columns=5,
            at={(1.0,1.35)},
            anchor=south,
        },
    ]
        \nextgroupplot[title={Sequential 40M Reads}]
            \addlegendimage{no markers,red}
            \addlegendimage{mark=otimes*,only marks,black}
            \addlegendimage{only marks,mark=square*,white}
            \addlegendimage{only marks,mark=square*,white}
            \addlegendimage{only marks,mark=square*,white}
            \addlegendimage{only marks,mark=square*,red}
            \addlegendimage{only marks,mark=triangle*,red}
            \addlegendimage{only marks,mark=diamond*,red}
            \addlegendimage{only marks,mark=pentagon*,red}
            \addlegendimage{only marks,mark=otimes,red}
            \addplot [thick, red] table [
                meta=cipher,
                x=score,
                y=latency,
                discard if symbol not={iop}{40m-r},
                discard if number not={ratio}{0},
                discard if symbol not={order}{seq},
                col sep=space,
            ] {data/tradeoff-ratios.dat};
            \addplot [only marks] table [
                x=score,
                y=latency,
                discard if symbol not={iop}{40m-r},
                discard if number not={ratio}{1},
                discard if symbol not={order}{seq},
                col sep=space
            ] {data/tradeoff-ratios.dat};
            \addplot [only marks] table [
                x=score,
                y=latency,
                discard if symbol not={iop}{40m-r},
                discard if number not={ratio}{2},
                discard if symbol not={order}{seq},
                col sep=space
            ] {data/tradeoff-ratios.dat};
            \addplot [only marks] table [
                x=score,
                y=latency,
                discard if symbol not={iop}{40m-r},
                discard if number not={ratio}{3},
                discard if symbol not={order}{seq},
                col sep=space
            ] {data/tradeoff-ratios.dat};
        \nextgroupplot[legend to name={throwaway4}, title={Sequential 4K Reads}]
            \addplot [thick, red] table [
                meta=cipher,
                x=score,
                y=latency,
                discard if symbol not={iop}{4k-r},
                discard if number not={ratio}{0},
                discard if symbol not={order}{seq},
                col sep=space,
            ] {data/tradeoff-ratios.dat};
            \addplot [only marks] table [
                x=score,
                y=latency,
                discard if symbol not={iop}{4k-r},
                discard if number not={ratio}{1},
                discard if symbol not={order}{seq},
                col sep=space
            ] {data/tradeoff-ratios.dat};
            \addplot [only marks] table [
                x=score,
                y=latency,
                discard if symbol not={iop}{4k-r},
                discard if number not={ratio}{2},
                discard if symbol not={order}{seq},
                col sep=space
            ] {data/tradeoff-ratios.dat};
            \addplot [only marks] table [
                x=score,
                y=latency,
                discard if symbol not={iop}{4k-r},
                discard if number not={ratio}{3},
                discard if symbol not={order}{seq},
                col sep=space
            ] {data/tradeoff-ratios.dat};
        \nextgroupplot[legend to name={throwaway5}, title={Sequential 40M Writes}]
            \addplot [thick, red] table [
                meta=cipher,
                x=score,
                y=latency,
                discard if symbol not={iop}{40m-w},
                discard if number not={ratio}{0},
                discard if symbol not={order}{seq},
                col sep=space,
            ] {data/tradeoff-ratios.dat};
            \addplot [only marks] table [
                x=score,
                y=latency,
                discard if symbol not={iop}{40m-w},
                discard if number not={ratio}{1},
                discard if symbol not={order}{seq},
                col sep=space
            ] {data/tradeoff-ratios.dat};
            \addplot [only marks] table [
                x=score,
                y=latency,
                discard if symbol not={iop}{40m-w},
                discard if number not={ratio}{2},
                discard if symbol not={order}{seq},
                col sep=space
            ] {data/tradeoff-ratios.dat};
            \addplot [only marks] table [
                x=score,
                y=latency,
                discard if symbol not={iop}{40m-w},
                discard if number not={ratio}{3},
                discard if symbol not={order}{seq},
                col sep=space
            ] {data/tradeoff-ratios.dat};
        \nextgroupplot[legend to name={throwaway6}, title={Sequential 4K Writes}]
            \addplot [thick, red] table [
                meta=cipher,
                x=score,
                y=latency,
                discard if symbol not={iop}{4k-w},
                discard if number not={ratio}{0},
                discard if symbol not={order}{seq},
                col sep=space,
            ] {data/tradeoff-ratios.dat};
            \addplot [only marks] table [
                x=score,
                y=latency,
                discard if symbol not={iop}{4k-w},
                discard if number not={ratio}{1},
                discard if symbol not={order}{seq},
                col sep=space
            ] {data/tradeoff-ratios.dat};
            \addplot [only marks] table [
                x=score,
                y=latency,
                discard if symbol not={iop}{4k-w},
                discard if number not={ratio}{2},
                discard if symbol not={order}{seq},
                col sep=space
            ] {data/tradeoff-ratios.dat};
            \addplot [only marks] table [
                x=score,
                y=latency,
                discard if symbol not={iop}{4k-w},
                discard if number not={ratio}{3},
                discard if symbol not={order}{seq},
                col sep=space
            ] {data/tradeoff-ratios.dat};
    \end{groupplot}
\end{tikzpicture}%
} \caption{Median sequential and
  random read and write performance comparison of Forward switching to
  baseline.}
 \label{fig:tradeoff-with-ratios}
\end{figure}

\figref{tradeoff-with-ratios} shows the security score versus median normalized
latency tradeoff between different stream ciphers when completing our sequential
and random I/O workloads \emph{with cipher switching using the Forward
strategy}. As with \figref{tradeoff-no-ratios}, each line represents one
workload and each symbol represents one of our ciphers. From left to right:
ChaCha8, ChaCha20, Freestyle Fast, Freestyle Balanced, and Freestyle Secure.

After a certain number of write-read I/O operations, a cipher switch is
initiated and SwitchCrypt begins using the secondary cipher to encrypt and
decrypt data. For each pair of ciphers, this is repeated three times: once at
every ratio point \emph{between} our static configuration points (\ie{1:3, 1:1,
and 3:1 described above}).

The point of this experiment is to determine if SwitchCrypt can effectively
transition the backing store between ciphers without devastating performance.
For the 40MB, 5MB, and 512KB workloads (40MB is shown), we see that SwitchCrypt
can achieve dynamic security/energy tradeoffs reaching points not accessible
with prior work, all with minimal overhead.

Again, due to the overhead of metadata management for non-Freestyle ciphers (see
\secref{implementation}) and the fast completion time of the 4KB workloads
preventing SwitchCrypt from taking advantage of amortization, ChaCha8 and
ChaCha20 take longer to complete than the higher scoring Freestyle Fast for 4KB
reads. This advantage is not enough to make the ratios involving Freestyle
Balanced or Secure complete faster than the ChaCha ratio variants, however.

We also see very high latency for ratios between Freestyle Fast and Freestyle
Secure cipher configurations. This is because Freestyle, while avoiding metadata
management overhead, is not length-preserving (so extra write operations must be
performed), and is generally much slower than the ChaCha variants (see
\figref{tradeoff-no-ratios}). Doubly invoking Freestyle in a ratio configuration
means these penalties are paid more often, ballooning latency.

\begin{figure}[ht]
  \textbf{Mirrored and Selective Switching I/O Ratio Performance}\par\medskip
  \centering
  {\begin{tikzpicture}[baseline]

    \pgfmathsetmacro{\ymax}{1.1} % set the maximum y value
    \pgfmathsetmacro{\ymaxbreak}{1.2} % set the y value at which overflow is drawn

    \begin{groupplot}[
        group style={
            group size=2 by 4,
            xlabels at=edge bottom,
            ylabels at=edge left,
            xticklabels at=edge bottom,
            yticklabels at=edge left,
            vertical sep=25pt,
            horizontal sep=15pt,
        },
        %axis x line*=bottom,
        height=3cm,
        width=\textwidth/4,
        tick align=outside,
        tick pos=bottom, % make sure ticks only appear at the bottom and left axes
        title style={yshift=-1.0ex},
        tick style={ black },
        y tick label style={ /pgf/number format/fixed, /pgf/number format/precision=0 },
        grid style={ dotted, gray },
        scatter,
        point meta=explicit symbolic,
        scatter/classes={
            c8={mark=square*},
            c20={mark=triangle*},
            ff={mark=diamond*},
            fb={mark=pentagon*},
            fs={mark=*},
            mirrored={mark=otimes},
            selective={mark=oplus}
        },
        %every node near coord/.append style={font=\tiny},
        %
        % magic to make the numbers appear above the overly long bars:
        % visualization depends on={rawy \as \rawy}, % save original y values
        % restrict y to domain*={ % now clip/restrict any y value to ymax
        %     \pgfkeysvalueof{/pgfplots/ymin}:\ymaxbreak
        % },
        % after end axis/.code={ % draw squiggly line indicating break
        %     \draw [semithick, white, decoration={snake,amplitude=0.1mm,segment length=0.75mm,post length=0.375mm}, decorate] (rel axis cs:0,1.01) -- (rel axis cs:1,1.01);
        % },
        % nodes near coords={\color{.!75!black}\pgfmathprintnumber\rawy}, % print the original y values (darkened in case they are too light)...
        % nodes near coords greater equal only=\ymax, % ... but ONLY if they are >= ymax
        clip=false, % allow clip to protrude beyond ymax
        % Custom stuff to edit per template
        %
        xlabel near ticks,
        %xlabel shift={-5mm},
        xmin=0, xmax=4,
        %%major x tick style=transparent,
        %enlarge x limits=0.2, % add some breathing room along the x axis's sides
        %
        ylabel={\footnotesize Latency (norm)},
        ylabel near ticks,
        %label shift={-1.5mm},
        ymajorgrids=true,
        ymin=0, ymax=\ymax,
        ytick={ 0, 1, \ymax },
        yticklabels={ 0, 1, \empty },
        %yticklabels={ 0, 0.5, 1.5, 2 },
        % extra y ticks={1},
        % extra y tick style={grid=major, grid style={dashed, black}},
        % extra y tick label={\empty},
        %bar width=4.5pt, % change size of bars
        %
        legend cell align=center,
        legend style={ column sep=1ex },
        legend entries={
            {\scriptsize Baseline I/O},
            {\scriptsize SwitchCrypt Ratio Configuration I/O}
        },
        legend style={
            draw=none,
            legend columns=4,
            at={(1.0,1.45)},
            anchor=south,
        },
    ]
        \nextgroupplot[title={40M Mirrored Reads}]
            \addlegendimage{mark=none,red}
            \addlegendimage{mark=otimes,only marks,black}
            \addplot [thick, red] table [
                meta=cipher,
                x=score,
                y=latency,
                discard if symbol not={iop}{40m-r},
                discard if symbol not={ratio}{0},
                discard if symbol not={strategy}{mirrored},
                col sep=space,
            ] {data/mirrored-selective-baseline.dat};
            \addplot [only marks] table [
                meta=strategy,
                x=score,
                y=latency,
                discard if symbol not={iop}{40m-r},
                discard if symbol not={ratio}{1},
                discard if symbol not={strategy}{mirrored},
                col sep=space
            ] {data/mirrored-selective-baseline.dat};
            \addplot [only marks] table [
                meta=strategy,
                x=score,
                y=latency,
                discard if symbol not={iop}{40m-r},
                discard if symbol not={ratio}{2},
                discard if symbol not={strategy}{mirrored},
                col sep=space
            ] {data/mirrored-selective-baseline.dat};
            \addplot [only marks] table [
                meta=strategy,
                x=score,
                y=latency,
                discard if symbol not={iop}{40m-r},
                discard if symbol not={ratio}{3},
                discard if symbol not={strategy}{mirrored},
                col sep=space
            ] {data/mirrored-selective-baseline.dat};
        \nextgroupplot[legend to name={throwaway19}, title={40M Selective Reads}]
            \addplot [thick, red] table [
                meta=cipher,
                x=score,
                y=latency,
                discard if symbol not={iop}{40m-r},
                discard if symbol not={ratio}{0},
                discard if symbol not={strategy}{selective},
                col sep=space,
            ] {data/mirrored-selective-baseline.dat};
            \addplot [only marks] table [
                meta=strategy,
                x=score,
                y=latency,
                discard if symbol not={iop}{40m-r},
                discard if symbol not={ratio}{1},
                discard if symbol not={strategy}{selective},
                col sep=space
            ] {data/mirrored-selective-baseline.dat};
            \addplot [only marks] table [
                meta=strategy,
                x=score,
                y=latency,
                discard if symbol not={iop}{40m-r},
                discard if symbol not={ratio}{2},
                discard if symbol not={strategy}{selective},
                col sep=space
            ] {data/mirrored-selective-baseline.dat};
            \addplot [only marks] table [
                meta=strategy,
                x=score,
                y=latency,
                discard if symbol not={iop}{40m-r},
                discard if symbol not={ratio}{3},
                discard if symbol not={strategy}{selective},
                col sep=space
            ] {data/mirrored-selective-baseline.dat};
        \nextgroupplot[legend to name={throwaway20}, title={4K Mirrored Reads}]
            \addplot [thick, red] table [
                meta=cipher,
                x=score,
                y=latency,
                discard if symbol not={iop}{4k-r},
                discard if symbol not={ratio}{0},
                discard if symbol not={strategy}{mirrored},
                col sep=space,
            ] {data/mirrored-selective-baseline.dat};
            \addplot [only marks] table [
                meta=strategy,
                x=score,
                y=latency,
                discard if symbol not={iop}{4k-r},
                discard if symbol not={ratio}{1},
                discard if symbol not={strategy}{mirrored},
                col sep=space
            ] {data/mirrored-selective-baseline.dat};
            \addplot [only marks] table [
                meta=strategy,
                x=score,
                y=latency,
                discard if symbol not={iop}{4k-r},
                discard if symbol not={ratio}{2},
                discard if symbol not={strategy}{mirrored},
                col sep=space
            ] {data/mirrored-selective-baseline.dat};
            \addplot [only marks] table [
                meta=strategy,
                x=score,
                y=latency,
                discard if symbol not={iop}{4k-r},
                discard if symbol not={ratio}{3},
                discard if symbol not={strategy}{mirrored},
                col sep=space
            ] {data/mirrored-selective-baseline.dat};
        \nextgroupplot[legend to name={throwaway14}, title={4K Selective Reads}]
            \addplot [thick, red] table [
                meta=cipher,
                x=score,
                y=latency,
                discard if symbol not={iop}{4k-r},
                discard if symbol not={ratio}{0},
                discard if symbol not={strategy}{selective},
                col sep=space,
            ] {data/mirrored-selective-baseline.dat};
            \addplot [only marks] table [
                meta=strategy,
                x=score,
                y=latency,
                discard if symbol not={iop}{4k-r},
                discard if symbol not={ratio}{1},
                discard if symbol not={strategy}{selective},
                col sep=space
            ] {data/mirrored-selective-baseline.dat};
            \addplot [only marks] table [
                meta=strategy,
                x=score,
                y=latency,
                discard if symbol not={iop}{4k-r},
                discard if symbol not={ratio}{2},
                discard if symbol not={strategy}{selective},
                col sep=space
            ] {data/mirrored-selective-baseline.dat};
            \addplot [only marks] table [
                meta=strategy,
                x=score,
                y=latency,
                discard if symbol not={iop}{4k-r},
                discard if symbol not={ratio}{3},
                discard if symbol not={strategy}{selective},
                col sep=space
            ] {data/mirrored-selective-baseline.dat};
        \nextgroupplot[
            legend to name={throwaway15},
            title={40M Mirrored Writes}
        ]
            \addplot [thick, red] table [
                meta=cipher,
                x=score,
                y=latency,
                discard if symbol not={iop}{40m-w},
                discard if symbol not={ratio}{0},
                discard if symbol not={strategy}{mirrored},
                col sep=space,
            ] {data/mirrored-selective-baseline.dat};
            \addplot [only marks] table [
                meta=strategy,
                x=score,
                y=latency,
                discard if symbol not={iop}{40m-w},
                discard if symbol not={ratio}{1},
                discard if symbol not={strategy}{mirrored},
                col sep=space
            ] {data/mirrored-selective-baseline.dat};
            \addplot [only marks] table [
                meta=strategy,
                x=score,
                y=latency,
                discard if symbol not={iop}{40m-w},
                discard if symbol not={ratio}{2},
                discard if symbol not={strategy}{mirrored},
                col sep=space
            ] {data/mirrored-selective-baseline.dat};
            \addplot [only marks] table [
                meta=strategy,
                x=score,
                y=latency,
                discard if symbol not={iop}{40m-w},
                discard if symbol not={ratio}{3},
                discard if symbol not={strategy}{mirrored},
                col sep=space
            ] {data/mirrored-selective-baseline.dat};
        \nextgroupplot[
            legend to name={throwaway16},
            title={40M Selective Writes}
        ]
            \addplot [thick, red] table [
                meta=cipher,
                x=score,
                y=latency,
                discard if symbol not={iop}{40m-w},
                discard if symbol not={ratio}{0},
                discard if symbol not={strategy}{selective},
                col sep=space,
            ] {data/mirrored-selective-baseline.dat};
            \addplot [only marks] table [
                meta=strategy,
                x=score,
                y=latency,
                discard if symbol not={iop}{40m-w},
                discard if symbol not={ratio}{1},
                discard if symbol not={strategy}{selective},
                col sep=space
            ] {data/mirrored-selective-baseline.dat};
            \addplot [only marks] table [
                meta=strategy,
                x=score,
                y=latency,
                discard if symbol not={iop}{40m-w},
                discard if symbol not={ratio}{2},
                discard if symbol not={strategy}{selective},
                col sep=space
            ] {data/mirrored-selective-baseline.dat};
            \addplot [only marks] table [
                meta=strategy,
                x=score,
                y=latency,
                discard if symbol not={iop}{40m-w},
                discard if symbol not={ratio}{3},
                discard if symbol not={strategy}{selective},
                col sep=space
            ] {data/mirrored-selective-baseline.dat};
        \nextgroupplot[
            legend to name={throwaway17},
            title={4K Mirrored Writes},
            xlabel={\footnotesize Security Score},
            xtick={ 0, 1, 2, 3, 4 },
            xticklabels={ 0,,, 3, \empty }
        ]
            \addplot [thick, red] table [
                meta=cipher,
                x=score,
                y=latency,
                discard if symbol not={iop}{4k-w},
                discard if symbol not={ratio}{0},
                discard if symbol not={strategy}{mirrored},
                col sep=space,
            ] {data/mirrored-selective-baseline.dat};
            \addplot [only marks] table [
                meta=strategy,
                x=score,
                y=latency,
                discard if symbol not={iop}{4k-w},
                discard if symbol not={ratio}{1},
                discard if symbol not={strategy}{mirrored},
                col sep=space
            ] {data/mirrored-selective-baseline.dat};
            \addplot [only marks] table [
                meta=strategy,
                x=score,
                y=latency,
                discard if symbol not={iop}{4k-w},
                discard if symbol not={ratio}{2},
                discard if symbol not={strategy}{mirrored},
                col sep=space
            ] {data/mirrored-selective-baseline.dat};
            \addplot [only marks] table [
                meta=strategy,
                x=score,
                y=latency,
                discard if symbol not={iop}{4k-w},
                discard if symbol not={ratio}{3},
                discard if symbol not={strategy}{mirrored},
                col sep=space
            ] {data/mirrored-selective-baseline.dat};
        \nextgroupplot[
            legend to name={throwaway18},
            title={4K Selective Writes},
            xlabel={\footnotesize Security Score},
            xtick={ 0, 1, 2, 3, 4 },
            xticklabels={ 0,,, 3, \empty }
        ]
            \addplot [thick, red] table [
                meta=cipher,
                x=score,
                y=latency,
                discard if symbol not={iop}{4k-w},
                discard if symbol not={ratio}{0},
                discard if symbol not={strategy}{selective},
                col sep=space,
            ] {data/mirrored-selective-baseline.dat};
            \addplot [only marks] table [
                meta=strategy,
                x=score,
                y=latency,
                discard if symbol not={iop}{4k-w},
                discard if symbol not={ratio}{1},
                discard if symbol not={strategy}{selective},
                col sep=space
            ] {data/mirrored-selective-baseline.dat};
            \addplot [only marks] table [
                meta=strategy,
                x=score,
                y=latency,
                discard if symbol not={iop}{4k-w},
                discard if symbol not={ratio}{2},
                discard if symbol not={strategy}{selective},
                col sep=space
            ] {data/mirrored-selective-baseline.dat};
            \addplot [only marks] table [
                meta=strategy,
                x=score,
                y=latency,
                discard if symbol not={iop}{4k-w},
                discard if symbol not={ratio}{3},
                discard if symbol not={strategy}{selective},
                col sep=space
            ] {data/mirrored-selective-baseline.dat};
    \end{groupplot}
\end{tikzpicture}%
} \caption{Median sequential
  and random read and write performance comparison of Mirrored and Selective
  switching strategies to baseline.}
 \label{fig:mirrored-selective-baseline}
\end{figure}

\figref{mirrored-selective-baseline} show the performance of the Mirrored and
Selective strategies with the same configuration of ratios as
\figref{tradeoff-with-ratios}.

For the 40MB, 5MB, and 512KB workloads (40MB is shown), we see that Mirrored and
Selective \emph{read} workloads and the Selective \emph{write} workload achieve
parity with the Forward strategy experiments. This makes sense, as most of the
overhead for Selective and Mirrored reads is determining which part of the
backing store to commit data to. The same applies to Selective writes.

For the 4KB Mirrored and Selective \emph{read} workloads and the Selective
\emph{write} workload, we see behavior similar to that in
\figref{tradeoff-with-ratios}, as expected.

Mirrored writes across all workloads are very slow. This is to be expected,
since the data is being mirrored across all areas of the backing store. In our
experiments, the backing store can be considered partitioned in half. This
overhead is most egregious for the 4KB Mirrored write workload. This makes
Selective preferable to Mirrored; however, Selective can never converge the
backing store to a single cipher configuration or survive the loss of an entire
region (see: \secref{usecases}).

\subsection{Cipher Switching Overhead}\label{subsec:3}

From the results of the previous three experiments, we calculate that Forward
switching has average overhead at 0.08x/0.10x for 40MB, 5MB and 512KB read/write
workloads compared to baseline I/O, demonstrating SwitchCrypt's amortization of
cipher switching costs. Average overhead is\\0.38x/0.44x for 4KB read/write
workloads when SwitchCrypt is unable to amortize cost. There is no spatial
overhead with the Forward switching strategy.

Similarly, we calculate that Selective switching has average overhead at 0x/0.3x
for 40MB, 5MB and 512KB read/write workloads compared to baseline I/O. Average
overhead is 0.22x/0.71x for 4KB read/write workloads. Spatial overhead in our
experiment was half of all writable space on the backing store.

Finally, we calculate that Mirrored switching has average overhead at
0.25x/0.61x for 40MB, 5MB and 512KB read/write workloads compared to baseline
I/O, with high write latency due to mirroring. Average overhead is 0.55x/0.77x
for 4KB read/write workloads. Spatial overhead in our experiment was half of all
writable space on the backing store.

These overhead numbers are the penalty paid for the additional flexibility of
being able to reach configurations points that are unachievable without
SwitchCrypt. We believe SwitchCrypt's design keeps these overheads acceptably
low in practice (see \secref{usecases}), achieving the desired goal of flexibly
navigating latency/security tradeoffs for FDE.

\subsection{Revisiting Our Threat Model}\label{subsec:4}

The incorporation of cipher switching requires considering each attack in the
context of each strategy and novel attacks on mixed-cipher nugget states.
\tblref{security} lists possible attacks and their results. It can be inferred
from these results and SwitchCrypt's design that SwitchCrypt addresses its
threat model and maintains confidentiality and integrity guarantees. In the case
of the final attack, we stress that even the least scored cipher is still
considered secure (\secref{design}).

\begin{table}[t]
  \caption{Attacks on SwitchCrypt and their results}\label{tbl:security}
  \footnotesize
  \centering
  \begin{tabu} to \linewidth { | X[l] | X[c] | X[c] | }
    \hline
    \textbf{Attack} & \textbf{Result} & \textbf{Explanation} \\
    \hline\hline
    Nugget data in backing store is mutated out-of-band online &
    SwitchCrypt immediately fails with exception on successive IO request &
    Regardless of switching strategy, Merkle Tree check fails on read-in\\
    \hline
    Header/cipher metadata in backing store is mutated out-of-band online &
    SwitchCrypt immediately fails with exception on successive IO request &
    Regardless of switching strategy, Merkle Tree check fails on read-in\\
    \hline
    Backing store is rolled back to a previously consistent state while online &
    SwitchCrypt immediately fails with exception on successive IO request &
    Regardless of switching strategy, secure monotonic counter is out of sync
    with the system\\
    \hline
    Backing store is rolled back to a previously consistent state while offline,
    secure counter wildly out of sync & SwitchCrypt refuses to mount; allows for
    force mount with root access & Regardless of switching strategy, secure
    monotonic counter is out of sync with the system\\
    \hline
    Merkle Tree made inconsistent by mutating backing store out-of-band while
    offline, secure counter in sync & SwitchCrypt refuses to mount & Secure
    monotonic counter is in sync with the system, yet illegal data manipulation
    occurred\\
    \hline
    Nugget accessed out-of-band in mixed-cipher backing store & Nugget's cipher
    may not match active cipher & Worst case, the nugget data is available
    encrypted with the lowest scored cipher in the system\\
    \hline\hline
  \end{tabu}
\end{table}
