\subsection{\sysB: Generic Cipher Interface}\label{subsec:des-crypts}

\textbf{Decoupling ciphers from the encryption process.} To flexibly switch
between configurations in \sys requires a generic cipher interface. This is
challenging given the variety of inputs required by various stream ciphers, the
existence of non-length-preserving ciphers, and other differences. We achieve
the required generality by defining independent storage units called
\emph{nuggets}; we borrow this terminology from prior work (see
\cite{StrongBox}) to easily differentiate our logical blocks (nuggets) from
physical drive and other storage blocks. And since they are independent, we can
use our interface to select any configuration to encrypt or decrypt any nugget
at any point.

... and interface for \sysB. Comes with an interface.

\hsg{what do you exactly mean by: \\
- ``crypts'' needs to be a thing? \\
- ``implementation detail'' \\
- ``cipher implementations'' \\
- the ``encryption/decryption process'' \\
- ``modification to third party code'' \\
- ``allows any stream cipher to be integrated to \sys'' \\
- what is the ``cryptographic driver'' \\
- ``interface'' -- interface between what and what? between the OS and \sys?
  between \sys and the cipher configuration? who calls the xorInterface,
  read/writeInterface? \\
- ``\sys internals'' \\
}

% About: challenge, input, output
One of the goals of \sys is that we might use any stream cipher regardless of
its implementation details. Yet this is entirely non-trivial. There are many
cipher implementations that we might use with \sys, each with unique input
requirements and output considerations. For instance, Salsa and Chacha
implementations require a certain IV and key size and handle plaintext input
through successive invocations of a single state update
function~\cite{Floodyberry}. Using OpenSSL's AES implementation in CTR mode
requires manually tracking the counter state and individual ciphertext blocks
are retrieved though corresponding function invocations~\cite{OpenSSL}.
Freestyle's reference implementation requires we calculate the extra space
necessary per nugget (due to ciphertext expansion) along with
configuration-dependent minimum and maximum rounds-per-block, hash interval, and
pepper bits~\cite{Freestyle}. HC-128 and other ciphers have similarly disparate
requirements.

\hsg{need a figure to show what is going on}

% About: the goals
Unlike prior work, \sys must be able to encrypt and decrypt arbitrary nuggets
\emph{with any of these ciphers} at any moment with low overhead and without
tight coupling to any specific implementation detail. Hence, there is a need for
an interface that completely decouples cipher implementations from the
encryption/decryption process. Our novel cipher interface allows any stream
cipher to be integrated into \sys without modifications to third-party code,
enabling normally incompatible ciphers to encrypt and decrypt arbitrary nuggets.
The ability for disparate cipher implementations to co-exist forms the
foundation for \sys's ability to switch the system between different cipher
configurations in our tradeoff space efficiently and effectively.

% About: single enc/dec model, the OS has one way to talk to \sys
To facilitate this, the Generic Stream Cipher Interface presents the
cryptographic driver with a single unified encryption/decryption model. \sys
receives I/O requests from the operating system at the block device level like
any other device-mapper. These requests come in the form of either reads or
writes. When a read request is received, the OS hands \sys an offset and a
length and expects a response with plaintext of that specific length starting at
that specific offset taken from the beginning of storage. When a write request
is received, the OS hands \sys an offset, a length, and a buffer of plaintext
and expects that plaintext to be encrypted and committed to storage such that
the plaintext is later retrievable given that same offset and length in a future
read request. These requests can either be handled together by a single function
or handled individually as distinct read and write operations, each with
different tradeoffs.

% ---------------------------------- xor interface

\mysub{\texttt{xor\_interface}} executes independently of \sys internals and
treats encryption and decryption as the same operation. Implementations receive
an integer offset $F$, an integer length $L$, a key buffer $K$ corresponding to
the current nugget, and an empty $L$-length XOR buffer. \sys expects the XOR
buffer to be populated with $L$ bytes of keystream output from some stream
cipher seeked to offset $F$ with respect to key $K$. The length of the key
buffer will always be exactly what the cipher implementation expects,
alleviating the burden of key management; similarly, the XOR buffer will be
XOR-ed with the appropriate portion of nugget contents automatically,
alleviating the burden of drive access and other tedious calculations.

% ---------------------------------- read/write interface

\mysub{\texttt{read\_interface} and \texttt{write\_interface}} Unlike
\texttt{xor\_interface}, encryption and decryption are distinct concerns at this
abstraction level. \\\texttt{read\_interface} handles decryption during reads.
\\\texttt{write\_interface} handles encryption during writes. Implementations
receive full access to \sys internals, giving wrapper code complete control over
the encryption and decryption process and allowing implementers to bypass parts
of the nugget-based drive layout abstraction (\ie BODY) if necessary. This comes
at the cost of 1) significantly increased code complexity, as the implementer
must perform certain I/O manually, distinguish between independent nuggets on
the drive, determine what to encrypt or decrypt at what offset and when, when to
commit which metadata and where and 2) potential performance implications, since
\sys must account for not having absolute control over its internal data
structures during function invocation. For a cipher like Freestyle,
configurations with lower minimum and maximum rounds per block may see a
performance improvement here, while configurations with higher minimum and
maximum rounds per block may see reduced performance.

And then with this interface, we have implemented \hsg{examples of cipher
wrappers implemented needed here}, each in \xxx-\xxx LOC without the core cipher
algorithm.




