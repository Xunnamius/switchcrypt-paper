\subsection{\sysB: Cipher Wrapper Interface}\label{subsec:des-crypts}

% About: challenge, input, output
One of the goals of \sys is that we might use any stream cipher regardless of
its implementation details. However, allowing these ciphers to co-exist in the
same volume is entirely non-trivial since there are many cipher implementations
that we might use with \sys, each with unique input requirements and output
considerations. For instance, Salsa and Chacha implementations require a certain
IV and key size and handle plaintext input through successive invocations of a
single state update function~\cite{Floodyberry}. Using OpenSSL's AES
implementation in CTR mode requires manually tracking the counter state and
individual ciphertext blocks are retrieved though corresponding function
invocations~\cite{OpenSSL}. Freestyle's reference implementation requires we
calculate the extra space necessary per nugget (due to ciphertext expansion)
along with configuration-dependent minimum and maximum rounds-per-block, hash
interval, and pepper bits~\cite{Freestyle}. HC-128 and other ciphers have
similarly disparate requirements.

Further, unlike prior work, \sys must be able to encrypt and decrypt arbitrary
nuggets \emph{with any of these ciphers} at any moment with low overhead and
without tight coupling to any specific implementation detail. Hence, we must
abstract away these input and output requirements by decoupling cipher
implementations from the core encryption process. We present \sysB, a collection
of interfaces that allow implementors to write light (<100 LOC) wrapper
functions around cipher implementations without modifications to third-party
code; we call these wrapped ciphers {\em crypts}. Crypts present \sys with a
uniform encryption and decryption interface for each cipher, enabling normally
incompatible ciphers to encrypt and decrypt arbitrary nuggets.

% About: single enc/dec model, the OS has one way to talk to \sys
The ability for disparate cipher implementations to co-exist in this way forms
the foundation for \sys's ability to switch the system between different cipher
configurations efficiently and effectively. To facilitate this, \sysB presents
the cryptographic driver with a single uniform encryption/decryption model. \sys
receives I/Os from the operating system at the block device level like any other
device-mapper. These I/Os come in the form of either reads or writes. When a
read is received, the OS hands \sys an offset and a length and expects a
response with plaintext of that specific length starting at that specific offset
taken from the beginning of storage. When a write is received, the OS hands \sys
an offset, a length, and a buffer of plaintext and expects that plaintext to be
encrypted and committed to storage such that the plaintext is later retrievable
given that same offset and length in a future read. Crypts handle these I/Os by
implementing either \texttt{xor\_interface} or both \texttt{read\_interface} and
\texttt{write\_interface}.

% ---------------------------------- xor interface

{\bf \texttt{xor\_interface}} executes independently of \sys internals and
treats encryption and decryption as the same operation. Crypts receive an
integer offset $F$, an integer length $L$, a key buffer $K$ corresponding to the
current nugget, and an empty $L$-length XOR buffer. \sys expects the XOR buffer
to be populated with $L$ bytes of keystream output from some stream cipher
seeked to offset $F$ with respect to key $K$. The length of the key buffer will
always be exactly what the cipher implementation expects, alleviating the burden
of key management; similarly, the XOR buffer will be XOR-ed with the appropriate
portion of nugget contents automatically, alleviating the burden of drive access
and other tedious calculations.

% ---------------------------------- read/write interface

{\bf \texttt{read\_interface}} and {\bf \texttt{write\_interface}}, on the other
hand, treat encryption and decryption as distinct concerns.
\texttt{read\_interface} handles decryption and re-ciphering during reads.
\texttt{write\_interface} handles encryption and re-ciphering during writes.
Crypts receive full access to \sys internals, giving wrapper code deep hooks
into the encryption and decryption process and allowing implementers to bypass
parts of the nugget-based storage layout if necessary. This comes at the cost of
increased code complexity and potential performance implications, since \sys
must account for not having absolute control over its internal data structures
when using this crypt.

For this work we have implemented \numConfigs crypts using \numCiphers
ciphers---ChaCha, Freestyle, AES-CTR, AES-XTS, Rabbit, Sosemanuk, HC-128---each
in under 100 LOC (excluding the cipher algorithm itself).
