\section{Introduction}\label{sec:introduction}

% INTRO

% About: security in general
Security is a very important property of storage systems. Decades of systems and
storage research in this space have looked into security in the context of disk
controllers/FTL and secure hardware \cite{hardware1, hardware2, hardware3,
hardware4}; filesystems \cite{filesystems1, filesystems2, filesystems3,
filesystems4, filesystems5, filesystems6, filesystems7}; network storage and
cloud storage \cite{network1, network2, network3, network4, network5, network6,
network7, network8, network9, network10, network11, network12, network13,
network14, network15, network16}, de-duplication and secure erasure
\cite{erase1, erase2, erase3}; as well as kernels, databases, and other software
\cite{software1, software2, software3, software4}.

% About: FDE with AES is bad
For local storage, the state of the art for securing data at rest---such as the
contents of a laptop's SSD---is Full Drive Encryption (FDE). Popular FDE
solutions include dm-crypt~\cite{dmcrypt, DmC-Android} for Linux and BitLocker
for Windows~\cite{bitlocker1, bitlocker2}. Behind these implementations is the
industry standard AES \emph{block cipher} in XTS mode: AES-XTS~\cite{XTS,
XTSComments, NISTXTS}. Unfortunately, using AES-XTS introduces overhead that
drastically impacts system performance and energy consumption. The story of FDE
on Google's Android OS illustrates the problem. Android supported FDE with the
release of Android 3.0, yet it was not enabled by default until Android
6.0~\cite{android-M-mobile-motivation}. Two years prior, Google attempted to
roll out FDE by default on Android 5.0 but had to backtrack. In a statement to
Engadget, Google blamed ``'performance issues on some partner devices' ... for
the backtracking''~\cite{google-engadget}. At the same time, AnandTech reported
a ``62.9\% drop in random read performance, a 50.5\% drop in random write
performance, and a staggering 80.7\% drop in sequential read performance''
versus Android 5.0 unencrypted storage for various
workloads~\cite{android-M-mobile-motivation-2}.

% About: recent advancements
Fortunately, in the last two years there have been significant advancements in
Full Drive Encryption; specifically, the slow AES block cipher can be replaced
with fast {\em stream} ciphers like \encB. For instance, constructions like
Google's HBSH (hash, block cipher, stream cipher, hash)/Adiantum~\cite{Adiantum}
and StrongBox~\cite{StrongBox} bring stream cipher based FDE to devices that do
not or cannot support hardware accelerated AES, generally providing improved
performance along the way~\cite{StrongBox}. A key to efficiently adopting stream
ciphers into the storage layer is to pair it with a Log-structured File System
(LFS) such as F2FS on flash devices. This is because LFSes naturally avoid
writing to the same location multiple times, which requires an expensive
re-keying operation when using stream cipher based FDE.

% About: the need flexibility
Traditionally, FDE at the block level requires enciphering drive contents {\em
homogeneously} (with a single cipher), but these advancements reveal a new
opportunity: storage systems that operate beyond the rigid constraints of prior
work to support flexible {\em heterogeneous} FDE. Yet we find no storage system
or OS that can support such a feature. To see where this might be useful,
suppose the user of an mobile device is required to encrypt their drive with
cipher \encA because it has the useful cryptographic property of being safe to
back up to cloud storage. In exchange for this property, \encA uses a
significant amount of energy when executing. Further suppose this device enters
a critical battery state and the user wishes to conserve as much energy as
possible. It would be beneficial if an energy-aware storage system could also
enter a battery saving state. With heterogeneous FDE, such a flexible
configuration is now achievable: such a system can temporarily switch out the
energy-inefficient \encA for the extremely energy-efficient \encB when accessing
data. Since data encrypted with \encB is not suitable for cloud backups, we
pause cloud backups until we leave the critical battery state and switch ciphers
again---saving even more energy.

% OUR SOLUTION

% About: the system and the benefits
We present \sys, to the best of our knowledge the {\em first} storage system to
provide block level kernel support for heterogeneous FDE. As FDE's impact on
drive performance and energy efficiency depends on a multitude of choices,
different ciphers expose different performance and energy efficiency
characteristics. By supporting heterogeneous FDE, \sys permits cipher choice to
be be viewed as a dynamically configurable parameter as opposed to a static
choice made at format or boot time. As a result, \sys enables the storage system
to adapt to changes that arise at runtime, including: resource availability and
environment, desired security properties, and the OS's energy budget. This
empowers users to perform cipher switching in space and time (\eg using more
secure storage for more sensitive files, and/or dynamically switching between
ciphers for one file), allowing the system to navigate the tradeoff space made
by balancing competing security and latency requirements.  \sys is composed
primarily of three components that realize these benefits. These represent the
three main contributions of this work.

% About: switching strategies
First, we introduce \sysA, a kernel configuration that exposes three switching
models: {\em Forward}, {\em Selective} and {\em Mirrored}. Switching allows  us
to ``re-cipher'' storage units dynamically, enabling us to tradeoff different
performance and security properties of various configurations at runtime. These
switching models define what the I/O layer should do upon the ongoing read/write
I/Os during the switching. These three switching models are motivated by real
case studies. For examples, Forward switching is motivated by the battery case
study above; Selective switching is motivated by cases where users require
certain files (\eg legal documents) to be stored more securely than others; and
Mirrored switching is motivated by scenarios where system administrators want to
change ciphers with zero downtime and without re-initializing the filesystem or
device mapper.

% About: crypts
Second, to support the switching models above, we implement \sysB, a block-level
module that contains encryption implementations (``crypts'') that have been
restructured to support switching. Prior work encrypts targets with a single
cipher---\ie homogeneous FDE---where the choice of cipher implementation is
integrated into the file/block layer system directly~\cite{StrongBox, dmcrypt}.
The key challenge with supporting heterogeneous FDE---where multiple ciphers
coexist on the same storage system---is that different ciphers take disparate
inputs and produce disparate outputs. For example: \encB's implementation
requires a key and nonce, \encA's additionally requires configuration for output
randomization, and \encC's instead requires plaintext and sector information. At
the same time, \encA and \encB's implementations output a keystream that needs
to be XORed with the plaintext to yield the ciphertext while \encC outputs the
ciphertext directly. Thus, we introduce a novel design substantively expanding
prior FDE work by wholly decoupling cipher implementations from the encryption
process. In \sysB, we write hooks that effectively normalize the inputs and
outputs required when switching between ciphers.

% About: tradeoffs
Finally, we present \sysC, a scheme that attempts to {\em quantify} the
tradeoffs in the the rich configuration space of stream ciphers made available
by the above. Using this scheme, we define a tradeoff space of cipher
configurations over competing concerns: total energy use, desirable security
properties, read and write performance (latency), total writable space on the
drive, and how quickly the contents of the drive can converge to a single
encryption configuration. \sysC helps users in understanding their cipher
choices as they come with a variety of performance, energy efficiency, and
security properties in the FDE context.

% About: evaluation
We conclude with a comprehensive evaluation of \sys.
%
First, we show that \sys successfully supports a wide variety of ciphers;
specifically we have integrated {\em \numCiphers ciphers} and a total of {\em
\numConfigs cipher configurations} into \sys in \locTotal lines of code as a
kernel block module~\footnote{\label{ftn:foss}\sys source: \sysURI}, and can act
as an off-the-shelf replacement for dm-crypt. The ciphers are
ChaCha~\cite{ChaCha20}, Freestyle~\cite{Freestyle}, Salsa~\cite{SalsaX}, AES in
counter mode (AES-CTR)~\cite{AESCTR}, Rabbit~\cite{Rabbit},
Sosemanuk~\cite{Sosemanuk}, HC-128~\cite{HC128}.
%
Second, we showcase the benefits of \sys with experiments illustrating \numCases
real-world case studies (\ie storage system responds to critical battery state,
filesystem-agnostic file-level encryption, and datacenter downtime avoidance)
and show the performance/energy tradeoffs.
%
Finally, we perform several benchmarks to demonstrate the flexibility and
performance of the individual ciphers and show the switching overhead of the
three switching models we provide.
