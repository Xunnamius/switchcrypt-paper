\section{Introduction}\label{sec:introduction}

% INTRO

% About: security in general
Security is a very important property of storage systems. Decades of systems and
storage research in this space have looked into security in the context of disk
controllers/FTL and secure hardware \cite{hardware1, hardware2, hardware3,
hardware4}; filesystems \cite{filesystems1, filesystems2, filesystems3,
filesystems4, filesystems5, filesystems6, filesystems7}; network storage and
cloud storage \cite{network1, network2, network3, network4, network5, network6,
network7, network8, network9, network10, network11, network12, network13,
network14, network15, network16}, de-duplication and secure erasure
\cite{erase1, erase2, erase3}; kernels, databases, and other software
\cite{software1, software2, software3, software4}; and many more.

% About: FDE with AES is bad
For local storage, the state of the art for securing data at rest---such as the
contents of a laptop's SSD---is Full Drive Encryption (FDE). Popular FDE
solutions include dm-crypt~\cite{dmcrypt, DmC-Android} for Linux and BitLocker
for Windows~\cite{bitlocker1, bitlocker2}. Behind these implementations is the
industry standard AES \emph{block cipher} in XTS mode: AES-XTS~\cite{XTS,
XTSComments, NISTXTS}. Unfortunately, using AES-XTS introduces overhead that
drastically impacts system performance and energy consumption. The story of FDE
on Google's Android OS illustrates the problem. Android supported FDE with the
release of Android 3.0, yet it was not enabled by default until Android
6.0~\cite{android-M-mobile-motivation}. Two years prior, Google attempted to
roll out FDE by default on Android 5.0 but had to backtrack. In a statement to
Engadget, Google blamed ``'performance issues on some partner devices' ... for
the backtracking''~\cite{google-engadget}. At the same time, AnandTech reported
a ``62.9\% drop in random read performance, a 50.5\% drop in random write
performance, and a staggering 80.7\% drop in sequential read performance''
versus Android 5.0 unencrypted storage for various
workloads~\cite{android-M-mobile-motivation-2}.

% About: recent advancements
Fortunately, in the last two years there have been significant advancements in
Full Drive Encryption; specifically, the slow AES block cipher can be replaced
with fast {\em stream} ciphers like ChaCha20. For instance, constructions like
Google's HBSH (hash, block cipher, stream cipher, hash)/Adiantum~\cite{Adiantum}
and StrongBox~\cite{StrongBox} bring stream cipher based FDE to devices that do
not or cannot support hardware accelerated AES, generally providing improved
performance along the way~\cite{StrongBox}. A key to efficiently adopting stream
ciphers into the storage layer is to pair it with a Log-structured File System
(LFS) such as F2FS on flash devices. This is because LFSes naturally avoid
writing to the same location multiple times, which requires an expensive
re-keying operation when using stream cipher based FDE.

% About: the need flexibility
Traditionally, FDE at the block level requires enciphering drive contents {\em
homogenously} (with a single cipher), but these advancements reveal a new
opportunity: storage systems that operate beyond the rigid constraints of prior
work to support flexible {\em heterogeneous} FDE. Yet we find no storage system
or OS that can support such a feature. To see where this might be useful,
suppose the user of an mobile device is required to encrypt their drive with
cipher \encA because it has the useful cryptographic property of being safe to
back up to cloud storage. In exchange for this property, \encA uses a
significant amount of energy when executing. Further suppose this device enters
a critical battery state and the user wishes to conserve as much energy as
possible. It would be beneficial if an energy-aware storage system could also
enter a battery saving state. With heterogeneous FDE, such a flexible
configuration is now achievable: such a system can temporarily switch the
battery-heavy \encA for the extremely energy efficient \encB when accessing
data. Since data encrypted with \encB is not suitable for cloud backups, we
pause cloud backups until we leave the critical battery state and switch ciphers
again---saving even more energy.

% OUR SOLUTION

% About: the system and the benefits
We present \sys, the {\em first} storage system to provide block level kernel
support for heterogeneous FDE to the best of our knowledge. As FDE's impact on
drive performance and energy efficiency depends on a multitude of choices,
different ciphers expose different performance and energy efficiency
characteristics. \sys allows cipher choice to be be viewed as a key
configuration parameter, as opposed to a static choice at format or boot time.
\sys allows the storage system to adapt to changes that arise at runtime,
including changes in resource availability or environment, desired security
properties, and respecting changing OS energy budgets. \sys allows users to
perform cipher switching in space and time (\eg, more secure files and temporal
switching). \sys allows the software system to navigate the tradeoff space made
by balancing competing security and latency requirements. \hsg{This is where we
list all the benefits; try not to be redundant} To achieve all these benefits,
\sys comes with three important elements, representing the three main
contributions of the paper.

% About: switching strategies
First, we introduce \sysA, a kernel configuration that exposes three types of
switching models: {\em forward}, {\em selective} and {\em mirrored}, to
``re-cipher'' storage units dynamically, allowing us to tradeoff different
performance and security properties of various configurations at runtime. These
switching models define what the I/O layer should do upon the ongoing read/write
I/Os during the switching. These three switching models are motivated from real
case studies. For examples, forward switching is motivated from the battery case
study above; selective switching is motivated from cases where users desire to
have certain files (\eg, legal documents) much more secure than the others; and
mirrored switching is motivated for server-side cases that would like to perform
``encryption upgrade'' without zero downtime.

% About: crypts
Second, to support the switching models above, we implement \sysB, a block-level
module that contains encryption implementations (``crypts'') that have been
restructured to support switching. Prior works mainly implement one cipher
choice and the implementation is very much integrated with the file/block layer
\cite{StrongBox, any-other-works-like-this?}. The key challenge to support
multiple ciphers is that different ciphers take different inputs and produce
different type of outputs. For examples, \encA requires nonce input, \encB
config(??\xxx) and \encC sector info and \encC outputs streams while the others
output \xxx. Thus, we introduce a novel design substantively expanding prior FDE
work by wholly decoupling cipher implementations from the encryption process. In
\sysB, we wrote hooks that manages the required input values and the output
format of different ciphers.

% About: tradeoffs
Finally, we initiate \sysC, a scheme that attempts to {\em quantify} the
tradeoffs in the the rich configuration space of stream ciphers. Using this
scheme, we define a tradeoff space of cipher configurations over competing
concerns: total energy use, desirable security properties, read and write
performance (latency), total writable space on the drive, and how quickly the
contents of the drive can converge to a single encryption configuration. \sysC
helps users in understanding their cipher choices as they come with a variety of
performance, energy efficiency, and security properties in the FDE context.

% About: evaluation
We performed a comprehensive evaluation in several ways.
%
First, we show that \sys successfully supports a wide variety of ciphers;
specifically we have integrated {\em \numCiphers ciphers} and a total of {\em
\numConfigs cipher configurations} into \sys in \locTotal as a kernel block
module (will be open-sourced), and can act as an off-the-shelf replacement for
dm-crypt. The ciphers are ChaCha8 and ChaCha12~\cite{ChaCha20},
Freestyle~\cite{Freestyle}), SalsaX~\cite{SalsaX}, AES in counter mode
(AES-CTR)~\cite{AESCTR}, Rabbit~\cite{Rabbit}, Sosemanuk~\cite{Sosemanuk}, \xxx.
%
Second, we showcase the benefits of \sys with experiments illustrating three
real-world case studies (battery low case, file-level protections, and no
downtime) and show the performance/energy tradeoffs.
%
Finally, we perform several benchmarking to show the performance and \xxx of the
individual ciphers and the switching overhead of the three switching models we
provide \hsg{???}.
