\subsection{Discussion}\label{subsec:des-discussion}

%Here we discuss other details surrounding our three main contributions.

% ---------------------------------- trade score (the x axis in the eval)

\mysub{Trade score.} With \sysC, we do not gain anything useful by comparing
across dimensions; \ie it is unhelpful to compare one crypt's round score to
another crypt's randomization score. However, we can consider the relative round
count and ciphertext randomization scores together. We call this simple sum a
crypt's {\em trade score}. This simple score gives users a quick and intuitive
method to reason about which crypt to use and when: a lower trade score denote
less rounds, less randomization, and higher performance while higher scores
denote more rounds, more randomization, but lower performance.

Trade scores also have the useful feature where scores $\leq1$ can be regarded
as not suitable for encrypting data in motion---a property that is necessary for
safe encrypted backups like in the battery saver mode example (see
\cref{sec:motivation}).

% ---------------------------------- metadata

\mysub{Secure metadata management.} The focus of this work is to implement
mechanisms and policies to perform flexible switching of cipher configurations.
These configurations are built atop an existing block-level encryption module
that manages our data structures and provides cryptographic support. There were
several open source choices for an encryption module: Linux's encryption device
mapper~\cite{dmcrypt,DmC-Android}, encryption-ready F2FS filesystem~\cite{F2FS},
or StrongBox~\cite{StrongBox}. We decided to build \sys atop StrongBox because
it already implements stream cipher based FDE and employs a nugget-based drive
layout.

\sys depends on the data structure management provided by StrongBox, such as its
{\em transaction} and {\em rekeying journals} to avoid overwrite
violations~\cite{StrongBox}, {\em Merkle tree} to avoid rollback and related
attacks~\cite{StrongBox}, {\em monotonic counter} on a trusted hardware to
prevent rollbacks, {\em keycount store} to derive unique encryption keys, and
{\em per-nugget metadata} to store cipher-specific metadata (useful for
ciphertext-expanding ciphers). Every \sys volume also has a ``head'' area that
stores implementation-wide data like the master encryption key or which cipher
is currently active.

While we reuse some StrongBox components, all of these are tightly integrated to
with a special ChaCha20 implementation. We had to untangle this, hence the \sysB
contribution, where we expose structured interfaces allowing cipher implementors
to easily build crypts from any off-the-shelf stream cipher implementation.
Specifically, of the six StrongBox components listed above, only the keycount
store and monotonic counter can be reused as is; the others were rewritten to
support \sysA and \sysB.

% ---------------------------------- security

\mysub{Threat model under switching.} In terms of {\em confidentiality}, an
adversary should not be able to reveal any information about encrypted plaintext
without the proper key. As with prior work, encryption is achieved via a binary
additive approach: cipher output (keystream) is combined with plaintext nugget
contents using XOR, generating confidential outputs, with metadata to track
writes and ensure 1) that pad reuse never occurs and 2) that the system can
recover from crashes into a secure state. Hence, confidentiality is guaranteed.

In terms of {\em integrity}, an adversary should not be able to tamper with
drive state and it go unnoticed. Nugget integrity is guaranteed by StrongBox's
in-memory Merkle tree~\cite{StrongBox}.

Our switching models add an additional security concern: even if we switch the
drive to a new crypt, depending on the switching model, there may still be data
on the drive that was encrypted with an old crypt. Does this create a
confidentiality problem? For the Forward model, this implies any nugget may at
any time be encrypted using the old crypt. For the Mirrored model, the volume is
divided into regions where nuggets use a specific crypt per region; hence, there
exists at least one region using the old crypt. However, like StrongBox, since
nuggets are encrypted and decrypted independently, the confidentiality guarantee
of \sys can be reduced to the confidentiality guarantee of the ``weakest''
cipher~\cite{StrongBox}. Hence, there is no problem unless the user purposely
chooses a broken cipher. For the Selective model, there is never a problem since
the user directly selects which crypt encrypts which nugget every time.

% ---------------------------------- integration
\mysub{Higher-level integration.} \sys expects a higher-level integration/policy
that communicates what kind of switching should be performed and when. An
example is the battery saver scenario, where the OS battery saver application is
expected to notify \sys when to move to an energy-efficient crypt versus a
backup-ready crypt using Forward switching. See \cref*{sec:usecases} for more
integrations.

% ---------------------------------- generality
\mysub{Generality.} \sys can be seen as a drop-in replacement for the popular
Linux dm-crypt layer. For performance reasons, like StrongBox, \sys recommends a
Log-structured File System (LFS) such as F2FS, YAFFS, or LogFS, which are
commonly used for flash devices. The reason for this is that supporting
streaming ciphers is more efficient in storage systems with no in-place
updates~\cite{StrongBox}. Though \sys is a software solution, the same LFS-based
logic can be adopted by hardware like flash devices. For example, the LFS-like
``no in-place updates'' nature of Flash Transition Layers (FTL) would allow
heterogeneous FDE implementations be performant at the device controller level,
allowing the use of other popular in-place update file systems such as ext4,
btrfs, and xfs.
