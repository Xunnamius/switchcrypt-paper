\section{Background and Motivation}\label{sec:motivation}

In this section we discuss recent advancements and limitations in stream cipher
based FDE, then provide \numCases case studies that motivate our work.


% =======================================================
\subsection{The Static FDE Problem}

The major issue with full drive encryption (FDE) is its static nature; users
must commit one FDE configuration without much flexibility. Let us suppose we
have an ARM-based ultra-low-voltage netbook provided to us by our employer who
requires that the drive is fully encrypted at all times and is constantly backed
up to an offsite system. Given that the FDE industry standard is AES-XTS, we
initialize our drive with it. Here, three primary concerns present themselves:
performance, vulnerability and inflexibility.

First, it is well known that AES-XTS adds significant latency and power overhead
to I/O operations, especially on mobile and battery-constrained
devices~\cite{google-engadget, android-M-mobile-motivation,
android-M-mobile-motivation-2}. As this scenario requires the drive encrypted at
all times, we must accept this hit to performance and battery life. Worse, if
our device does not support hardware accelerated AES (which is hardly
ubiquitous) performance can be degraded even further; I/O latency can be as high
as 3--5x~\cite{StrongBox}.

Second, AES-XTS is designed to mitigate threats to drive data ``at rest,'' which
assumes an attacker cannot access snapshots of our encrypted data nor manipulate
our data without those manipulations being immediately obvious to us. However,
access to multiple snapshots of a drive's AES-XTS-encrypted contents presents a
vulnerability---an attacker can passively glean information about the plaintext
over time by contrasting those snapshots, leading to confidentiality violations
in some situations~\cite{XEX, XTS}. In this case, we might want to employ a
strong encryption solution such as stream ciphers, but they are known to perform
worst in many file systems.

Third, our system is battery constrained, placing a cap on our energy budget
that can change at any moment as we transition from line power to battery power
and back. Our system should respond to these changing requirements without
violating any other concerns.


% =======================================================
\subsection{Recent Advancements}

In this paper we categorize two types of FDE: {\em block} and {\em stream}
ciphers. Block ciphers were a defacto solution for storage while stream ciphers
were prevalent in networking \hsg{true??}. Their main differences are \hsg{fill
here, 1 or 2 sentences}. There have been major advancements in the FDE
technology that distances away from the slow block cipher technology and
successfully implements stream ciphers full device encryption \cite{Adiantum,
StrongBox} even without hardware accelerated encryption.

There are two technological shifts that enable this confidential,
high-performance storage with stream ciphers. First, devices today commonly
employ solid-state storage with Flash Translation Layers (FTL), which operate
similarly to Log-structured File Systems (LFS) \cite{LFS, F2FS, NILFS}. The no
in-place update nature of FTL or LFS-like storage allows the overwrite-heavy
stream ciphers to be efficiently implemented (\eg ChaCha with F2FS performs \xxx
times better than with ext4 \cite{StrongBox}). Second, mobile devices now
support trusted hardware, such as Trusted Execution Environments (TEE)
\cite{TrustZone, TEE} and secure storage areas \cite{eMMC-standard} which means
the drive encryption module has access to persistent, monotonically increasing
counters that can be used to prevent rollback attacks when overwrites do occur.


% =======================================================
\subsection{No One-Size-Fits-All Case Studies}

As the recent advancements show that many different ciphers can be implemented
in efficient manner, this opens up an opportunity for the storage layer to
support a wide variety of encryption technologies. Clearly, many case studies
show that there is no one-size-fits-all encryption, and some flexibility is
demanded. Below we present \numCases real-world case studies.

{\em Battery-life saver.} In the example above, in certain situations, the user
might want to prioritize reducing the total energy use. For example, while in
battery low mode, the kernel now switches to a more energy-efficient encryption
and pauses the offline cloud backup momentarily. Users might be willing to
accept this tradeoff and violate the ``backup at all time'' rule, knowing that
within a few hours the user will have access to power (no difference from
internet connection problem). This case study is pervasive \cite{hackernews-???,
many-more??}.

{\em Cipher upgrade without downtime.} From mobile devices, we now turn to
server-side storage that often require encryption upgrade \cite{some-citations}.
A server provider might decide to completely upgrade from a encryption
technology that has been broken into a stronger encryption technology. Ideally
during the switch, the server still can continuously serves users without any
downtime. However, without kernel-level support, the server provider must write
an application level software that performs the whole switching operation and
manually redirects users to the appropriate files.

{\em Select files.} Learning from the wireless literature, it is advocated that
certain files that traverse the Internet or wireless connection in particular,
should be encryption with a more secure way. Example of those files include
government documents \cite{citation-to-this-example}, credit card information
\cite{a}, \xxx \cite{a, b, c}.

