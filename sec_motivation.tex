\section{Background and Motivation}\label{sec:motivation}

In this section we discuss recent advancements and limitations in stream cipher
based Full Drive Encryption (FDE), then provide \numCases case studies that
motivate our work.


% =======================================================
\subsection{The Homogeneous FDE Problem}

The major issue with FDE is its homogeneous nature; users must commit to one FDE
storage configuration at initialization time~\cite{dmcrypt}. Suppose we have an
ultra-low-voltage netbook provided by our employer who requires that the drive
is 1) fully encrypted at all times and 2) backed up to an offsite system
whenever possible. Given that the FDE industry standard is AES-XTS, we consider
initializing our drive with it. At this point, three primary concerns present
themselves: performance, vulnerability and inflexibility.

First, it is well known that AES-XTS adds significant latency and power overhead
to I/O operations, especially on mobile and battery-constrained
devices~\cite{google-engadget, android-M-mobile-motivation,
android-M-mobile-motivation-2}. As this scenario requires the drive encrypted at
all times, we must accept this hit to performance and battery life. Worse, if
our device does not support hardware accelerated AES (which is hardly
ubiquitous) performance can be degraded even further; I/O latency can be as high
as 3--5x~\cite{StrongBox}. There are {\em faster more energy-efficient} stream
ciphers we might consider instead.

Second, AES-XTS is designed to mitigate threats to drive data ``at rest,'' which
assumes an attacker cannot access snapshots of our encrypted data nor manipulate
our data without those manipulations being immediately obvious to us. Access to
multiple snapshots of a drive's AES-XTS-encrypted contents presents a
vulnerability: an attacker can passively glean information about the plaintext
over time by contrasting those snapshots, leading to confidentiality violations
in some situations~\cite{XEX, XTS}. Unfortunately, the backups required by this
scenario function as snapshots for an attacker. There are {\em ciphertext
randomizing} stream ciphers we can use that can be safely backed up, but they're
known to add latency and power overhead like AES-XTS~\cite{Freestyle}.

Third, our device is battery constrained, placing a cap on our energy budget
that can change at any moment as we transition from line power to battery power
and back. Our storage system should respond to these changing requirements,
including reducing energy use when the user wants to conserve power (i.e.
``battery saver mode''), all without violating any other concerns.

These concerns are each best addressed by using different ciphers. Hence, a
homogeneous FDE solution will always sacrifice one concern for the other where a
flexible {\em heterogeneous} approach may not.


% =======================================================
\subsection{Recent Advancements}

Broadly speaking, an FDE storage system can be implemented in two ways: using
{\em block} ciphers or using {\em stream} ciphers. Block ciphers were the de
facto solution for storage while stream ciphers were prevalent in networking and
elsewhere. The relevant difference: encryption using a stream cipher is simpler
and thus faster than a block cipher, especially in software; however, stream
ciphers have their own non-trivial problems (\ie rollback attacks and
overwrites) and are hence not a drop-in replacement for block
ciphers~\cite{StrongBox}. There have been major advancements in the FDE
technology that moves us away from slower block ciphers and towards the adoption
of faster stream cipher based FDE~\cite{Adiantum, StrongBox}.

There are two technological shifts that enable secure, high-performance storage
with stream ciphers. First, devices today commonly employ solid-state storage
with Flash Translation Layers (FTL), which operate similarly to Log-structured
File Systems (LFS)~\cite{LFS, F2FS, NILFS}. The no in-place update
``overwrite-averse'' nature of FTL and LFS-like storage allows stream cipher
based FDE to be efficiently implemented (\eg ChaCha with F2FS performs on
average $1.72$× better than ext4~\cite{StrongBox}). Second, mobile devices now
support trusted hardware, such as Trusted Execution Environments
(TEE)~\cite{TrustZone, TEE} and secure storage areas~\cite{eMMC-standard}, which
means the system has access to persistent, monotonically increasing counters
that can be used to prevent rollback attacks when overwrites do occur.


% =======================================================
\subsection{No One-Size-Fits-All Case Studies}

Recent advancements demonstrate that stream cipher based FDE can be implemented
in efficient manner, which opens up an opportunity for the storage layer to
support multiple encryption technologies in various modes and configurations.
For instance, such a system could switch from using an energy-inefficient but
backup-safe cipher to an energy-efficient cipher to save battery life when
entering battery saver mode and later switch back (and resume uploading backups)
after leaving of battery saver mode. Clearly, there is no one-size-fits-all
encryption solution and some flexibility is demanded. Below we present \numCases
such case studies.

{\em ``Battery saver'' mode.} In situations like the netbook example above, the
user might want to prioritize reducing the total energy use when the battery is
low. While in this ``battery saver mode,'' it would be beneficial if the storage
system could switch to a more energy-efficient cipher and pause the cloud backup
momentarily. A user would be willing to accept this tradeoff knowing that within
a few hours they will have access to power and could resume uploads when the
storage system switches back, which is no different from an internet connection
problem. This case is pervasive~\cite{battery-saver1, battery-saver2,
battery-saver3, battery-saver4, battery-saver5}.

{\em Cipher upgrade without downtime.} From mobile devices, we now turn to
server-side storage where we might require an encryption
upgrade~\cite{upgrade-encryption1,upgrade-encryption2,upgrade-encryption3}. A
server provider might decide to completely upgrade from one encryption
technology (\eg that has been superseded) to another. Ideally, during the
switch, the server would maintain its service-level agreements without any
restarts or downtime. However, without kernel-level support, the server provider
must write application-level software that performs the whole switching
operation and manually redirects users to the appropriate files. It would be
beneficial if this could be handled entirely at the storage layer instead.

{\em Scalable encryption.} Learning from the wireless literature, it is accepted
that wireless systems are subject to variable transmission rates and data
quality requirements when transacting blocks consisting of both high- and
low-priority data that should be encrypted differently~\cite{ScalableSecurity}.
It would be beneficial if our storage system handled the heterogeneous
block-by-block ``scalable'' nature of our encryption requirement. Examples of
data with variable encryption requirements include corporate and government
documents where information is routinely classified and/or redacted, credit card
information in a shopping transaction~\cite{ScalableSecurity}, and file-level
encryption transparent to the filesystem.
