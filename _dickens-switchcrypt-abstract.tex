\documentclass[pageno]{jpaper}

% Replace XXX with the submission number you are given from the ASPLOS submission site.
\newcommand{\asplossubmissionnumber}{XXX}

% The actual title (used elsewhere)
\newcommand{\TITLE}{SwitchCrypt: Navigating Tradeoffs in Stream Cipher Based Full Drive Encryption}

% Shrink space between figs and text
% \setlength{\textfloatsep}{8pt plus 2pt minus 1pt}

% Shrink space between sections/subsections
% \usepackage{titlesec}
% \titlespacing\section{0pt}{8pt plus 2pt minus 1pt}{0pt plus 2pt minus 1pt}
% \titlespacing\subsection{0pt}{6pt plus 2pt minus 1pt}{0pt plus 2pt minus 1pt}
% \titlespacing\subsubsection{0pt}{6pt plus 2pt minus 1pt}{0pt plus 2pt minus 1pt}

%% Some recommended packages
\usepackage[normalem]{ulem}
\usepackage{booktabs}
\usepackage{pgfplotstable}
\usepackage{subcaption}
%\usepackage{natbib}
%\usepackage{epsfig}
%\usepackage[utf8x]{inputenc}
\usepackage{amsmath}
\let\Bbbk\relax
\usepackage{amssymb}
\usepackage{algorithm}
\usepackage{algorithmicx}
\usepackage[noend]{algpseudocode}
\usepackage{enumitem}      % adjust spacing in enums
%\usepackage{subfig}
%\usepackage{caption}
%\usepackage[hyphen]{url}

%enumitem settings
\setlist{
  listparindent=\parindent,
  parsep=0pt,
}

%% Custom packages
\usepackage{minted}
\usepackage{multirow}
\usepackage{rotating}
\usepackage{wrapfig}
\usepackage{tabu}
\usepackage{adjustbox}
\usepackage{pgfplots}
\usepackage{balance}
\usepackage{cleveref}
\usepackage{array}

% Fancy subsection references with cref
\crefformat{section}{\S#2#1#3} % see manual of cleveref, section 8.2.1
\crefformat{subsection}{\S#2#1#3}
\crefformat{subsubsection}{\S#2#1#3}

% Flexible table column specifications
\newcolumntype{C}[1]{>{\centering\arraybackslash}m{#1}}

%% Custom colors that are colorblind safe, print friendly, and photocopy safe
\definecolor{purpleDark}{RGB}{94, 60, 153}
\definecolor{purpleLight}{RGB}{178, 171, 210}
\definecolor{orangeDark}{RGB}{230, 97, 1}
\definecolor{orangeLight}{RGB}{253, 184, 99}

%% Custom fill patterns
\makeatletter
\pgfdeclarepatternformonly[\LineSpace]{custom north east lines}{\pgfqpoint{-1pt}{-1pt}}{\pgfqpoint{\LineSpace}{\LineSpace}}{\pgfqpoint{\LineSpace}{\LineSpace}}%
{
    \pgfsetcolor{\tikz@pattern@color}
    \pgfsetlinewidth{0.4pt}
    \pgfpathmoveto{\pgfqpoint{0pt}{0pt}}
    \pgfpathlineto{\pgfqpoint{\LineSpace + 0.1pt}{\LineSpace + 0.1pt}}
    \pgfusepath{stroke}
}

\pgfdeclarepatternformonly[\LineSpace]{custom north west lines}{\pgfqpoint{-1pt}{-1pt}}{\pgfqpoint{\LineSpace}{\LineSpace}}{\pgfqpoint{\LineSpace}{\LineSpace}}%
{
    \pgfsetcolor{\tikz@pattern@color}
    \pgfsetlinewidth{0.4pt}
    \pgfpathmoveto{\pgfqpoint{0pt}{\LineSpace}}
    \pgfpathlineto{\pgfqpoint{\LineSpace + 0.1pt}{-0.1pt}}
    \pgfusepath{stroke}
}
\makeatother

\newdimen\LineSpace
\tikzset{
    line space/.code={\LineSpace=#1},
    line space=3pt
}

%% Other custom colors
\definecolor{gray}{gray}{0.75}

% Ability to filter the values plotted from tables via `discard if not` key
\pgfplotsset{
    discard if symbol/.style 2 args={
        x filter/.append code={
            \edef\tempa{\thisrow{#1}}
            \edef\tempb{#2}
            \ifx\tempa\tempb
                \def\pgfmathresult{inf}
            \fi
        }
    }
}

\pgfplotsset{
    discard if symbol not/.style 2 args={
        x filter/.append code={
            \edef\tempa{\thisrow{#1}}
            \edef\tempb{#2}
            \ifx\tempa\tempb
            \else
                \def\pgfmathresult{inf}
            \fi
        }
    }
}

\pgfplotsset{
    discard if number/.style 2 args={
        x filter/.append code={
            \ifdim\thisrow{#1} pt=#2pt
                \def\pgfmathresult{inf}
            \fi
        }
    }
}

\pgfplotsset{
    discard if number not/.style 2 args={
        x filter/.append code={
            \ifdim\thisrow{#1} pt=#2pt
            \else
                \def\pgfmathresult{inf}
            \fi
        }
    }
}

%% Shortcuts
\newcommand{\ie}{\textit{i.e., }}
\newcommand{\eg}{\textit{e.g., }}
\newcommand{\CC}{C\nolinebreak\hspace{-.05em}\raisebox{.5ex}{\tiny\bf +}\nolinebreak\hspace{-.10em}\raisebox{.5ex}{\tiny\bf +}}

%% Units
\newcommand{\us}{\,$\mu$s}
\newcommand{\ms}{\,ms}
\newcommand{\KB}{\,KB}
\newcommand{\MB}{\,MB}
\newcommand{\GB}{\,GB}
\newcommand{\MHz}{\,MHz}
\newcommand{\GHz}{\,GHz}

%% Project name
%\newcommand{\SYSTEM}{SwitchCrypt}
\newcommand{\SystemURI}{https://github.com/ananonrepo2/SwitchCrypt}%https://github.com/research/SwitchCrypt}

%% Reference parts of the paper
\newcommand{\figref}[1]{Fig.~\ref{fig:#1}}
\newcommand{\figsref}[2]{Figures~\ref{fig:#1} and~\ref{fig:#2}}
\newcommand{\figrref}[2]{Figures~\ref{fig:#1}--\ref{fig:#2}}
\newcommand{\secref}[1]{Section~\ref{sec:#1}}
\newcommand{\secsref}[2]{Sections~\ref{sec:#1} and~\ref{sec:#2}}
\newcommand{\eqnref}[1]{Eqn.~\ref{eqn:#1}}
\newcommand{\eqnsref}[2]{Equations~\ref{eqn:#1} and~\ref{eqn:#2}}
\newcommand{\eqnrref}[2]{Equations~\ref{eqn:#1}--\ref{eqn:#2}}
\newcommand{\insref}[1]{Instruction~\ref{ins:#1}}
\newcommand{\tblref}[1]{Table~\ref{tbl:#1}}
\newcommand{\appref}[1]{Appendix~\ref{app:#1}}
\newcommand{\algoref}[1]{Algorithm~\ref{algo:#1}}

\makeatletter
\newcommand\footnoteref[1]{\protected@xdef\@thefnmark{\ref{#1}}\@footnotemark}
\makeatother

%% Math
\newcommand{\argmin}{\arg\!\min}
\newcommand{\argmax}{\arg\!\max}
\newcommand{\minimize}{minimize}
\newcommand{\optimize}{optimize}
\newcommand{\ceil}[1]{\lceil #1 \rceil}
\newcommand{\floor}[1]{\lfloor #1 \rfloor}
\newcommand{\st}{s.t.}

%% Custom
\newcommand{\PUNT}[1]{}
\newcommand{\TODO}[1]{\textcolor{gray}{\textbf{\ [TODO:\ #1]\ }}}
\newcommand{\FIX}[1]{\textcolor{red}{\textbf{\ [FIX:\ #1]\ }}}
\newcommand{\hank}[1]{\textcolor{purple}{(Hank: #1)}}

%% For algorithms
\newcommand{\LineComment}[1]{\Statex \hfill\textit{#1}}

%% Configurations

\DeclareCaptionFormat{subfig}{\figurename~#1#2#3}
\DeclareCaptionSubType*{figure}
\captionsetup[subfigure]{format=subfig,labelsep=colon,labelformat=simple}

% Options for pgfplots
\pgfplotsset{compat=1.16,compat/show suggested version=false}
\usetikzlibrary{plotmarks}
\usetikzlibrary{calc}
\pgfplotsset{
    /pgfplots/bar  cycle  list/.style={/pgfplots/cycle  list={%
        {black,fill=black!30!white,mark=none},%
        {black,fill=red!30!white,mark=none},%
        {black,fill=green!30!white,mark=none},%
        {black,fill=yellow!30!white,mark=none},%
        {black,fill=brown!30!white,mark=none},%
    }},
}

% Begin of externalization
\usetikzlibrary{external}
\tikzexternalize[prefix=out/]
\tikzexternalize

% Ensure letter paper
\pdfpagewidth=8.5in
\pdfpageheight=11in

% Further configure pgfplots and tikz
\usetikzlibrary{patterns}
\usepgfplotslibrary{groupplots}
\pgfplotsset{
    every axis label/.append style={font=\small},
    tick label style={font=\small},
}

\pdfstringdefDisableCommands{
    \def\\{}
    \def\unskip{}
    \def\texttt#1{<#1>}
}

\graphicspath{{./figs/}{./data/}}

\pgfkeys{
    /pgf/number format/precision=1,
    /pgf/number format/fixed zerofill=true,
}

\pgfplotsset{
    nodes near coords greater equal only/.style={
        small value/.style={
            /tikz/coordinate,
        },
        every node near coord/.append style={
            check for small values/.code={
                \begingroup
                \pgfkeys{/pgf/fpu}
                \pgfmathparse{\pgfplotspointmeta<#1}
                \global\let\result=\pgfmathresult
                \endgroup
                \pgfmathfloatcreate{1}{1.0}{0}
                \let\ONE=\pgfmathresult
                \ifx\result\ONE
                    \pgfkeysalso{/pgfplots/small value}
                \fi
            },
            check for small values
        },
    },
}

\pgfplotsset{
    nodes near coords greater only/.style={
        small value/.style={
            /tikz/coordinate,
        },
        every node near coord/.append style={
            check for small values/.code={
                \begingroup
                \pgfkeys{/pgf/fpu}
                \pgfmathparse{\pgfplotspointmeta<=#1}
                \global\let\result=\pgfmathresult
                \endgroup
                \pgfmathfloatcreate{1}{1.0}{0}
                \let\ONE=\pgfmathresult
                \ifx\result\ONE
                    \pgfkeysalso{/pgfplots/small value}
                \fi
            },
            check for small values
        },
    },
}

% \author{Bernard Dickens III}
% \affiliation{
%   %\position{Position1}
%   %\department{Department1}              %% \department is recommended
%   \institution{University of Chicago}
%   % \streetaddress{5801 S Ellis Ave}
%   % \city{Chicago}
%   % \state{IL}
%   % \postcode{60637}
%   % \country{USA}
% }
% \email{bd3@cs.uchicago.edu}

% \author{Haryadi S. Gunawi}
% \affiliation{
%   %\position{Position1}
%   %\department{Department1}              %% \department is recommended
%   \institution{University of Chicago}
%   % \streetaddress{5801 S Ellis Ave}
%   % \city{Chicago}
%   % \state{IL}
%   % \postcode{60637}
%   % \country{USA}
% }
% \email{haryadi@cs.uchicago.edu}

% \author{David Cash}
% \affiliation{
%   %\position{Position1}
%   %\department{Department1}              %% \department is recommended
%   \institution{University of Chicago}
%   % \streetaddress{5801 S Ellis Ave}
%   % \city{Chicago}
%   % \state{IL}
%   % \postcode{60637}
%   % \country{USA}
% }
% \email{davidcash@cs.uchicago.edu}

% \author{Henry Hoffmann}
% \affiliation{
%   %\position{Position1}
%   %\department{Department1}              %% \department is recommended
%   \institution{University of Chicago}
%   % \streetaddress{5801 S Ellis Ave}
%   % \city{Chicago}
%   % \state{IL}
%   % \postcode{60637}
%   % \country{USA}
% }
% \email{hankhoffmann@cs.uchicago.edu}


\begin{document}

\title{\TITLE \\ \textbf{Extended Abstract}}

\date{}
\maketitle

\thispagestyle{empty}

ASPLOS'21 will be piloting the submission of {\it Extended
Abstracts}. Your extended abstract — inspired by the model used for
IEEE Micro Top
Picks — should be {\it two pages long}, and it will be submitted separately
from your main paper. The deadline for the extended abstract and the
full paper will be identical. Except for the page limit, all other
formatting and anonymity requirements are identical to those for full
papers. Extended abstracts should be self-contained, though they may
contain references to the full paper.

We \emph{recommend}---but do not require--- that you use
the following organization for your abstract. 
Sections~\ref{sec:motivation} through \ref{sec:key-contributions}
should be a summary of your full paper. Section~\ref{sec:motivation}
motivates the paper; Section~\ref{sec:limitations} describes
limitations of the state of the art, if applicable;
Section~\ref{sec:key-insights} presents the key new insight or
insights of the paper; 
Section~\ref{sec:main-artifacts} presents the main artifacts described
in the paper;  Section~\ref{sec:key-contributions} summarizes the key
results and technical contributions of your paper. Finally,
Section~\ref{sec:why-asplos} should explain why the paper is suitable
for ASPLOS, and Section~\ref{sec:citation} should state what its
citation would be for a ten year test-of-time award. In
Section~\ref{sec:revisions}, you may optionally include a paragraph
describing how your paper has been revised, if it was previously
submitted to another conference.

The extended abstracts must be submitted in printable PDF format and should contain a
{\bf maximum of 2 pages} of single-spaced two-column text, {\bf not
  including references}.  You may include any number of pages for
references, but we suggest you limit your bibliography to
only the most relevant references. The extended
abstracts should use the same formating as the regular papers. If you are using
\LaTeX~\cite{lamport94} to typeset your extended abstract, then we suggest that
you use \href{https://asplos-conference.org/wp-content/uploads/2020/06/asplos21-templates.zip}{this template}.
If you use a different
software package to typeset your paper, then please adhere to the
guidelines given in Table~\ref{table:formatting}.

The \href{ https://asplos-conference.org/wp-content/uploads/2020/06/asplos21-extended-abstract-template.pdf}{sample file} for the
extended abstract includes guidelines the information your abstract
should include.

The references section of your extended abstract will not count
towards the two page limit. We suggest you limit your bibliography to
only the most relevant references~\cite{lamport94}.
The extended abstract should not have an abstract. Start with Section~\ref{sec:motivation}.

\section{Motivation}
\label{sec:motivation}

\begin{itemize}
\item What is the problem your work attacks? Be specific.
\item Why is it an important problem?
\end{itemize}

\vspace{1em}

\noindent
Articulate the importance of this problem to the broader ASPLOS
community, using as little jargon as possible. \emph{Be specific}
about the problem you are addressing; it should be the one that your
paper directly addresses.

\section{Limitations of the State of the Art}
\label{sec:limitations}

\begin{itemize}
\item What is the state of the art in solving this problem today (if any)?
\item What are its limits?
\end{itemize}

\section{Key Insights}
\label{sec:key-insights}

\begin{itemize}
\item What are the one or two key new insights in this paper?
\item How does it advance the state of the art?
\item What makes it more effective than past approaches?
\end{itemize}

\section{Main Artifacts}
\label{sec:main-artifacts}

\begin{itemize}
\item What are the key artifacts presented in your paper: a
  methodology, a hardware design, a software algorithm, an
  optimization or control technique, etc.?
  \item How were your artifacts implemented and evaluated? 
\end{itemize}

\section{Key Results and Contributions}
\label{sec:key-contributions}

\begin{itemize}
  \item What are the most important \emph{one or two} empirical or theoretical
    results of this approach?
  \item What are the contributions that this paper makes to the state of the
    art? List them in an \texttt{itemize} section. Each contribution should be no more than a few sentences long.
  \item Clearly describe its advantages over past work, including how it overcomes their limitations.
\end{itemize}

\section{Why ASPLOS}
\label{sec:why-asplos}

ASPLOS emphasizes multidisciplinary research; explain how this
  paper emphasizes synergy of \emph{two or more ASPLOS areas}: architecture,
  programming languages, operating systems, and related areas (broadly
  interpreted).

\noindent
If you are unsure whether your paper falls within the scope of ASPLOS,
please check with the program chairs -- ASPLOS is a broad,
multidisciplinary conference and encourages new topics.

\section{Citation for Most Influential Paper Award}
\label{sec:citation}

Provide the citation for your paper if it won the Most Influential
Paper award. The citations for past winners are available
\href{https://www.sigarch.org/benefit/awards/acm-sigarch-sigplan-sigops-asplos-influential-paper-award/}{here}.

%  \url{https://rb.gy/hd1hms}).

\section{Revisions}
\label{sec:revisions}

\emph{Optional:} Describe how this paper has been revised, if it was previously submitted to another conference.

\pagebreak
\bibliographystyle{plain}
\bibliography{refs}

\end{document}
