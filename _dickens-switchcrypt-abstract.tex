\documentclass[pageno]{jpaper}

% Replace XXX with the submission number you are given from the ASPLOS submission site.
\newcommand{\asplossubmissionnumber}{XXX}

% The actual title (used elsewhere)
\newcommand{\TITLE}{SwitchCrypt: Navigating Tradeoffs in Stream Cipher Based Full Drive Encryption}

% Shrink space between figs and text
% \setlength{\textfloatsep}{8pt plus 2pt minus 1pt}

% Shrink space between sections/subsections
% \usepackage{titlesec}
% \titlespacing\section{0pt}{8pt plus 2pt minus 1pt}{0pt plus 2pt minus 1pt}
% \titlespacing\subsection{0pt}{6pt plus 2pt minus 1pt}{0pt plus 2pt minus 1pt}
% \titlespacing\subsubsection{0pt}{6pt plus 2pt minus 1pt}{0pt plus 2pt minus 1pt}

%% Some recommended packages
\usepackage[normalem]{ulem}
\usepackage{booktabs}
\usepackage{pgfplotstable}
\usepackage{subcaption}
%\usepackage{natbib}
%\usepackage{epsfig}
%\usepackage[utf8x]{inputenc}
\usepackage{amsmath}
\let\Bbbk\relax
\usepackage{amssymb}
\usepackage{algorithm}
\usepackage{algorithmicx}
\usepackage[noend]{algpseudocode}
\usepackage{enumitem}      % adjust spacing in enums
%\usepackage{subfig}
%\usepackage{caption}
%\usepackage[hyphen]{url}

%enumitem settings
\setlist{
  listparindent=\parindent,
  parsep=0pt,
}

%% Custom packages
\usepackage{minted}
\usepackage{multirow}
\usepackage{rotating}
\usepackage{wrapfig}
\usepackage{tabu}
\usepackage{adjustbox}
\usepackage{pgfplots}
\usepackage{balance}
\usepackage{cleveref}
\usepackage{array}

% Fancy subsection references with cref
\crefformat{section}{\S#2#1#3} % see manual of cleveref, section 8.2.1
\crefformat{subsection}{\S#2#1#3}
\crefformat{subsubsection}{\S#2#1#3}

% Flexible table column specifications
\newcolumntype{C}[1]{>{\centering\arraybackslash}m{#1}}

%% Custom colors that are colorblind safe, print friendly, and photocopy safe
\definecolor{purpleDark}{RGB}{94, 60, 153}
\definecolor{purpleLight}{RGB}{178, 171, 210}
\definecolor{orangeDark}{RGB}{230, 97, 1}
\definecolor{orangeLight}{RGB}{253, 184, 99}

%% Custom fill patterns
\makeatletter
\pgfdeclarepatternformonly[\LineSpace]{custom north east lines}{\pgfqpoint{-1pt}{-1pt}}{\pgfqpoint{\LineSpace}{\LineSpace}}{\pgfqpoint{\LineSpace}{\LineSpace}}%
{
    \pgfsetcolor{\tikz@pattern@color}
    \pgfsetlinewidth{0.4pt}
    \pgfpathmoveto{\pgfqpoint{0pt}{0pt}}
    \pgfpathlineto{\pgfqpoint{\LineSpace + 0.1pt}{\LineSpace + 0.1pt}}
    \pgfusepath{stroke}
}

\pgfdeclarepatternformonly[\LineSpace]{custom north west lines}{\pgfqpoint{-1pt}{-1pt}}{\pgfqpoint{\LineSpace}{\LineSpace}}{\pgfqpoint{\LineSpace}{\LineSpace}}%
{
    \pgfsetcolor{\tikz@pattern@color}
    \pgfsetlinewidth{0.4pt}
    \pgfpathmoveto{\pgfqpoint{0pt}{\LineSpace}}
    \pgfpathlineto{\pgfqpoint{\LineSpace + 0.1pt}{-0.1pt}}
    \pgfusepath{stroke}
}
\makeatother

\newdimen\LineSpace
\tikzset{
    line space/.code={\LineSpace=#1},
    line space=3pt
}

%% Other custom colors
\definecolor{gray}{gray}{0.75}

% Ability to filter the values plotted from tables via `discard if not` key
\pgfplotsset{
    discard if symbol/.style 2 args={
        x filter/.append code={
            \edef\tempa{\thisrow{#1}}
            \edef\tempb{#2}
            \ifx\tempa\tempb
                \def\pgfmathresult{inf}
            \fi
        }
    }
}

\pgfplotsset{
    discard if symbol not/.style 2 args={
        x filter/.append code={
            \edef\tempa{\thisrow{#1}}
            \edef\tempb{#2}
            \ifx\tempa\tempb
            \else
                \def\pgfmathresult{inf}
            \fi
        }
    }
}

\pgfplotsset{
    discard if number/.style 2 args={
        x filter/.append code={
            \ifdim\thisrow{#1} pt=#2pt
                \def\pgfmathresult{inf}
            \fi
        }
    }
}

\pgfplotsset{
    discard if number not/.style 2 args={
        x filter/.append code={
            \ifdim\thisrow{#1} pt=#2pt
            \else
                \def\pgfmathresult{inf}
            \fi
        }
    }
}

%% Shortcuts
\newcommand{\ie}{\textit{i.e., }}
\newcommand{\eg}{\textit{e.g., }}
\newcommand{\CC}{C\nolinebreak\hspace{-.05em}\raisebox{.5ex}{\tiny\bf +}\nolinebreak\hspace{-.10em}\raisebox{.5ex}{\tiny\bf +}}

%% Units
\newcommand{\us}{\,$\mu$s}
\newcommand{\ms}{\,ms}
\newcommand{\KB}{\,KB}
\newcommand{\MB}{\,MB}
\newcommand{\GB}{\,GB}
\newcommand{\MHz}{\,MHz}
\newcommand{\GHz}{\,GHz}

%% Project name
%\newcommand{\SYSTEM}{SwitchCrypt}
\newcommand{\SystemURI}{https://github.com/ananonrepo2/SwitchCrypt}%https://github.com/research/SwitchCrypt}

%% Reference parts of the paper
\newcommand{\figref}[1]{Fig.~\ref{fig:#1}}
\newcommand{\figsref}[2]{Figures~\ref{fig:#1} and~\ref{fig:#2}}
\newcommand{\figrref}[2]{Figures~\ref{fig:#1}--\ref{fig:#2}}
\newcommand{\secref}[1]{Section~\ref{sec:#1}}
\newcommand{\secsref}[2]{Sections~\ref{sec:#1} and~\ref{sec:#2}}
\newcommand{\eqnref}[1]{Eqn.~\ref{eqn:#1}}
\newcommand{\eqnsref}[2]{Equations~\ref{eqn:#1} and~\ref{eqn:#2}}
\newcommand{\eqnrref}[2]{Equations~\ref{eqn:#1}--\ref{eqn:#2}}
\newcommand{\insref}[1]{Instruction~\ref{ins:#1}}
\newcommand{\tblref}[1]{Table~\ref{tbl:#1}}
\newcommand{\appref}[1]{Appendix~\ref{app:#1}}
\newcommand{\algoref}[1]{Algorithm~\ref{algo:#1}}

\makeatletter
\newcommand\footnoteref[1]{\protected@xdef\@thefnmark{\ref{#1}}\@footnotemark}
\makeatother

%% Math
\newcommand{\argmin}{\arg\!\min}
\newcommand{\argmax}{\arg\!\max}
\newcommand{\minimize}{minimize}
\newcommand{\optimize}{optimize}
\newcommand{\ceil}[1]{\lceil #1 \rceil}
\newcommand{\floor}[1]{\lfloor #1 \rfloor}
\newcommand{\st}{s.t.}

%% Custom
\newcommand{\PUNT}[1]{}
\newcommand{\TODO}[1]{\textcolor{gray}{\textbf{\ [TODO:\ #1]\ }}}
\newcommand{\FIX}[1]{\textcolor{red}{\textbf{\ [FIX:\ #1]\ }}}
\newcommand{\hank}[1]{\textcolor{purple}{(Hank: #1)}}

%% For algorithms
\newcommand{\LineComment}[1]{\Statex \hfill\textit{#1}}

%% Configurations

\DeclareCaptionFormat{subfig}{\figurename~#1#2#3}
\DeclareCaptionSubType*{figure}
\captionsetup[subfigure]{format=subfig,labelsep=colon,labelformat=simple}

% Options for pgfplots
\pgfplotsset{compat=1.16,compat/show suggested version=false}
\usetikzlibrary{plotmarks}
\usetikzlibrary{calc}
\pgfplotsset{
    /pgfplots/bar  cycle  list/.style={/pgfplots/cycle  list={%
        {black,fill=black!30!white,mark=none},%
        {black,fill=red!30!white,mark=none},%
        {black,fill=green!30!white,mark=none},%
        {black,fill=yellow!30!white,mark=none},%
        {black,fill=brown!30!white,mark=none},%
    }},
}

% Begin of externalization
\usetikzlibrary{external}
\tikzexternalize[prefix=out/]
\tikzexternalize

% Ensure letter paper
\pdfpagewidth=8.5in
\pdfpageheight=11in

% Further configure pgfplots and tikz
\usetikzlibrary{patterns}
\usepgfplotslibrary{groupplots}
\pgfplotsset{
    every axis label/.append style={font=\small},
    tick label style={font=\small},
}

\pdfstringdefDisableCommands{
    \def\\{}
    \def\unskip{}
    \def\texttt#1{<#1>}
}

\graphicspath{{./figs/}{./data/}}

\pgfkeys{
    /pgf/number format/precision=1,
    /pgf/number format/fixed zerofill=true,
}

\pgfplotsset{
    nodes near coords greater equal only/.style={
        small value/.style={
            /tikz/coordinate,
        },
        every node near coord/.append style={
            check for small values/.code={
                \begingroup
                \pgfkeys{/pgf/fpu}
                \pgfmathparse{\pgfplotspointmeta<#1}
                \global\let\result=\pgfmathresult
                \endgroup
                \pgfmathfloatcreate{1}{1.0}{0}
                \let\ONE=\pgfmathresult
                \ifx\result\ONE
                    \pgfkeysalso{/pgfplots/small value}
                \fi
            },
            check for small values
        },
    },
}

\pgfplotsset{
    nodes near coords greater only/.style={
        small value/.style={
            /tikz/coordinate,
        },
        every node near coord/.append style={
            check for small values/.code={
                \begingroup
                \pgfkeys{/pgf/fpu}
                \pgfmathparse{\pgfplotspointmeta<=#1}
                \global\let\result=\pgfmathresult
                \endgroup
                \pgfmathfloatcreate{1}{1.0}{0}
                \let\ONE=\pgfmathresult
                \ifx\result\ONE
                    \pgfkeysalso{/pgfplots/small value}
                \fi
            },
            check for small values
        },
    },
}

% \author{Bernard Dickens III}
% \affiliation{
%   %\position{Position1}
%   %\department{Department1}              %% \department is recommended
%   \institution{University of Chicago}
%   % \streetaddress{5801 S Ellis Ave}
%   % \city{Chicago}
%   % \state{IL}
%   % \postcode{60637}
%   % \country{USA}
% }
% \email{bd3@cs.uchicago.edu}

% \author{Haryadi S. Gunawi}
% \affiliation{
%   %\position{Position1}
%   %\department{Department1}              %% \department is recommended
%   \institution{University of Chicago}
%   % \streetaddress{5801 S Ellis Ave}
%   % \city{Chicago}
%   % \state{IL}
%   % \postcode{60637}
%   % \country{USA}
% }
% \email{haryadi@cs.uchicago.edu}

% \author{David Cash}
% \affiliation{
%   %\position{Position1}
%   %\department{Department1}              %% \department is recommended
%   \institution{University of Chicago}
%   % \streetaddress{5801 S Ellis Ave}
%   % \city{Chicago}
%   % \state{IL}
%   % \postcode{60637}
%   % \country{USA}
% }
% \email{davidcash@cs.uchicago.edu}

% \author{Henry Hoffmann}
% \affiliation{
%   %\position{Position1}
%   %\department{Department1}              %% \department is recommended
%   \institution{University of Chicago}
%   % \streetaddress{5801 S Ellis Ave}
%   % \city{Chicago}
%   % \state{IL}
%   % \postcode{60637}
%   % \country{USA}
% }
% \email{hankhoffmann@cs.uchicago.edu}


\begin{document}

\title{\TITLE}

\date{}
\maketitle

\thispagestyle{empty}

\hank{Page 1 of this document is an unborken wall of text.  That needs
  to change.  I suggest either adding a picture or finding subsections
  you can add, or break the first section into two: do a motivation
  and a limitations section.  Or do all of these, but this needs to be
  more digestible and welcoming to readers.}

\section{Motivation: The Limitations of Prior Work}
\label{sec:motivation}

Recent work on Full Drive Encryption shows that stream ciphers achieve
significantly improved performance over block ciphers while offering
stronger security guarantees. However, optimizing for performance
often conflicts with other key concerns like energy usage and desired
security properties. \hank{Really need citations at the end of each of
  the first two sentences.  That would always be the case, but even
  more important here as this will be the only thing reviewed in round
  1.} This paper presents SwitchCrypt, a software mechanism that
navigates the tradeoff space made by balancing competing security and
latency requirements via \emph{cipher switching} in space or time. Our
key insight in achieving low-overhead switching is leveraging the
overwrite-averse behavior of underlying solid-state storage and the
per-unit independence of our drive/filesystem layout. This observation
coupled with our software mechanisms allow us to trade throughput for
reduced energy use and/or certain security properties.

As an example: suppose we have a low-voltage netbook provided to us by
our employer. As this is an enterprise device, our employer requires
that 1) our drive is fully encrypted at all times and 2) our encrypted
data is constantly backed up to an offsite system. The industry
standard in full drive encryption is AES-XTS, so we initialize our
drive with it. Given these requirements, three primary concerns
present themselves: (1) Latency, (2) AES assumptions, and (3) power
source.

\paragraph*{Latency} it is well known that FDE using AES-XTS adds
significant latency overhead to I/O operations, especially on mobile
and low-power devices~\cite{google-engadget,
  android-M-mobile-motivation, android-M-mobile-motivation-2}. To keep
our drive encrypted at all times with AES-XTS means we must accept
this hit to performance. Worse, if our device does not support
hardware accelerated AES, performance can be degraded even further;
I/O latency can be as high as 3--5x~\cite{StrongBox}.  Hardware
accelerated AES is hardly ubiquitous, and the existence of myriad
devices that do not support it cannot simply be ignored, hence
Google's investment in Adiantum---an FDE solution for ``the next
billion users'' of low-cost devices without support for hardware
accelerated AES~\cite{Adiantum}.

\paragraph*{AES Assumptions} AES-XTS only mitigates threats to drive
data ``at rest,'' which assumes an attacker cannot access snapshots of
our encrypted data nor manipulate it without those manipulations being
immediately obvious to us. With access to multiple snapshots of a
drive's AES-XTS-encrypted contents, an attacker can passively glean
information about the plaintext over time by contrasting those
snapshots, leading to confidentiality violations in some
situations~\cite{XEX, XTS}. \hank{The ``in some situations'' really
  weakens the overall point.  Is tha qualifier really necessary here?
  If so, it would probably be better to explain those situations and
  argue that they are common.} Similarly, an attacker that can
manipulate encrypted bits without drawing our attention will corrupt
any eventual plaintext, violating data integrity and, in the worst
case, influencing the behavior of software. \hank{Citation on the
  previous sentence is necessary.}  Unfortunately, in real life, data
rarely remains ``at rest'' in these ways. In our example, our employer
requires we back up the contents of our drive to some offsite backup
service; this service will receive periodic snapshots of the encrypted
state of our drive, violating our ``at rest'' invariant. These backups
occur at a layer above the drive controller, meaning any encryption
happening at the Flash Translation Layer (FTL) or below is irrelevant.

\paragraph{Power Source} Our system is battery constrained, placing a
cap on our energy budget that can change at any moment as we
transition from line power to battery power and back. Our system
should respond to these changing requirements without violating any
other concerns.

To alleviate the latency concern, we can choose a stream cipher like
ChaCha20 rather than the AES-XTS block cipher. Using StrongBox, an
encryption driver built for ChaCha20-based FDE, we can achieve on
average a 1.7x speedup and a commensurate reduction in energy
use~\cite{StrongBox}.

When it comes to the security concern, StrongBox solves both the snapshot and
integrity problems by 1) never writing data encrypted with the same key to the
same location and 2) tracking drive state using a Merkle tree and monotonic
counter supported by trusted hardware to prevent rollbacks. This ensures data
manipulations cannot occur and guarantees confidentiality even when snapshots
are compared regardless of the stream cipher used. Unfortunately, restoring from
a backup necessitates a forced rollback of drive state, potentially opening us
back up to confidentiality-violating snapshot comparison
attacks~\cite{StrongBox}.

To truly address the security concern requires a cipher with an additional
security property: \emph{ciphertext randomization}. Without ciphertext
randomization, an attacker can map plaintexts to their ciphertext counterparts
during snapshot comparison, especially if they can predict what might be written
to certain drive regions. However, with ciphertext randomization, a cipher will
output a different ``random'' ciphertext even when given the same key, nonce,
and plaintext; this means, even after a forced rollback of system state and/or
legitimate restoration from a backup, comparing future writes is no longer
violates confidentiality because each snapshot will always consist of different
ciphertext regardless of the plaintext being encrypted or the state of the
drive. Using Freestyle~\cite{Freestyle}, a ChaCha20-based stream cipher that
supports ciphertext randomization, we can guarantee data confidentiality in this
way. So, we switch from StrongBox to an encryption driver that supports
Freestyle.

Unfortunately, like AES-XTS, Freestyle has significant overhead compared to the
original ChaCha20. In exchange for stronger security properties, Freestyle is up
to 1.6x slower than ChaCha20, uses more energy, has a higher initialization
cost, and expands the ciphertext which reduces total writeable drive
space~\cite{Freestyle}.

Further complicating matters is our final concern: a constrained energy budget.
Our example system is battery constrained. Even if we accepted trading off
performance, drive space, and energy for security in some situations, in other
situations we might prioritize reducing total energy use. For example, when we
trigger ``battery saver'' mode, we expect our device to conserve as much energy
as possible. It would be ideal if our device could pause backups and the
encryption driver could switch from the ciphertext-randomizing Freestyle
configuration back to our high performance energy-efficient ChaCha20
configuration when conserving energy is a top priority, and then switch back to
the Freestyle configuration when we connect to a charger and backups are
eventually resumed.

To overcome these limitations, we present SwitchCrypt, a device mapper
that can trade off between different cipher configurations without
compromising security or performance or requiring the device be
restarted. With prior work, the user must select a static operating
point in the energy-security-latency space at initialization time and
hope it is optimal across all workloads and use case. If they choose
the Freestyle configuration, their device will be slower and battery
hungry even when they are not backing up. If they choose the more
performant ChaCha20 configuration, they risk confidentiality-violating
snapshot comparison attacks.  SwitchCrypt solves this problem by
encrypting data with the high performance ChaCha20 when the battery is
low and switching to Freestyle when plugged in and syncing with the
backup service resumes.

\section{Challenges}
\label{sec:key-challenges}

To trade off between different cipher configurations, we must address three key
challenges. First, we must determine what cipher configurations are most
desirable in which contexts and why. This requires we \emph{quantify} the
desirable properties of these configurations. Second, we must have some way to
encrypt independent storage units with any one of these configurations. This
requires we \emph{decouple} cipher implementations from the encryption process
used in prior work. Third, we need to determine when to re-encrypt those units,
which configuration to use, and where to store the output, all with minimal
overhead. This requires we implement efficient cipher \emph{switching
strategies}.

\textbf{Quantifying the properties traded off between ciphers.} To
obtain a space of configurations that we might reason about, it is
necessary to compare certain properties of stream ciphers useful in
the FDE context. However, different ciphers have a wide range of
security properties, performance profiles, and output characteristics,
including those that randomize their outputs and those with
non-length-preserving outputs---\ie{the cipher outputs more data than
  it takes in}. To address this, we propose a framework for
quantitative cipher comparison in the FDE context; we use this
framework to define our configurations.

\textbf{Decoupling ciphers from the encryption process.} To flexibly switch
between configurations in SwitchCrypt requires a generic cipher interface. This
is challenging given the variety of inputs required by various stream ciphers,
the existence of non-length-preserving ciphers, and other differences. We
achieve the required generality by defining independent storage units called
\emph{nuggets}; we borrow this terminology from prior work (see
\cite{StrongBox}) to easily differentiate our logical blocks (nuggets) from
physical drive and other storage blocks. And since they are independent, we can
use our interface to select any configuration to encrypt or decrypt any nugget
at any point.

\textbf{Implementing efficient switching strategies.} Finally, to determine when
to switch a nugget's cipher and to where we commit the output, we implement a
series of high-level policies we call \textit{cipher switching strategies}.
These strategies leverage our generic cipher interface and flexible drive layout
to selectively ``re-cipher'' groups of nuggets, whereby the key and the cipher
used to encrypt/decrypt a nugget are switched at runtime. These strategies allow
SwitchCrypt to move from one configuration point to another or even settle on
optimal configurations wholly unachievable with prior work. The challenge here
is to accomplish this while minimizing overhead.

\section{Contributions and Results}
\label{sec:key-contributions}

\begin{itemize}
  \item Define a scheme to \emph{quantify} the usefulness of each cipher based
  on key security properties relevant to FDE in context. Using this scheme, we
  define a tradeoff space of cipher configurations over competing concerns:
  total energy use, desirable security properties, read and write performance
  (latency), total writable space on the drive, and how quickly the contents of
  the drive can converge to a single encryption configuration.

  \item Introduce a novel design substantively expanding prior FDE work by
  wholly \emph{decoupling} cipher implementations from the encryption process.

  \item Develop the idea of cipher switching using \emph{switching strategies}
  to ``re-cipher'' storage units dynamically, allowing us to tradeoff different
  performance and security properties of various configurations at runtime.
\end{itemize}

We implement SwitchCrypt and three switching strategies---\emph{Forward},
\emph{Selective}, and \emph{Mirrored}---to dynamically transition the system
between configurations using our generic cipher interface. We then study the
utility of cipher switching through three case studies where latency, energy,
and desired security properties change over time. In one study, where we require
the filesystem to react to a shrinking energy budget, we find that SwitchCrypt
achieves up to a 3.3x total energy use reduction compared to a static approach
using only the Freestyle stream cipher. In another case, where we allow the user
to manually switch between ChaCha20 and Freestyle stream ciphers dynamically, we
achieve at minimum a 3.1x reduction in read latency compared to static
approaches.


\hank{I think you need a ``Why ASPLOS'' section here that briefly
  argues this work is interdisciplinary and at the intersetcion of OS
  and hardware (and explain why that is true).}

\pagebreak
\bibliographystyle{plain}
\bibliography{refs}

\end{document}
