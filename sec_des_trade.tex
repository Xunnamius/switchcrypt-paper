\subsection{\sysC: Evaluation Framework}\label{subsec:des-trade}

% About: background
To reason about when to switch cipher configurations, users must have a way to
compare and evaluate their utility in the novel context of heterogeneous FDE.
However, different ciphers have a wide range of desirable security properties,
performance profiles, and output characteristics, including those that randomize
their outputs and those with non-length-preserving outputs. To address this
need, we propose \sysC, a novel evaluation framework that quantifies stream
ciphers according to three features relevant to heterogeneous FDE: {\em relative
rounds}, which gives users a sense for the strength of the implementation;
{\em ciphertext randomization}, which gives a sense for how safe a cipher's
output is to backup/snapshot (see \cref{sec:motivation}); and {\em ciphertext
expansion}, which gives users a sense for how a cipher configuration impacts the
total available drive space. Taken together, these features illustrate a rich
tradeoff space of cipher configurations optimizing for different combinations of
energy-efficiency, security, and performance concerns.

% ---------------------------------- RR

\mysub{Relative Rounds (Rounds).} The ciphers we examine in this paper are all
constructed around the notion of {\em rounds}, where a higher number of rounds
(and/or possibly longer key) is positively correlated with a higher resistance
to brute force given no fatal related-key or other
attacks~\cite{ChaCha-Cryptanalysis}. Hence, this feature, scored from 0 (lowest
relative round count) to 1 (highest relative round count), represents how many
rounds a cipher executes relative to other implementations of the same
algorithm. For instance: ChaCha8 (score: 0) is a reduced-round version of
ChaCha12 (score: 0.5), which is a reduced-round version of the standardized
ChaCha20 (score: 1), all using the ChaCha
algorithm~\cite{ChaCha20,ChaCha-Cryptanalysis}.

\hank{My previous comments about the scoring apply to this draft as well. This score is presented arbitrarily.  Why does it start at 0?  Is ChaCha12 infinitely harder to break than ChaCha8 and only half as hard to break as ChaCha20?  Does breaking the cipher have nothing to do with this score?  If so, then what does it mean? }

% ---------------------------------- CR

\mysub{Ciphertext Randomization (Randomization).} A cipher with ciphertext
randomization generates different ciphertexts non-deterministically given the
same key, nonce, and plaintext. This makes it much more difficult to execute
chosen-ciphertext attacks (CCA), key re-installation attacks, XOR-based
cryptanalysis and other comparison attacks, and other confidentiality-violating
schemes where the ciphertext is in full control of the adversary
~\cite{Freestyle}. This property is useful in cases where we cannot prevent the
same key, nonce, and plaintext from being reused, such as with data ``in
motion'' (see \cref{sec:motivation}). Ciphers without this property---such as
ChaCha20 and AES-XTS on which prior work is based---are trivially broken when
key-nonce-plaintext 3-tuples are reused. In StrongBox, this is referred to as an
``overwrite condition'' or simply ``overwrite''~\cite{StrongBox}.

Though there are many ways to achieve ciphertext randomization, the ciphers
included in our analysis implement it using a random number of rounds for each
block of the message where the exact number of rounds are unknown to the
receiver a priori~\cite{Freestyle}. In configuring the minimum and maximum
number of rounds used per block in this non-deterministic mode of operation, we
can customize the computational burden an attacker must bear by choosing lower
or higher minimums and maximums. Hence, this is not a binary feature; scores
range from 0 (no ciphertext randomization support) to 1 (lowest minimum and
maximum rounds per block) to 3 (highest minimum and maximum rounds per block).

\hank{Again, why is 3 the highest here, where 1 is the highest before?
Can we compare across these dimensions?  For example, is it more
important to have output randomization, so that is why the max in this
dimension is higher than in round count?}

% ---------------------------------- CE

\mysub{Ciphertext Expansion (Expansion).} A cipher that exhibits ciphertext
expansion is non-length-preserving: it outputs more ciphertext than was
originally input as plaintext. This can cause major problems in any FDE context.
For instance, cryptosystems that rely on AES-XTS (e.g. Linux's
dm-crypt~\cite{dmcrypt} and Microsoft's BitLocker~\cite{bitlocker1}) or ChaCha
(e.g. StrongBox~\cite{StrongBox}, Google's Adiantum~\cite{Adiantum}) have
storage layouts that hold length-preserving ciphertext output as an invariant,
making ciphers that do not exhibit this property incompatible with their
implementations; yet, ciphertext expansion is often {\em but not always} a
necessary side-effect of ciphertext randomization. Further,
non-length-preserving ciphers {\em use more drive space} than length-preserving
ciphers when encrypting the same plaintext, meaning users will have less total
available drive space.

The ciphers included in our analysis that exhibit ciphertext expansion have an
overhead of around 1.56\% per plaintext message block~\cite{Freestyle}. Even a
single byte of additional ciphertext vs plaintext would make a cipher
inappropriate for use with prior work. Hence, this is a binary feature in that a
cipher either outputs ciphertext of the same length as its plaintext input or it
does not. A cipher scores either a 0 if it {\em is not} length-preserving in
this way or a 1 if the ciphertext is always the same length as the plaintext.

% About: describing the table
Table \cref{tbl:trade} shows the result of using \sysC to evaluate 12 cipher
crypts (see \cref{subsec:des-crypts}) as described below. Note that we have
included the special crypt {\em Freestyle*} in our analysis, which is a vacuous
configuration of our Freestyle implementation (see \cref{sec:impl}) using the
minimum possible ciphertext randomization options.

\begin{table}[t]
    \begin{center}
        \small
        \centering
        \begin{tabular}{@{}c|cccc@{}}
            \textbf{Cipher} & \textbf{Rounds} & \textbf{Randomization} & \textbf{Expansion} \\
            \midrule
            ChaCha8         & 0           & 0           & 1           \\
            ChaCha12        & 0.5         & 0           & 1           \\
            ChaCha20        & 1           & 0           & 1           \\
            Salsa8          & 0           & 0           & 1           \\
            Salsa12         & 0.5         & 0           & 1           \\
            Salsa20         & 1           & 0           & 1           \\
            HC128           & 0           & 0           & 1           \\
            HC256           & 1           & 0           & 1           \\
            Freestyle (F)   & 0           & 2           & 0           \\
            Freestyle (B)   & 0.5         & 2.5         & 0           \\
            Freestyle (S)   & 1           & 3           & 0           \\
        \end{tabular}
    \end{center}

    \mycaption{tbl:trade}{Quantifying ciphers}{\sysC, our framework for
    classifying ciphers as discussed in \cref{subsec:des-trade}.}
\end{table}

% ------------- floating: \begin{floatingtable}[r]{ % note the open curly
% bracket \begin{tabular}{...} \end{tabular}....} % note the closed bracket
% here \mycaption{fig-pass}{x}{x.} \end{floatingtable}


\hank{I think this really needs a couple sentences of summary that
show that these scores are useful.  Maybe use the battery saver case
study again and say how you can read the scores to find a cipher that
works for high battery and another that would be appropriate for low
battery (or at least would be appropriate for times when the backups
aren't happening since you probably don't want to talk about
performance here since this scoring section is only about security).}
