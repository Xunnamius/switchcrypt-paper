\subsection{\sysC: An Evaluation Framework}\label{subsec:des-trade}

% About: background
To reason about which crypt to use and when, non-expert users must have a way to
evaluate and compare their utility in the novel context of heterogeneous FDE.
However, different ciphers have a wide range of desirable security properties,
performance profiles, and output characteristics, including those that randomize
their outputs and those with non-length-preserving outputs. To simplify
comparison, we propose \sysC, a novel evaluation framework that quantifies
crypts according to three features relevant to generic heterogeneous FDE: {\em
relative rounds}, which gives users a sense for the strength of the crypt; {\em
ciphertext randomization}, which gives a sense for how safe a crypt's output is
when encrypting ``data in motion'' (\eg to backup/snapshot; see
\cref{sec:motivation}); and {\em ciphertext expansion}, which gives users a
sense for how a crypt will impact total available drive space.

\cref{subsec:des-discussion} describes how we might apply \sysC to reason about
crypt choice.

\begin{table}[t]
    \begin{center}
        \small
        \centering
        \begin{tabular}{@{}c|cccc@{}}
            \textbf{Cipher} & \textbf{Rounds} & \textbf{Randomization} & \textbf{Expansion} \\
            \midrule
            ChaCha8         & 0           & 0           & 1           \\
            ChaCha12        & 0.5         & 0           & 1           \\
            ChaCha20        & 1           & 0           & 1           \\
            Salsa8          & 0           & 0           & 1           \\
            Salsa12         & 0.5         & 0           & 1           \\
            Salsa20         & 1           & 0           & 1           \\
            HC128           & 0           & 0           & 1           \\
            HC256           & 1           & 0           & 1           \\
            Freestyle (F)   & 0           & 2           & 0           \\
            Freestyle (B)   & 0.5         & 2.5         & 0           \\
            Freestyle (S)   & 1           & 3           & 0           \\
        \end{tabular}
    \end{center}

    \mycaption{tbl:trade}{Quantifying ciphers}{\sysC, our framework for
    classifying ciphers as discussed in \cref{subsec:des-trade}.}
\end{table}

% ------------- floating: \begin{floatingtable}[r]{ % note the open curly
% bracket \begin{tabular}{...} \end{tabular}....} % note the closed bracket
% here \mycaption{fig-pass}{x}{x.} \end{floatingtable}


% ---------------------------------- RR

\mysub{Relative Rounds (Rounds).} The cipher implementations we examine in this
paper are all constructed around the notion of {\em rounds}, where a higher
number of rounds (and/or larger key space) is positively correlated with a
higher resistance to brute force given no fatal related-key or other
attacks~\cite{ChaCha-Cryptanalysis}; in other words: more rounds implies
stronger encryption. Groups of crypts implement the same cipher but use
different configurations, \eg the crypts for the ChaCha cipher configured for 8
rounds (ChaCha8), 12 rounds (ChaCha12), and the standard 20 rounds (ChaCha20).
These crypts are scored across a discrete uniform distribution from 0,
representing the lowest round configuration (\eg 8-round ChaCha) to 1,
representing the highest round configuration (\eg 20-round ChaCha).

Table \cref{tbl:trade} shows the relative round score for twelve crypts grouped
by the four ciphers they implement: ChaCha, Salsa, HC-X, and Freestyle. ChaCha
and Salsa each have three crypts that differ only by round count, yielding the
scores {\em 0, 0.5, and 1}. HC-X has two, yielding {\em 0 and 1}. Freestyle has
four crypts, but only three of them differ by round count ({\em Freestyle (F)}
and {\em Freestyle*} use the same number of rounds). If a crypt is the singular
implementation of some cipher, it receives a 1.

% ---------------------------------- CR

\mysub{Ciphertext Randomization (Randomization).} A cipher employing ciphertext
randomization generates different ciphertexts non-deterministically given the
same key, nonce, and plaintext. This makes it much more difficult to execute
chosen-ciphertext attacks (CCA), key re-installation attacks, XOR-based
cryptanalysis, and other confidentiality-violating schemes where the ciphertext
is in full control of the adversary ~\cite{Freestyle}. This property is useful
in cases where we cannot prevent the same key, nonce, and plaintext from being
reused, such as with data ``in motion'' (see \cref{sec:motivation}). Ciphers
without this property---such as ChaCha20 and AES-XTS on which prior work is
based---are trivially broken when key-nonce-plaintext 3-tuples are reused. In
StrongBox, this is referred to as an ``overwrite''~\cite{StrongBox}.

Though there are many ways to achieve ciphertext randomization, the crypts
included in our analysis implement it using a random number of rounds for each
block of the message where the exact number of rounds are unknown to the
receiver a priori~\cite{Freestyle}. In configuring the minimum and maximum
number of rounds used per block in this non-deterministic mode of operation, we
can customize the computational burden an attacker must bear by choosing lower
or higher minimums and maximums.

Table \cref{tbl:trade} shows the ciphertext randomization score for twelve
crypts grouped by the four ciphers they implement: ChaCha, Salsa, HC-X, and
Freestyle. ChaCha, Salsa, and HC-X {\em do not employ ciphertext randomization
at all}, so the crypts implementing them score a 0. Freestyle {\em does} employ
ciphertext randomization (see \cref{sec:impl}), so the four crypts implementing
them are scored from 1 (least randomized output) to 2 (most randomized output)
across an even distribution similar to relative rounds yielding the scores {\em
1, 1.34, 1.67, and 2}.

% ---------------------------------- CE

\mysub{Ciphertext Expansion (Expansion).} A cipher that exhibits ciphertext
expansion is non-length-preserving: it outputs more bytes of ciphertext after
encryption than were input. This can cause major problems in any FDE context.
For instance, cryptosystems that are tightly coupled to AES-XTS (e.g. Linux's
dm-crypt~\cite{dmcrypt} and Microsoft's BitLocker~\cite{bitlocker1}) or ChaCha
(e.g. StrongBox~\cite{StrongBox}, Google's Adiantum~\cite{Adiantum}) have
storage layouts that hold length-preserving ciphertext output as an invariant,
making ciphers that do not exhibit this property incompatible with their
implementations; yet, ciphertext expansion is often (but not always) a necessary
side-effect of ciphertext randomization. Further, since non-length-preserving
ciphers {\em use more drive space} than length-preserving ciphers when
encrypting the same plaintext, users will have less total writable drive space.

The crypts included in our analysis that exhibit ciphertext expansion have an
overhead of around 1.56\% per plaintext message block~\cite{Freestyle}. Hence,
this is a binary feature in that a crypt either outputs ciphertext of the same
length as its plaintext input or it does not. A crypt scores a 0 if it {\em is
not} length-preserving or a 1 if it is, as shown in \cref{tbl:trade}.
