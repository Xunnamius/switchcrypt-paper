\section{Implementation}

Our SwitchBox implementation consists of 9,491 lines of C code; our test suite
consists of 6,077 lines of C code. All together, our solution is comprised of
15,568 lines of C code.

SwitchBox uses OpenSSL version 1.1.0h and LibSodium version 1.0.12 for its
AES-XTS and AES-CTR implementations. Open source ARM NEON optimized
implementations of ChaCha are provided by Floodyberry~\cite{Floodyberry}. The
Freestyle cipher reference implementation is from the original Freestyle
paper~\cite{Freestyle} and, thanks to its output randomization and natural
resistance to brute force, does not require the same metadata management
overhead (\ie{transaction journal updates, re-keying on overwrites}) as ciphers
without these desirable security properties. The eSTREAM Profile 1 cipher
implementations are from the open source libestream cryptographic
library~\cite{libestream} by Lucas Clemente Vella. The Merkle Tree
implementation is from the Secure Block Device~\cite{SBD}. SwitchBox
implementation is publicly available open-source\footnote{\SystemURI}.

We implement SwitchBox on top of the BUSE~\cite{BUSE} virtual block device,
using it as our mock device controller. BUSE is a thin (200 LoC) wrapper around
the standard Linux Network Block Device (NBD). BUSE allows an operating system
to transact block I/O requests to and from virtual block devices exposed via
domain socket.

For experimental purposes, our implementation makes the choice of ciphers
binary: either the system wants SwitchBox to access the backing store using the
primary cipher or the secondary cipher. However, there is no technical
limitation preventing various different nuggets encrypted with three, four, or
more unique ciphers from co-existing on the backing store.

Cipher switching is a cross-cutting concern---for instance, the desire to switch
ciphers could come from an unrelated process in user space. For both spatial and
temporal cipher switching, when SwitchBox should be using the primary cipher or
the secondary cipher to interact with the backing store, this intent is
communicated via POSIX message queue in our implementation. It is not a
requirement of SwitchBox that a POSIX message queue be used over any other
method of IPC so long as SwitchBox can be notified asynchronously when the wider
system desires one cipher be active over the other.

A production-ready implementation would be greatly simplified by adding an
``intent'' parameter to the POSIX \textit{read()} and \textit{write()} system
calls, allowing SwitchBox to more exactly map individual I/O operations to
specific areas of the backing store when spatially switching. \PUNT{This is
especially important when considering the selective switching strategy; a
production-ready implementation supporting selective switching would need to
differentiate between metadata operations belonging to the filesystem (should be
mirrored across all partitions) and actual end-user data (should be selectively
read from and written to nuggets in specific partitions).}

\subsection{Backing Store Initialization}

To operate securely, SwitchBox must be seeded with random data initially rather
than have the backing store consist of all zeroes. This is a one-time cost paid
during initialization and has no tangible effect on performance.

If nuggets are not seeded with random data initially, any write operation into
an ``empty'' nugget might leak information or constitute an overwrite when the
predictable state of the backing store (\ie{initialized to all zeroes}) leads to
an overwrite-style condition.
