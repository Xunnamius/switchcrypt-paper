\section{Implementation}\label{sec:implementation}

Our SwitchCrypt implementation consists of 9,491 lines of C code; our test suite
consists of 6,077 lines of C code. All together, our solution is comprised of
15,568 lines of C code. Our implementation is also publicly available
open-source\footnote{\SystemURI}.

SwitchCrypt uses OpenSSL version 1.1.0h and LibSodium version 1.0.12 for its
AES-XTS and AES-CTR implementations. Open source ARM NEON optimized
implementations of ChaCha are provided by Floodyberry~\cite{Floodyberry}. The
Freestyle cipher reference implementation is from the original Freestyle
paper~\cite{Freestyle}. The eSTREAM Profile 1 cipher implementations are from
the open source libestream cryptographic library~\cite{libestream} by Lucas
Clemente Vella. The Merkle Tree implementation is from the Secure Block
Device~\cite{SBD}.

We implement SwitchCrypt on top of the BUSE~\cite{BUSE} virtual block device,
using it as our mock device controller. BUSE is a thin (200 LoC) wrapper around
the standard Linux Network Block Device (NBD). BUSE allows an operating system
to transact block I/O requests to and from virtual block devices exposed via
domain socket.

\PUNT{For experimental purposes, our implementation makes the choice of ciphers
binary: either the system wants SwitchCrypt to access the backing store using
the active cipher or the inactive cipher. However, there is no technical
limitation preventing various different nuggets encrypted with three, four, or
more unique ciphers.

We use POSIX message queues to indicate intent to switch. A production-ready
implementation would be greatly simplified by adding an ``intent'' parameter to
the POSIX \textit{read()} and \textit{write()} system calls, allowing
SwitchCrypt to more exactly map individual I/O operations to specific areas of
the backing store when spatially switching. We simulate this with IPC.
\TODO{This intent parameter could also be a security score or something right?
You should spell out exactly what that intent parameter means and maybe change
its name so its use is clearer.} \PUNT{This is especially important when
considering the selective switching strategy; a production-ready implementation
supporting selective switching would need to differentiate between metadata
operations belonging to the filesystem (should be mirrored across all regions)
and actual end-user data (should be selectively read from and written to nuggets
in specific regions).}

Further, to operate securely, SwitchCrypt must be seeded with random data
initially rather than have the backing store consist of all zeroes. This is a
one-time cost paid during initialization and has no tangible effect on
performance. SSDs that support ISE can accomplish this with minimal wear.}

Finally, as the Freestyle cipher is highly configurable, we implement it in
three different configurations: a ``fast'' mode with parameters
\\\texttt{FreestyleFast($R_{min}$=$8$, $R_{max}$=$20$, $H_I$=$4$, $I_C$=$8$)}, a
``balanced'' mode with parameters \texttt{FreestyleBalanced($R_{min}$=$12$,
$R_{max}$=$28$, $H_I$=$2$, $I_C$=$10$)}, and a ``secure'' mode with parameters
\texttt{FreestyleSecure($R_{min}$=$20$, $R_{max}$=$36$, $H_I$=$1$, $I_C$=$12$)}.

Thanks to Freestyle's output randomization (see \secref{design}), we can skip
the overhead of tracking, detecting, and handling overwrites when nuggets are
using it, offsetting the 1.6x to 3.2x performance loss of using Freestyle versus
ChaCha20~\cite{Freestyle}.

\PUNT{\subsubsection{Implementing Cipher Switching}

A naive implementation is trivial (\eg{execute the chosen strategy on every I/O
operation}), this navigation must occur with acceptable overhead by preserving
performance wherever possible. The cryptographic driver provides such a
mechanism, tying together cipher switching strategies and the generic stream
cipher API. \TODO{Which cryptographic driver? You need to clarify if you are
talking about a piece of our design or something we are using from prior work.}

In the cases of Mirrored and Selective switching, we use offset to determine in
which area of the backing store receives I/O.

In the case of Forward switching, it is tempting to implement it such that a
nugget is completely re-ciphered during I/O every time its metadata indicates
that it was previously encrypted using a non-active cipher. However, such a
naive implementation can have disastrous effects on performance. \TODO{It is not
clear why that would be disastrous.}

First, a nugget is considered \emph{pristine} if it has not had any data written
into it yet. SwitchCrypt determines if a nugget is pristine by checking the
state of the transaction journal for that nugget. All nuggets start out pristine
with metadata indicating that they are to be encrypted and decrypted by the
initially active cipher. This is true \emph{even if the nugget has not been
written to yet}. This means, on read and write operations after a different
cipher becomes the active cipher configuration using a naively implemented
forward switching strategy, every write operation will trigger a re-keying,
which carries significant overhead.

Our solution was to divide forward switching into \emph{soft re-ciphering} and
\emph{hard re-ciphering}. During soft re-ciphering, only the nugget's metadata
is changed to indicate that the nugget can be encrypted and decrypted with the
newly active cipher configuration but \emph{without actually re-ciphering the
nugget data itself}. This keeps the nugget in its pristine condition, preserving
SwitchCrypt's ability to write data into it without triggering a costly
re-keying operation every time. On the other hand, during hard re-cipher, the
nugget's metadata is changed to match the active cipher configuration \emph{and}
the nugget data is encrypted using the new cipher.

\TODO{Maybe repeat that mirrored relies on someone to implement the fast secure
erase, so that you can read the fast region until it is time to panic and then
you quickly erase?  Are there other uses of mirrored?}}

\PUNT{When using forward switching other that 0-forward, \ie{N-forward} where $N
> 0$, only read operations are allowed to trigger hard re-ciphering for nuggets
other than the currently active nugget. This is still not enough to preserve our
performance advantage, however, as I/O operations can span multiple nuggets, and
attempting to take advantage of spatial locality after interacting with every
nugget is counterproductive. Hence, only the last nugget touched by a read
operation will trigger the more aggressive N-forward behavior if $N > 0$. These
considerations have the effect of 1) preserving our performance advantage and 2)
allowing more aggressive N-forward behavior (where $N > 0$) to take advantage of
spatial locality during read-heavy workloads to result in a further performance
advantage (see: \secref{evaluation}).}
