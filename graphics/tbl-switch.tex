\begin{table}[t]
    \begin{center}
        \small
        \centering
        \begin{tabular}{lll}
                            & {\bf Read} & {\bf Write} \\\hline
            {\bf Forward}   & 1st read: $D_{C_1}$                & $E_{C_2}$ (only if necessary) \\
                            & 2nd read: $D_{C_1}$ then $E_{C_2}$ & \\
                            & Nth read: $D_{C_2}$                & \\
            \hline
            {\bf Mirrored}  & When migrating: $D_{C_1}$  & I/O is duplicated: \\
                            & from \cone's region        & $E_{C_1}$ to \cone's region \\
                            & After migrating: $D_{C_2}$ & and $E_{C_2}$ to \ctwo's region \\
                            & from \ctwo's region        & \\
            \hline
            {\bf Selective} & Encrypted with \cone: $D_{C_1}$ & Should use \cone: $E_{C_1}$ \\
                            & Encrypted with \ctwo: $D_{C_2}$ & Should use \ctwo: $E_{C_2}$ \\
            \hline
        \end{tabular}
    \end{center}

    \mycaption{tbl:switch}{Switching Models and Re-ciphering}{This table
    summarizes for each switching model what happens on I/Os to a nugget when
    \sys switches from cipher \cone to cipher \ctwo (see
    \cref{subsec:des-switch}). ``Read'' denotes returning data during the
    switch. ``Write'' denotes committing new data. $E_X$ denotes encryption (\ie
    re-ciphering) and $D_X$ denotes decryption using cipher $X$.}
\end{table}

% ------------- floating: \begin{floatingtable}[r]{ % note the open curly
% bracket \begin{tabular}{...} \end{tabular}....} % note the closed bracket
% here \mycaption{fig-pass}{x}{x.} \end{floatingtable}
