\section{SwitchBox System Design}\label{sec:design}

Todo!

\subsection{Quantifying the Security Dimension}

To reason about trading off the security guarantees provided by various ciphers,
a loose ranking of these ciphers must be established. This ranking ranges from
\emph{offers more security} to \emph{offers less security}. For our purposes, we
arrived at three key features that, when scored, give us something akin to an
ideal distribution:

\begin{itemize}

 \item \emph{Output randomization.} A cipher that exhibits output randomization
 can output ciphertext non-deterministically given the same input, which is
 extremely useful for FDE\@. This is a binary feature in that a cipher either
 outputs deterministically or it does not. A cipher with output randomization
 scores a 1 for this feature while a cipher without it scores a 0.

 \item \emph{Resistance to cryptanalysis.} A cipher that is resistant to
 cryptanalysis can resist theoretical cryptanalytical attacks such as
 known-plaintext and chosen-plaintext attacks, offline key-guessing attacks, et
 cetera. Scores for this feature range from 0 to 1, where 0.5 represents
 standard resistance to cryptanalysis for stream ciphers in the general case\@.

 \item \emph{Round count vs standard.} The ciphers we examine in this research
 are all constructed around the notion of "rounds," where a higher number of
 rounds implies a stronger confidentiality guarantee. This feature represents
 how many rounds the cipher executes compared to the accepted "standard" round
 count for that cipher. For instance, ChaCha8 is a reduced round version of the
 standardized ChaCha20. Variants are distributed evenly from 0-1. For instance,
 ChaCha8 scores 0, ChaCha12 scores 0.5, and ChaCha20 scores 1\@.

\end{itemize}

The ideal rank distribution would be a continuous “security gradient”.
Unfortunately, in reality, the best we can do is a discrete and non-uniform
distribution, but for the purpose of quantifying our tradeoff space, it is
adequate.
