\section{Related Work} \label{sec:related}

The standard approach to FDE, using AES-XTS, introduces significant overhead.
Recently, it has been established that encryption using \emph{stream ciphers}
for FDE is faster than using AES~\cite{StrongBox}, but in practice it is
non-trivial. Prior work explores several approaches: a non-deterministic CTR
mode (Freestyle~\cite{Freestyle}), a length-preserving ``tweakable
super\\-pseudorandom permutation'' (Adiantum~\cite{Adiantum}), and a stream cipher
in a binary additive (XOR) mode leveraging LFS overwrite-averse behavior to
prevent overwrites (StrongBox~\cite{StrongBox}).

Unlike StrongBox and other work, which focuses on optimizing performance despite
re-ciphering due to overwrites, SwitchCrypt maintains overwrite protections
while abstracting the idea of re-ciphering nuggets out into cipher switching;
instead of myopically pursuing a performance win, we can pursue energy/battery
and security wins as well.

Further, trading off security for energy, performance, and other concerns is not
a new research area~\cite{ScalableSecurity, WolterReinecke, ZengChow1,
HaleemEtAl, LiOmiecinski, Merkel4, Merkle3}. Goodman et al. introduced
selectively decreasing the security of some data to save
energy~\cite{ScalableSecurity}. However, their approach is designed for
communication and only considered iteration/round count, thus it did not
anticipate the need for SwitchCrypt's generic interface, switching strategies,
or security scores. Wolter and Reinecke study approaches to quantifying security
in several contexts~\cite{WolterReinecke}. This study anticipates the value of
dynamically switching ciphers but proposes no mechanisms to enable this in FDE.
Similarly, companies like LastPass and Google have explored performance-security
tradeoffs. Google's Adiantum (above) uses a less secure reduced round version of
ChaCha~\cite{Adiantum}. While not an FDE solution, LastPass has dealt with
scaling the number of iterations of PBKDF\#2, trading performance for security
during login sessions~\cite{LastPass}.
