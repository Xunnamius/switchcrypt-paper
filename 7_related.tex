\section{Related Work}\label{sec:related}

The standard approach to FDE, using AES-XTS, introduces significant overhead.
Recently, it has been established that encryption using \emph{stream ciphers}
for FDE is faster than using AES~\cite{StrongBox, AnotherPaper1, AnotherPaper2}.
However, when used naively in drive encryption, stream ciphers are widely known
to be vulnerable to ``overwrite attacks'' like pad reuse and
rollback~\cite{KatzLindell, StrongBox}. To enable FDE using stream ciphers,
prior work explores several approaches:

\begin{itemize}
   \item Use a non-deterministic CTR mode with specially designed cipher and
   filesystem (Freestyle~\cite{Freestyle}).
   \item Use a length-preserving ``tweakable super-pseudorandom permutation''
   construction with nonce-accepting stream cipher (Adiantum~\cite{Adiantum}).
   \item Use a stream cipher in a binary additive (XOR) mode leveraging metadata
   management and Log-structured File Systems' (LFS) overwrite-averse behavior
   to prevent overwrites (StrongBox~\cite{StrongBox}).
\end{itemize}

Unlike StrongBox and other work, which focuses on optimizing performance despite
re-ciphering due to overwrites, SwitchCrypt maintains overwrite protections
while abstracting the idea of re-encrypting nuggets out into cipher switching;
instead of myopically pursuing a performance win, this allows the trade off of
various cipher configurations dynamically.

However, trading off security for energy, performance, and other concerns is not
a new idea~\cite{ScalableSecurity, WolterReinecke, ZengChow1, ZengChow2,
HaleemEtAl, LiOmiecinski}. For instance, Goodman et al. (1998) introduced
trading decreased security for saving energy dissipated to encrypt a bit during
wireless communication~\cite{ScalableSecurity}. Similar in intent to VSRs,
Goodman minimizes energy consumption by separating low-priority communications
from high-priority and encrypting them differently. Goodman et al.'s approach is
designed for communication and only considered changing key lengths, thus it did
not anticipate the need for SwitchCrypt's generic API, switching strategies, or
security scores. Further, Wolter and Reinecke study performance and security
tradeoffs, exploring approaches to quantifying security in several
contexts~\cite{WolterReinecke}. \TODO{Again, need to say something more to
differentiate us.  Something like: "This study anticipates the value of
dynamically switching ciphers but proposes no mechanisms to actually enable this
switch."}

In the wild, companies like Google~\cite{AndroidM} and Apple~\cite{iOSFDE} have
explored performance-security tradeoffs in hardware when considering FDE
support. Google's Adiantum uses a less secure reduced round version of
ChaCha~\cite{Adiantum}. LastPass (LogMeIn) has dealt with scaling the number of
iterations of its PBKDF\#2, trading performance for security~\cite{LastPass}. \TODO{Need a summary sentence here, too.  Again, need to make it clear that what we are doing is related to this work, but solved many novel challenges.}
