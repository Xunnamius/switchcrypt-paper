\section{Related Work}\label{sec:related}

The standard approach to FDE, using AES-XTS, introduces significant overhead.
Within the last year and a half it has been established that encryption using
\emph{stream ciphers} for FDE is faster than using AES~\cite{StrongBox,
AnotherPaper1, AnotherPaper2}. However, when used naively in drive encryption,
stream ciphers are widely known to be vulnerable to ``overwrite attacks'' like
pad reuse and rollback~\cite{KatzLindell, StrongBox}. To enable FDE using stream
ciphers, prior work explores several approaches:

\begin{itemize}
   \item Use a non-deterministic CTR mode with specially designed cipher and
   filesystem (Freestyle~\cite{Freestyle}).
   \item Use a length-preserving ``tweakable super-pseudorandom permutation''
   construction with nonce-accepting stream cipher (Adiantum~\cite{Adiantum}).
   \item Use a stream cipher in a binary additive (XOR) mode leveraging metadata
   management and Log-structured File Systems' (LFS) overwrite-averse behavior
   to prevent overwrites (StrongBox~\cite{StrongBox}).
\end{itemize}

Unlike StrongBox and other work, which focuses on optimizing performance despite
re-ciphering due to overwrites, SwitchCrypt maintains overwrite protections
while abstracting the idea of re-encrypting nuggets out into cipher switching;
instead of myopically pursuing a performance win, this allows the trade off of
various cipher configurations dynamically.

However, trading off security for energy, performance, and other concerns is not
a new idea~\cite{ScalableSecurity, WolterReinecke, ZengChow1, ZengChow2,
HaleemEtAl, LiOmiecinski}. For instance: \textit{An Energy/Security Scalable
Encryption Processor Using an Embedded Variable Voltage DC/DC Converter},
published by Goodman et al. in 1998, introduced trading security for decreased
energy dissipated to encrypt a bit~\cite{ScalableSecurity}. Similar in intent to
VSRs and the Selective strategy (see \secref{usecases}), they minimize energy
consumption by separating low-priority data from high-priority and encrypting
them differently. Similarly, Wolter and Reinecke study performance and security
tradeoffs, exploring approaches to quantifying security~\cite{WolterReinecke}.

In the wild, companies like Google~\cite{AndroidM} and Apple~\cite{iOSFDE} have
explored performance-security tradeoffs in hardware when considering FDE
support. Google's Adiantum uses a less secure reduced round version of
ChaCha~\cite{Adiantum}. LastPass (LogMeIn) has dealt with scaling the number of
iterations of its PBKDF\#2, trading performance for security~\cite{LastPass}.
