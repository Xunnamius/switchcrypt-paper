\section{Related Work}\label{sec:related}

The standard approach to FDE, using AES-XTS, introduces significant overhead.
Recently, it has been established that encryption using \emph{stream ciphers}
for FDE is faster than using AES~\cite{StrongBox, AnotherPaper1, AnotherPaper2},
but in practice it is non-trivial. Prior work explores several approaches: a
non-deterministic CTR mode (Freestyle~\cite{Freestyle}), a length-preserving
``tweakable super-pseudorandom permutation'' (Adiantum~\cite{Adiantum}), and a
stream cipher in a binary additive (XOR) mode leveraging LFS overwrite-averse
behavior to prevent overwrites (StrongBox~\cite{StrongBox}).

Unlike StrongBox and other work, which focuses on optimizing performance despite
re-ciphering due to overwrites, SwitchCrypt maintains overwrite protections
while abstracting the idea of re-encrypting nuggets out into cipher switching;
instead of myopically pursuing a performance win, we trade off various cipher
configurations dynamically.

However, trading off security for energy, performance, and other concerns is not
a new idea~\cite{ScalableSecurity, WolterReinecke, ZengChow1, ZengChow2,
HaleemEtAl, LiOmiecinski}. For instance, Goodman et al. introduced selectively
decreasing security to save energy~\cite{ScalableSecurity}. Similar in intent to
VSRs, Goodman et al. minimizes energy consumption by separating low-priority
communications from high-priority and encrypting them differently. Goodman et
al.'s approach is designed for communication and only considered iteration/round
count, thus it did not anticipate the need for SwitchCrypt's generic API,
switching strategies, or security scores. Further, Wolter and Reinecke study
approaches to quantifying security in several contexts~\cite{WolterReinecke}.
This study anticipates the value of dynamically switching ciphers but proposes
no mechanisms to enable this in FDE.

Companies like LastPass and Google have explored performance-security tradeoffs.
Google's Adiantum uses a less secure reduced round version of
ChaCha~\cite{Adiantum}; LastPass has dealt with scaling the number of iterations
of PBKDF\#2, trading performance for security~\cite{LastPass}. \TODO{Need a
summary sentence here, too.  Again, need to make it clear that what we are doing
is related to this work, but solved many novel challenges.}
