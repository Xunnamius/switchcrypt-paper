\section{Conclusion}\label{sec:conclusion}
%This paper advocates for a more agile approach to full 
%drive encryption where the storage system can dynamically 
%alter the tradeoffs between security and latency (or energy) 
%for data at rest.  To support this vision of agile encryption,
%we have proposed an API that allows mulitple stream ciphers, with 
%different input and output characteristics to be composed in a 
%generic manner.  We have identified three strategies for using 
%this API to switch ciphers dynamically, but with low overhead.  
%We have also proposed a scoring method for determining when to 
%use one cipher over another. Our case studies show how different 
%strategies can be used to achieve different goals in practice.
%We believe that agile encryption will become increasingly important 
%as successful operation of computer systems increasingly requires 
%balancing seemingly conflicting operational requirements.  We hope 
%that this work inspires further research into achieving this balance, 
%including both mechanisms and policies for exposing and navigating 
%the underlying tradeoff spaces.

Stream ciphers are fast and offer strong security properties, but optimizing for
performance often conflicts with other key concerns. In this paper we presented
SwitchCrypt to navigate the security and latency/energy tradeoff space via
\emph{cipher switching} in space and time. We provided empirical results
demonstrating the conditions under which different switching strategies are
optimal and explored four related cases. In all cases, we found that SwitchCrypt
achieves reduced energy usage and I/O latency compared to static approaches.
Perhaps more importantly, though, in all cases SwitchCrypt achieves flexibility
that is simply not possible with prior designs.  We hope that this work inspires
further research into balancing competing security, energy, and performance
tradeoffs, including both mechanisms and policies for exposing and navigating
these tradeoff spaces.

\PUNT{\TO\DO{Add David to acknowledgements once we get out of anon reviews.}}
