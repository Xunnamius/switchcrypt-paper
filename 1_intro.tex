\section{Introduction}\label{sec:introduction}

There are several important concerns modern filesystems must balance. Paramount
among them are: security of data at rest; performance, \ie{I/O latency or
throughput}); and overall energy budget. However, these concerns exist in
contention with one another.

The state of the art for securing data at rest is Full-Drive Encryption
(FDE)\footnote{The common term is full-\emph{disk} encryption, but this work
targets SSDs, so we use \emph{drive}.}. FDE employs a \emph{cryptographic
cipher} to encrypt, decrypt, and authenticate backing store contents on the fly.
Understandably, this comes with considerable impact on system performance and
energy use. \TODO{Android M FDE example from Google. Perhaps another from
someone else~\cite{Google}.} Yet, end users expect their device to be secure
\emph{and} performant \emph{and} energy efficient, hence these concerns must be
balanced with respect to the most general use case.

Prior work on balancing the FDE's impact on performance and energy efficiency
focuses on choice of cipher, as different ciphers express different performance
and efficiency characteristics while offering contrasting security guarantees.
The standard choice in FDE is the AES \emph{block cipher} in XTS mode (AES-XTS).
The recent StrongBox project, however, shows that \emph{stream ciphers} achieve
improved performance versus AES-XTS and other block ciphers while providing a
stronger integrity guarantee, at the cost of a small amount of metadata
overhead~\cite{StrongBox}. This prior work specifically uses the ChaCha20 stream
cipher~\cite{ChaCha20} as its AES replacement.

But ChaCha20 is not the only stream cipher we can configure for use with FDE.
Other stream ciphers exist with a plethora of performance characteristics and
security properties beyond integrity guarantees. These include: the reduced
round ChaCha8~\cite{ChaCha8}, SalsaX~\cite{SalsaX}, mimicking a stream cipher
with AES in counter mode (AES-CTR)~\cite{AES-CTR}, Rabbit~\cite{Rabbit},
Sosemanuk~\cite{Sosemanuk}, Freestyle~\cite{Freestyle}, et cetera. Thus, the
choice of stream cipher can be viewed as a key \emph{configuration} parameter
for FDE.

%Prior work requires a static configuration choice made offline at compile- or
%boot-time---a single cipher configuration is chosen to work generally well for
%all possible scenarios that might arise at runtime.
The cipher configuration providing the fastest I/O throughput is almost always
different than the configuration providing the strongest security guarantees,
and both may differ from the configuration that uses the least amount of energy.
\TODO{Hank: I still think we could improve the flow in this paragraph, but we
can fine tune later.}

While configuration parameters allow filesystems to balance security, energy,
and performance, a static,  compile- or boot-time configuration choice cannot
adapt to changes that might arise at runtime.  Such changes include resource
availability, runtime environment, or even the required latency, security and
energy. Hence, any static configuration will unilaterally sacrifice one concern
for another until the filesystem can be shut down and restarted.

Fortunately, the StrongBox project provides deeper insights: the performance
improvement of stream ciphers over AES-XTS comes from 1) the "append-mostly"
behavior of underlying log-structured filesystems (LFS) and 2) the organization
of the backing store into discrete units a layer above the block and sector
storage paradigm. We can leverage these insights to create a novel tradeoff
mechanism for \emph{online cipher configuration switching} to meet dynamic
latency, energy, and security goals with minimal overhead.

\TODO{We present: \SYSTEM{}! We demonstrate \SYSTEM{}'s effectiveness on ...; we
measure using ...; we further demonstrate with use cases ...}

\TODO{Short explanation of the remaining structure of the paper.}
