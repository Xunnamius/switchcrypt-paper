\section{Introduction}\label{sec:introduction}

There are several important concerns modern filesystems must balance.
Paramount among them are: security of data at rest; performance, \ie{I/O latency
or throughput}); and overall energy budget. However, these concerns exist in
contention with one another. \TODO{More?}

The state of the art for securing data at rest is Full-Drive Encryption
(FDE)\footnote{The common term is full-\emph{disk} encryption, but this work
targets SSDs, so we use \emph{drive}.}. FDE employs a complex algorithm or
cryptographic \emph{cipher} to encrypt, decrypt, and authenticate backing store
contents on the fly. Understandably, this comes with considerable impact on
system performance and energy use. \TODO{Android M FDE example from Google.
Perhaps another from someone else. \cite{Google}} Yet, end users expect their
device to be secure \emph{and} performant \emph{and} energy efficient, hence
these concerns must be balanced with respect to the most general use case.

Prior work on balancing the FDE's impact on performance and energy efficiency
focuses on choice of cipher, as different ciphers express different performance
and efficiency characteristics and offer contrasting security guarantees from
one another. The standard choice in FDE is the AES \emph{block cipher} in XTS
mode (AES-XTS). However, Dickens et al. show that we can achieve improved
performance versus AES-XTS and other block ciphers while providing a stronger
integrity guarantee by choosing a \emph{stream cipher} instead, such as the
ChaCha20\cite{ChaCha20} stream cipher.

But ChaCha20 is not the only stream cipher we can configure for use with FDE.
Other stream ciphers exist with a plethora of different performance
characteristics and security properties beyond integrity guarantees. These
include: the reduced round ChaCha8\cite{ChaCha8}, SalsaX\cite{SalsaX}, mimicking
a stream cipher with AES in counter mode (AES-CTS)\cite{AES-CTR},
Rabbit\cite{Rabbit}, Sosemanuk\cite{Sosemanuk}, Freestyle\cite{Freestyle}, et
cetera. Given these various configurations points, prior work requires a static
configuration choice made offline at compile time or initialization; a single
cipher configuration is deemed optimal given only the most generic usecase.
However, the cipher configuration providing the fastest I/O throughput may be
different than the configuration providing the strongest security guarantees,
and both may differ from the configuration that uses the least amount of energy.
Unfortunately, this is inflexible; while filesystems work to must balance
security, energy, and performance, a static compile-time configuration choice
cannot adapt to changes in resource availability or runtime environment.

Prior work focuses on performance vs the state of the art AES-XTS. Speedup is
accomplished using stream ciphers over block ciphers.

\TODO{However, prior work on using fast stream ciphers for FDE yield deeper
insights; performance win over AES-XTS comes from "append-mostly" behavior of
underlying LFS and we can leverage this further to create a mechanism for
changing ciphers online during runtime to meet dynamic latency, energy, and
security goals with minimal overhead.}

\TODO{Why this flexibility is useful and why anyone would want to pay some
overhead to take advantage of it: a taste of possibilities and the coming
usecases.}

\TODO{We present: SwitchBox!}

\TODO{We demonstrate SwitchBox's effectiveness on ...; we measure using ...; we further demonstrate with usecases ...}

\TODO{Explanation of remaining sections.}

Motivation in this section? Figures that demonstrate ciphers.

A point somewhere about when to use different strategies.

List of concerns that determine when to switch to what and why.

Liberal use of todos as placeholders.

Current state of the art misses one goal or the other. Oracle knowledge is too slow. Dynamic is only one that works.

Equate flexibility with the ability to operate between the static points. Repeat this wording in the evaluation section.
