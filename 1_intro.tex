\section{Introduction}\label{sec:introduction}

There are several important concerns modern filesystems must balance. Paramount
among them are: security of data at rest; performance, \ie{I/O latency or
throughput}; and overall energy budget. However, these concerns exist in
stark contention.

The state of the art for securing data at rest is Full-Drive Encryption
(FDE)\footnote{The common term is full-\emph{disk} encryption, but this work
targets SSDs, so we use \emph{drive}.}. FDE employs a \emph{cryptographic
cipher} to encrypt, decrypt, and authenticate backing store contents on the fly.
Understandably, this process impacts system performance and energy use
considerably. Yet, end users expect their device to be secure \emph{and}
performant \emph{and} energy efficient. The obvious solution is to balance these
concerns with respect to the most general use case, but this is not always
desirable. \TODO{Introduce two examples: 1) Android M FDE example from
Google~\cite{AndroidM}. 2) Another example from someone
else~\cite{AnotherExample}.}

Traditionally, FDE's impact on drive performance and energy efficiency depends
on choice of filesystem/mapper and the choice of cipher, as different ciphers
express different performance and efficiency characteristics while offering
contrasting security guarantees. In this way, cipher choice is perhaps the most
important.

The standard choice for FDE is the relatively slow AES~\cite{AES} \emph{block
cipher} in XTS mode (AES-XTS)~\cite{AES-XTS}. Prior work~\cite{StrongBox}
introduced the motivation and method for using \emph{stream ciphers} like the
high performance ChaCha20~\cite{ChaCha20} cipher for FDE in lieu of AES-XTS.
More recent work achieves something similar, such as Google's HBSH~\cite{HBSH}
(hash, block cipher, stream cipher, hash) and Adiantum~\cite{Adiantum}, among
others.

The benefits of choosing a more performant cipher include increased filesystem
performance(\ie{overall reduced FDE overhead}), reduced energy use, and the
ability to encrypt devices that would otherwise be too slow or
energy-inefficient to support FDE, as was the case with many mid- and low-tier
Android devices at the time FDE was considered as a default for Android M.

And ChaCha20 is not the only stream cipher we can choose for FDE. Other stream
ciphers exist with a variety of different performance characteristics and
opposing security properties. Some of these include: ChaCha variants with
different round counts (e.g., ChaCha8 \cite{ChaCha8} and ChaCha20
\cite{ChaCha20}) or exotic security properties (e.g.,
Freestyle~\cite{Freestyle}) and others---SalsaX~\cite{SalsaX}, AES in counter
mode (AES-CTR)~\cite{AES-CTR}, Rabbit~\cite{Rabbit}, Sosemanuk~\cite{Sosemanuk},
et cetera.

In this way, the choice of (stream) cipher can be viewed as a key
\emph{configuration} parameter for systems supporting FDE; \ie{the configuration
providing the fastest I/O throughput is almost always different than the
configuration providing the most desirable security guarantees, and both may
differ from the configuration that uses the least amount of energy or allows the
most writable drive space}. Further, while the cipher choice can be configured
statically at compile- or boot-time given some generic case, in traditional FDE,
these configurations cannot adapt to changes that might arise while the system
is running. Examples include changes in resource availability, runtime
environment, desired security properties, and energy budget and OS ``battery
saver'' functionality.

Hence, any static configuration will inevitably sacrifice one concern for
another, even when it is not optimal to the end user. But what if our system did
not have to sit at a static configuration point, especially when another
configuration becomes more desirable later on?

\subsection{Our Contributions}

We introduce a simple scheme to quantify the usefulness or relative ``strength''
of ciphers based on key security properties relevant to SwitchBox FDE.

Using this scheme, we define a novel tradeoff space of cipher configurations
over competing concerns: total energy use, useful desirable security properties
in the FDE context, read and write performance, total writable space on the
drive, and how quickly the drive can converge to a single configuration.

Finally, we present SwitchBox, a novel design substantively expanding on prior
work by 1) decoupling the cipher implementation from the encryption mechanism,
2) allowing ciphertext from different ciphers to coexist side-by-side on the
same drive as independent storage units, 3) employing various \emph{switching
strategies} to ``re-cipher'' those units dynamically at run-time. SwitchBox
accomplishes these without significant overhead, allowing us to navigate our
tradeoff space of cipher configurations online to ensuring the system is near
the most optimal configuration given the constraints at the moment.

\TODO{We demonstrate SwitchBox's effectiveness on ...; we measure using ...; we
further demonstrate with use cases ...}

\TODO{Short explanation of the remaining structure of the paper.}
