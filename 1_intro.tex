\section{Introduction} \label{sec:introduction}

The state of the art for securing data at rest, such as the contents of a
laptop's SSD, is Full Drive Encryption (FDE). Traditional FDE employs a single
cryptographic cipher to encrypt, decrypt, and authenticate drive contents on the
fly. Unfortunately, encryption introduces overhead that drastically impacts
system performance and total energy consumption. Hence, modern storage systems
must carefully balance these competing concerns. On the one hand, stronger
security guarantees come with increased latency and threaten to balloon energy
consumption. On the other, capping total energy consumption requires tolerating
increased latency or weaker security guarantees.

FDE implementations such as dm-crypt~\cite{dmcrypt,DmC-Android} for Linux and
BitLocker for Windows~\cite{bitlocker} balance these concerns with respect to
some generic ``common case'' workload, but this approach often fails to deliver
an optimal outcome for specific workloads. Google's Android OS is a good
example. Android supported FDE with the release of Android 3.0, yet it was not
enabled by default until Android 6.0~\cite{android-M-mobile-motivation}. Two
years prior, Google attempted to roll out FDE by default on Android 5.0 but had
to backtrack. In a statement to Engadget, Google blamed ``'performance issues on
some partner devices' ... for the backtracking''~\cite{google-engadget}. At the
same time, AnandTech reported a ``62.9\% drop in random read performance, a
50.5\% drop in random write performance, and a staggering 80.7\% drop in
sequential read performance'' versus Android 5.0 unencrypted storage for various
workloads~\cite{android-M-mobile-motivation-2}.

FDE's impact on drive performance and energy efficiency depends on a multitude
of choices. Paramount among them is the choice of cipher, as different ciphers
expose different performance and energy efficiency characteristics. In this way,
cipher choice can be viewed as a key \emph{configuration parameter} for FDE
systems. The benefits of choosing one cipher over another might include: 1)
improved performance (\ie{overall reduction in FDE overhead}), 2) reduced energy
use, 3) more useful security guarantees, and 4) the ability to encrypt devices
that would otherwise be too slow or energy-inefficient to support FDE.

However, the standard cipher choice for FDE is the slow AES \emph{block cipher}
(AES-XTS)~\cite{XTS, XTSComments, NISTXTS}. Dickens et al. introduced a method
for using ciphers other than AES-XTS for FDE; specifically, the high performance
ChaCha20 \emph{stream cipher}~\cite{StrongBox, ChaCha20}. More recent
work---such as Google's HBSH (hash, block cipher, stream cipher, hash) and
Adiantum~\cite{Adiantum}---brings stream cipher based FDE to devices that do not
or cannot support hardware accelerated AES. Hence, it has been demonstrated that
using stream ciphers for FDE is both desirable in a variety of contexts and
feasible at industry scale.

In this paper, we explore the rich configuration space of stream ciphers beyond
the narrow scope of previous work. Each cipher comes with a variety of
performance, energy efficiency, and security properties in the FDE context.
These ciphers include ChaCha variants with different round counts (e.g.,
ChaCha8 and ChaCha12)~\cite{ChaCha20}, ciphers with stronger security
guarantees versus more robust adversaries (e.g., Freestyle~\cite{Freestyle}),
and other ciphers like SalsaX~\cite{SalsaX}, AES in counter mode
(AES-CTR)~\cite{AESCTR}, Rabbit~\cite{Rabbit}, Sosemanuk~\cite{Sosemanuk}, etc.
Given this variety, the cipher configuration providing the least latency
overhead is almost always different than the configuration providing the most
desirable security properties, and both may differ from the configuration using
the least energy or preserving the most free space on the encrypted drive.

Further, while cipher choice might be configured statically at compile or boot
time with respect to some common case in traditional FDE, such configurations
cannot dynamically adapt to changes that arise while the system is running.
Examples include changes in resource availability, runtime environment, desired
security properties, and respecting changing OS energy budgets. Hence, any
static common-case FDE configuration will sacrifice one concern for another,
even when it is not optimal to do so for a given workload.

Hence we present SwitchCrypt, a storage layer that can dynamically transition
into a more desirable configuration given runtime changes. In one case study,
where we require the filesystem to react to a shrinking energy budget by
switching ciphers (\cref{subsec:uc1}), we find that SwitchCrypt achieves up to a
3.3x total energy use reduction compared to prior static approaches that only
use the Freestyle stream cipher without switching. In another case, where we
allow the user to manually switch between ChaCha20 and Freestyle stream ciphers
dynamically (\cref{subsec:uc2}), we achieve a 3.1x to 4.8x reduction to read
latency and 1.6x to 2.8x reduction to write latency compared to prior static
approaches.

\PUNT{We make the SwitchCrypt source publicly available open
source\footnote{\label{note1}\SystemURI}.}
