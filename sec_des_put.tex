\subsection{Putting It All Together}\label{subsec:summary}

After describing the three main contributions, now we discuss other details
surrounding the three main components.

% ---------------------------------- metadata

\mysub{Secure metadata management.} The focus of this paper is to implement
mechanisms and policies to perform flexible switching of cipher configurations
that can be built on top of existing block-level encryption module (aka.
``encryption driver'') that already provides the management of the encryption
data structures. There were a couple of open-source choices to start with such
as Linux dm-crypt \cite{x}, a-second-one \cite{y}, or StrongBox
\cite{StrongBox}. We decided to build atop StrongBox because it already
implements stream ciphers such as ChaCha which is more secure than block cipher.

\sys depends on several data structure management provided in StrongBox, such as
its transaction and rekeying journals (for never writing data encrypted with the
same key to the same location, hence avoiding pad reuse violations), Merkle tree
(for tracking the drive state such as \hsg{explain a bit more}), monotonic
counter (on a trusted hardware to prevent rollbacks), keycount store (to derive
the nugget's unique encryption key from some master secret and limit the maximum
length of any plaintext input to ciphers), and per-nugget metadata (to store
cipher-specific extra metadata \hsg{true??}). Every drive partition also has a
``head'' area that indicates which cipher is currently active.

While we reuse some of the components, all of these are tightly integrated to
the ciphers that they implemented (specifically ChaCha, and \xxx). We had to
untangle this, hence the contribution in the \sysB component where we now expose
more structured interfaces allowing cipher implementors to easily build the
metadata management of their cipher algorithm around the interfaces. More
specifically, out of all the five \hsg{true??} StrongBox components above, only
\xxx can be reused as is, while the rest needs to be modularized and rewritten.

% ---------------------------------- security

\mysub{Threat model under switching.} In terms of {\em confidentiality}, an
adversary should not be able to reveal any information about encrypted plaintext
without the proper key. As with prior works, encryption is achieved via a binary
additive approach: cipher output (keystream) is combined with plaintext nugget
contents using XOR, with metadata to track writes and ensure that pad reuse
never occurs during overwrites and that the system can recover from crashes into
a secure state.

In terms of {\em data integrity}, an adversary should not be able to tamper with
ciphertext and it go unnoticed. Nugget integrity is tracked by StrongBox's
in-memory Merkle tree (see \cite[Section \xxx]{StrongBox} for further details).

Switching strategies add an additional security concern not addressed by prior
work: even if we initiate a ``cipher switch,'' there may still be data on the
drive that was encrypted with an inactive configuration. Is this a problem? For
the Forward strategy, this implies data may at any time be encrypted using the
``least desirable cipher''. For the Mirrored and Selective strategies, the drive
is partitioned into regions where nuggets are guaranteed to be encrypted with
each cipher, including the ``least desirable cipher''. However, in terms of
confidentiality, the confidentiality guarantee of \sys can be reduced to the
individual confidentiality guarantees of the available ciphers used to encrypt
nuggets. \hsg{need to double check that this statement is still true, after I
already rewrite the design section.}

% ---------------------------------- integration
\mysub{Higher-level integration.} \sys expects a higher-level integration/policy
that will tell \sys what kind of switching should be performed and when. For
example, for the battery-life scenario, \sys expects that the OS battery saver
application will trigger the forward switching. We provide more in the case
studies section.

% ---------------------------------- generality
\mysub{Generality.} \sys can be seen as a drop-in replacement for the popular
Linux dm-crypt layer (encryption driver). For performance reasons, just like
StrongBox, \sys recommends a log-based file system such as F2FS, \xxx, or \xxx,
which are commonly used for flash devices. The reason for this is that
supporting streaming ciphers is more efficient in storage systems with no
in-place updates. \sys is a software solution, however the same logic can be
adopted to storage devices in the future, especially flash devices. For example,
the no in-place update of the FTL nature will help stream ciphers be performance
while at the same time users can use other popular in-place update file systems
such as ext4, btrfs, and xfs.
