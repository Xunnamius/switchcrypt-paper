\subsection{\sysA: Cipher Switching Models}\label{subsec:des-switch}

% About: when and where
Though not a technical limitation, it is helpful to think of \sys as having a
single {\em active cipher configuration} among several available. With \sysA,
our storage system has the flexibility to switch from one cipher configuration
to another, or even employ multiple configurations in a way unachievable with
prior work. When a cipher switch is triggered, a different configuration becomes
the active configuration, and ``re-ciphering'' must be done, \ie using the
deactivated configuration to decrypt a nugget's contents and using the active
configuration to re-cipher it.

% About: challenge
The challenge here is to accomplish re-ciphering while minimizing overhead. A
naive approach would switch every nugget to the active configuration
immediately, but the latency and energy cost would be unacceptable. Hence, a
more strategic approach is is necessary. For example, depending on the use case,
it may make the most sense to re-cipher a nugget immediately, or eventually, or
to maintain several areas of differently-ciphered nuggets concurrently. Our
models allow for nuggets to be re-ciphered in a variety of cases with minimal
impact on performance and battery life and without compromising security; they
are: Forward, Mirrored, and Selective. How \sys reacts to I/Os under each model
is summarized in \cref{tbl:switch}.

\begin{table}[t]
    \begin{center}
        \small
        \centering
        \begin{tabular}{lll}
                            & {\bf Read} & {\bf Write} \\\hline
            {\bf Forward}   & 1st read: $D_{C_1}$                & $E_{C_2}$ (only if necessary) \\
                            & 2nd read: $D_{C_1}$ then $E_{C_2}$ & \\
                            & Nth read: $D_{C_2}$                & \\
            \hline
            {\bf Mirrored}  & When migrating: $D_{C_1}$  & I/O is duplicated: \\
                            & from \cone's region        & $E_{C_1}$ to \cone's region \\
                            & After migrating: $D_{C_2}$ & and $E_{C_2}$ to \ctwo's region \\
                            & from \ctwo's region        & \\
            \hline
            {\bf Selective} & Encrypted with \cone: $D_{C_1}$ & Should use \cone: $E_{C_1}$ \\
                            & Encrypted with \ctwo: $D_{C_2}$ & Should use \ctwo: $E_{C_2}$ \\
            \hline
        \end{tabular}
    \end{center}

    \mycaption{tbl:switch}{Switching Models and Re-ciphering}{This table
    summarizes for each switching model what happens on I/Os to a nugget when
    \sys switches from cipher \cone to cipher \ctwo (see
    \cref{subsec:des-switch}). ``Read'' denotes returning data during the
    switch. ``Write'' denotes committing new data. $E_X$ denotes encryption (\ie
    re-ciphering) and $D_X$ denotes decryption using cipher $X$.}
\end{table}

% ------------- floating: \begin{floatingtable}[r]{ % note the open curly
% bracket \begin{tabular}{...} \end{tabular}....} % note the closed bracket
% here \mycaption{fig-pass}{x}{x.} \end{floatingtable}


% ---------------------------------- Forward switching

\mysub{Forward switching.} In the case of the battery saver mode
example, we want to switch from cipher ``\cone'', a highly-secure,
energy-expensive cipher to cipher ``\ctwo,'' a less-secure but more
energy-efficient cipher. However, for efficiency, {\em we only switch
nuggets on demand}; \ie we change those nuggets that are touched
during I/O. This switching model would be enabled as part of a
higher-level policy (\ie battery saver mode) enacted by the OS or
user.

\Cref{tbl:switch} summarizes what happens on I/Os immediately after the
active cipher is switched to \ctwo. For nugget-sized writes, new data is
ciphered with \ctwo and committed as normal. For intra-nugget writes, the nugget
is first decrypted using \cone and then re-ciphered using \ctwo with the new
data included. The new nugget is then committed as normal. Re-ciphering only
occurs when necessary; once a nugget is ciphered with \ctwo, no further
switching is required. For reads, the nugget is decrypted using \cone, it is
re-ciphered with \ctwo, and then the data is returned. On subsequent reads, the
nugget is decrypted with \ctwo and read as normal.

Later, when battery saver mode is turned off and \cone becomes the active cipher
again, nuggets are not automatically re-ciphered as, again, that would be
inefficient. Nuggets will be re-ciphered on demand using the converse of the
process described above. In general, Forward switching limits the performance
and battery impact of cipher switching to the individual nuggets being accessed.
The data at rest (ciphered with \cone) that is never accessed during the switch
remains in its original state.

% About: variants
We note that there can be variants to the Forward switching model. For instance,
a variation on read operations: we consider the first, second, and nth read
accesses distinctly. On the first read, the nugget is decrypted using \cone and
the data returned. Only if that same nugget is read a second time is it
re-ciphered with \ctwo before the data is returned. On subsequent reads, the
nugget is decrypted with \ctwo and read as normal. The intuition is that Forward
switching only happens temporarily (\eg while the battery is low), hence data
being read might only be read once and stay in memory. However, if the nugget is
read the second time, then the data is re-ciphered to \ctwo. This and other
possible variants are left for future work. So far, we find our version of
Forward switching suits a common battery-saving scenario.


% ---------------------------------- Mirrored switching

\mysub{Mirrored switching.} Next, we consider a different scenario where
``\cone'' is a cipher that has been superseded by ``\ctwo'', a newly recommended
cipher that has just been added to \sys's latest kernel/module upgrade. Let us
imagine a datacenter storage operator who wants to upgrade from \cone to \ctwo
without any service interruptions. Anticipating this, our operator partitions
available storage into two regions. That is, to support a full drive cipher
switching supported by the block layer and without application/file system
modification, we must pay the space cost to anticipate such a switch.

During the migration, as summarized in \cref{tbl:switch}, all write
operations that hit \cone's region will be mirrored to \ctwo's region.
Meanwhile, in the background, the data in \cone's region is incrementally
re-ciphered and written to \ctwo's region. Until the migration is complete, read
operations will be served by \cone's region. This is very similar to VM live
migration \cite{live-vm-migration}. After the migration completes, one can use a
mechanism such as SSD Instant/Secure Erase~\cite{ISE1,ISE2,ISE3} to securely
erase the previous storage region's data.

In general, Mirrored switching allows us to converge a drive volume to a single
cipher configuration without losing any data, suffering outages or downtime,
requiring onerous manual effort, or violating any critical service-level
agreements. We note that, after the migration, the previous region's data needs
to be securely erased to complete the upgrade; here, not overlapping the two
regions makes Instant/Secure Erase straightforward~\cite{ISE1,ISE2}.


% ---------------------------------- Selective switching

\mysub{Selective switching.} Finally, we consider the case of ``scalable
encryption.'' The key insight lies in recognizing that, when storing classified
materials, corporate secrets, intellectual property, and other information that
requires the highest level of discretion, such sensitive information often
appears within a (much) larger amount of data that is valued less. For example,
perhaps banking transaction information is littered throughout a PDF; perhaps
passwords and other sensitive information exists within several much larger log
files. Using prior techniques, either all the data would be stored with
high-overhead cipher, the critical data would be stored without the mandated
cipher type, or the data would have to be split among separate files requiring a
complex and potentially error-prone encryption management scheme.

\sys allows us to sidestep these issues by supporting heterogeneous FDE, \ie
encrypting different nuggets with different cipher configurations in the same
drive volume and even the same file. With Forward switching, we toggled nuggets
between two cipher configurations to trade off battery life and security; with
Selective switching, we want to select a specific cipher configuration each time
we encrypt a nugget to trade off performance and security.

\Cref{tbl:switch} summarizes what happens on I/Os when \sys is initialized
using the Selective model with two available cipher configurations: the
high-overhead high-security \cone and the low-overhead reduced-security \ctwo.
Suppose we have a multi-gigabyte corporate document that needs to be continually
disseminated to and consumed by multiple parties. On writes, we want certain
high priority secrets in this document to be available to only a subset of
receivers, so we should use our powerful but resource-intensive \cone to encrypt
nuggets containing these secrets. On the other hand, the remainder of
surrounding nuggets should use the more performant \ctwo, allowing other parties
to interact with the low-priority data (perhaps on a lightweight mobile device)
without experiencing degraded storage performance. On reads, \sys uses its own
metadata to automatically determine which cipher configuration to use for
decryption.
