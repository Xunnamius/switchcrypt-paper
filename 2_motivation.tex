\section{Motivation}\label{sec:motivation}

% Overview
\TODO{We might be able to forgo the next three paragraphs in this spot.  They could be really useful in the design section as an overview of Strongbox.  Here, though, they are talking about how to dynamically switch and in this section, I think we just need to motivate that dynamically switching might be a good idea.  Then say how to do it in the next section.}
To use stream ciphers for performant FDE, StrongBox leverages two key insights:
1) the overwrite-averse \emph{append-mostly} behavior of Log-structured File
Systems (LFS) and 2) the division of the backing store into discrete same-size
metadata-managed units \emph{above the block I/O layer} referred to as
\emph{nuggets}. When the rare overwrite does occur during I/O, StrongBox
modifies the cipher keystream or \emph{rekeys} the affected nugget(s) to prevent
a confidentiality violation.

This approach naturally leads to transitioning individual nuggets between any
cipher during runtime. This is because 1) encryption and decryption of nuggets
is compartmentalized; nugget-level operations occur independently of one another
and 2) all rekeying operations end up committing data to "empty"
(\ie{initialized with random data}) space on the backing store.

We can build on this behavior by abstracting the rekeying process out into a
\emph{re-ciphering} process, whereby the key \emph{and the cipher used to
encrypt/decrypt the nugget} can both be switched at runtime. This allows us to
trade off between different ciphers and their characteristics dynamically,
whereas prior work can only accomplish a static tradeoff at compile time or
at filesystem initialization.

\TODO{I think we can get away with one paragraph as an overview for this section.  Here is my suggestion:}
Our goal is to dynamically trade security for performance or energy.  To ensure that these tradeoffs are made optimally, it is desirable to quantify the security of various ciphers, which can be challenging when different schemes are resistant to different types of attacks. So, we first propose a method for quantitative cipher comparison, and then present some empirical results showing the wide range of security and energy tradeoffs that are available with state-of-the-art ciphers.  We conclude by discussing example scenarios where user requirements for security and energy usage change dynamically.


\subsection{Quantifying the Security Dimension}

To reason about trading off the security guarantees provided by various ciphers,
the strength of these guarantees must be quantified through scoring. \TODO{We do not want to say "For our purposes."  We want to propose something that would be useful for as many people as possible.  Also, I want to be very careful about attribution here.  Is this scoring system all stuff you came up with?  Did David help at all?  Is there anything here we can cite?  Basically, if this is all original work, then we should claim it as a contribution and argue that it is a generally useful scheme.  If some or all of it is prior work, that is fine, too, we just need to attribute those pieces and make it clear that others have already argued this is a useful way to compare ciphers.  No matter what, we need a stronger argument for this scheme than just that it serves our purposes.} For our
purposes, we consider three key features that, when scored, give us a useful
quantification of cipher security (see: \tblref{security-quant}).

\TODO{It might be worth talking about why it is hard to quantify security.  Maybe mention that if this is not your area, you might think it could be quantified by the time to brute force, but there is more subtlety than that including what assumptions you make about the attacker's access. }

\begin{itemize}

 \item \emph{Output randomization.} A cipher that exhibits output randomization
 can output ciphertext non-deterministically given the same input, which is
 extremely useful for FDE\@. This is a binary feature in that a cipher either
 outputs deterministically or it does not. A cipher with output randomization
 scores a 1 for this feature while a cipher without it scores a 0. \TODO{So, how does AES-XTS get a 0.2?}

 \item \emph{Resistance to cryptanalysis.} A cipher that is resistant to
 cryptanalysis can resist theoretical cryptanalytical attacks such as
 known-plaintext and chosen-plaintext attacks, offline key-guessing attacks, et
 cetera. Scores for this feature range from 0 to 1, where 0.5 represents
 standard resistance to cryptanalysis for stream ciphers in the general case\@.
 \TODO{Can we rename this feature?  I feel like "resistance to cryptanalysis" should be equivalent to the overall cipher strength just going by the meaning of the words.  Also, this bullet needs to be expanded more.  Are these attacks a strict hierarchy, like if you are resistant to offline key-guessing attacks are you resistant to everything below that?  Also, since this is a key point of the paper, we cannot just use et cetera, but we need to list all attacks we consider to arrive at this score. }

 \item \emph{Round count vs standard.} The ciphers we examine in this research
 are all constructed around the notion of \emph{rounds}, where a higher number
 of rounds implies a stronger confidentiality guarantee. This feature represents
 how many rounds the cipher executes compared to the accepted "standard" round
 count for that cipher. For instance, ChaCha8 is a reduced round version of the
 standard ChaCha20. Variants are distributed evenly from 0-1. For instance,
 ChaCha8 scores 0, ChaCha12 scores 0.5, and ChaCha20 scores 1\@. 

\end{itemize}

\TODO{It is very important that any reader who was sufficiently motivated could follow the above bullets and arrive at the same values you have in this table.  If that is not the case, then we don't have a quantifiable metric of security as much as a mapping of our own objective scores onto a number.  For example, Rabbit gets a CR of 0.4.  I am not sure how that number is derived, but we have to have a deterministic process so that everyone who applied it would arrive at the same number.  In general, I think this section just needs more detail.  The right level of detail is explain it so that anyone who read the section would derive the same numbers that you have in this table.}

\begin{table}[]
   \begin{tabular}{@{}lllll@{}}
   \toprule
   \textbf{Cipher} & \textbf{OR} & \textbf{CR} & \textbf{RR/RK} & \textbf{Rank} \\ \midrule
   ChaCha8         & 0           & 0.5         & 0              & 0.5           \\
   ChaCha12        & 0           & 0.5         & 0.5            & 1             \\
   ChaCha20        & 0           & 0.5         & 1              & 1.5           \\
   Salsa8          & 0           & 0.4         & 0              & 0.4           \\
   Salsa12         & 0           & 0.4         & 0.5            & 0.9           \\
   Salsa20         & 0           & 0.4         & 1              & 1.4           \\
   AES128-CTR      & 0           & 0.5         & 0              & 0.5           \\
   AES256-CTR      & 0           & 0.5         & 1              & 1.5           \\
   HC128           & 0           & 0.5         & 0              & 0.5           \\
   HC256           & 0           & 0.5         & 1              & 1.5           \\
   Rabbit          & 0           & 0.4         & 1              & 1.4           \\
   Sosemanuk       & 0           & 0.4         & 1              & 1.4           \\
   Freestyle (F)   & 1           & 1           & 0              & 2             \\
   Freestyle (B)   & 1           & 1           & 0.5            & 2.5           \\
   Freestyle (S)   & 1           & 1           & 1              & 3             \\
   AES128-XTS      & 0.2         & 0.5         & 1              & 1.7
   \end{tabular}
   \caption{\TODO{Table caption goes here.}}
   \label{tbl:security-quant}
\end{table}

\subsection{Statically Trading off Energy/Latency for Security}

\begin{figure}[ht]
 \centering
  \includegraphics[width=0.8\linewidth]{drawn/1.png}
   \caption{\TODO{Caption goes here}\TODO{No curve in this version!}}\label{fig:40mb-read}
\end{figure}

\figref{40mb-read} shows the security versus I/O latency static
tradeoff between different stream ciphers under StrongBox when completing a 40MB
read of encrypted storage. The experiment was performed on an ARM big.LITTLE
Exynos Octa processor, which is similar to the processors used in the Samsung
Galaxy line of phones and other devices. \TODO{In the LaTeX figure, pareto
frontier is X while other results are some other shape. Explain this here.}

Ciphers with relatively stronger security guarantees result in higher latency
for I/O operations while ciphers with relatively weaker security guarantees
result in lower latency.

\begin{figure}[ht]
 \centering
  \includegraphics[width=0.8\linewidth]{drawn/5.png}
   \caption{\TODO{Caption goes here}}\label{fig:energy-latency-linearity}
\end{figure}

\figref{energy-latency-linearity} shows that, for the stream ciphers included in
our experiments, there is a linear relationship between cipher performance and
total energy used during the I/O operation.

With prior work, \TODO{we could cite other filesystems and storage layers here
other than just StrongBox, such as ZFS, dmcrypt?} cipher configuration is made
statically at compile time or at filesystem initialization. This static choice forces
developers and end-users to choose a configuration that works best in the most
general case and stick with it, even if a different configuration becomes more
optimal at a later point. The only way to switch ciphers when a new
configuration becomes more optimal is to recreate the entire underlying
filesystem, which is rarely desirable. \TODO{This paragraph should be fleshed
out more?} \TODO{Yes, please do flesh it out more.  Add the citations you suggested.  I think you should start with the good thing: different ciphers have a wide range of energy/latency vs security properties.  Then you can highlight the two key drawbacks (maybe even bullet-point them): (1) They don't actually form a continous tradeoff space, but represent discrete points, and what if you need to be in the middle? and (2) They are fixed by the time the system finishes booting (?) and if requirements change online the system requires a reboot.}

\subsection{Dynamically Trading off Energy/Latency for Security}

\begin{figure}[ht]
 \centering
  \includegraphics[width=0.8\linewidth]{drawn/1.png}
   \caption{\TODO{Caption goes here}\TODO{Include pareto curve in this version??}\TODO{No pareto curve here.  I don't think we need a figure in this one.}}\label{fig:40mb-read-with-forward}
\end{figure}

But what if our system didn't have to sit at a static configuration point, even
when another configuration became more optimal in context? This motivates the
need for some dynamic mechanism to navigate this tradeoff space online at
runtime. More than just switching between static configuration points, such a
mechanism would allow the system to dynamically switch to otherwise unreachable
configurations along the pareto frontier curve between and including the
discrete points achievable with prior work depending on the needs of the system
at that moment in time. In \figref{40mb-read-with-forward}, we see examples of
such points. \TODO{\figref{40mb-read-with-forward} will show configuration
points along pareto curve and differentiate between static points and dynamic
points with shapes versus the other version of this chart (above) which shows
none of this}

% Or should this be a new section?
Two examples motivating this mechanism follow. The first illustrates the need for \emph{temporal} switching, where the desired security and energy tradeoffs change over time.  The second motivates \emph{spatial} switching where the desired tradeoff is different for different parts of the same file.  \\


\noindent
\textbf{Temporal switching: streaming video with a curtailed energy budget}

\TODO{Are we stepping on the use case section with this?} \TODO{No, in fact it will be a nice symmetry: you pose these as problems at the beginning and then solve them at the end. }

\TODO{I am surprised to see this works with reading a video.  If the video is stored with a high security, don't you have to decrypt it before you read and pay that high-cost (in energy) no matter what?  I had expected writing a new video where I could start with high-security and then switch to writing the second part with lower-security to save battery life.  The first (reading) is even better, but it is not clear to me how we can do that while still obeying the laws of physics---I hope I missed something, but if I did, it needs to be added to the description.}
Suppose we were streaming a video stored our mobile device to a WiFi connected
television or projector. We want to store video and other data on our device as
securely as possible, so we chose to initialize the system at configuration
point offering very strong security. \figref{energy-latency-linearity}
correlates stronger security guarantees with greater total energy use, meaning
our device is using a lot more energy to facilitate FDE at this configuration
point.

Eventually, our device determines its remaining battery life is too low and
enters an energy saving mode. With a traditional filesystem, we are stuck with
our static high-energy configuration chosen at system initialization.

However, with the ability to operate at points along the pareto frontier (c.f.
\figref{40mb-read-with-forward}), even at points between the discrete
configurations available to prior work, a context-aware system can switch to a
configuration that trades the security of the portions of storage that are being
used to stream the video so that we stay within our curtailed energy budget.

When we return to a non-curtailed energy budget, the system can return to its
more secure configuration dynamically, allowing the cipher strength of the
backing store to eventually recover without having to recreate the entire
underlying filesystem.\\

\noindent
\textbf{Spatial switching: securing a massive multi-cipher document}

\TODO{Are we stepping on the use case section with this?}

Suppose a large organization that is working on a document with hundreds of
gigabytes of data. The organization is decentralized, so this document must pass
back and forth among several disparate parties who are constantly encrypting and
decrypting certain sections of it to work on. Of course, this document is the
target of frequent brute force, supply chain, and other confidentiality
violating attacks by powerful adversaries with unknown and potentially unlimited
means at their disposal. The confidentiality of the document must never be
violated, so it must be encrypted. However, some tiny amount of the data
dispersed throughout the document is \emph{highly} sensitive; and while a
confidentiality violation of the wider document would be embarrassing, a
confidentiality violation of this very sensitive data would very likely destroy
the organization in its entirety.

With prior work, this organization is limited to static configurations that
generally fall within two camps: 1) encrypt the entire document with a cipher
providing the strongest security guarantees possible to protect the small amount
of highly sensitive data, which would add an unacceptable overhead to the
frequent encryption and decryption operations of the organization during I/O or
2) encrypt the entire document with a cipher providing the least amount of
latency during I/O, which could potentially leave the organization vulnerable to
its adversaries.

However, with the ability to operate at points between the discrete
configurations available to prior work, a dynamic system can encrypt highly
sensitive portions of the backing store with the cipher with the strongest
security guarantees while allowing the rest of the document to be encrypted with
the standard cipher so as to not unduly burden the employees of this
organization as they operate on the document.
