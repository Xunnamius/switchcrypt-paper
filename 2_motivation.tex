\section{Motivation}\label{sec:motivation}

\subsection{Background}

\TODO{Stream ciphers are hard to use for FDE.} To enable FDE using stream
ciphers, several approaches have been explored:

\begin{itemize}
   \item Use a non-deterministic CTR mode with specially designed cipher and
   filesystem (Freestyle~\cite{Freestyle}).
   \item Use a length-preserving ``tweakable super-pseudorandom permutation''
   with nonce-accepting stream cipher (Adiantum~\cite{Adiantum}).
   \item Use any stream cipher in a CTR-like mode with metadata management to
   prevent overwrites (StrongBox~\cite{StrongBox}).
\end{itemize}

In this paper, we focus on the lattermost approach. StrongBox is a stream
cipher-based FDE and metadata layer that exploits Log-structured File System's
(LFS) overwrite-averse behavior to achieve high-performance encryption and
decryption by ensuring costly \emph{re-keying} operations---where groups of
storage sectors referred to as \emph{nuggets} are re-encrypted with a new
key---are triggered as rarely as possible.

\TODO{Explain flakes/nuggets and add figure; explain how flakes allow for
overwrite detection and nuggets allow for re-keying and re-keying allows SB to
deal with overwrites and remain secure and the LFS behavior ensures rekeying is
rare, which ensures the construction is performant.}

\subsection{Key Insights}

\textbf{Nuggets contents are encrypt and decrypt independently.} Nuggets usually
all the same cipher. The nugget/flake divisions naturally lend themselves to
encrypting different portions of the backing store with different ciphers. This
would allow the filesystem to support mixed cipher configurations that can be
encrypted, decrypted, swapped individually.

\textbf{ChaCha20 is a single point in a space of cipher configurations.} What if
we used StrongBox with a cipher other than ChaCha20? Redesigning StrongBox with
a cipher API to enable ciphers other than ChaCha20, we get a space of cipher
configurations that tradeoff perf, energy, security, \TODO{other dimensions}.
StrongBox itself is initialized at one of these points (at ChaCha20).

\textbf{Navigate the configuration space online with cipher switching.} But what
if our system didn't have to sit at a static configuration point, even when
another configuration became more optimal later on? By abstracting the rekeying
process out into a re-ciphering or \emph{cipher switching} process, whereby the
key and the cipher used to encrypt/decrypt the nugget can both be switched at
runtime, we can trade off between different ciphers and their characteristics
dynamically. Comparatively, prior work can only accomplish a static tradeoff at
compile time or at filesystem initialization.

Hence, we present SwitchBox. \TODO{More explanations. Our goal is to dynamically
trade security for performance or energy.}

\subsection{Challenges}

Our goal is to dynamically trade security for performance or energy. To ensure
that these tradeoffs are made optimally, it is desirable to quantify the
security of various ciphers, which can be challenging when different schemes are
resistant to different types of attacks. To address this, we propose a method
for quantitative cipher comparison, and then present some empirical results
showing the wide range of security and energy tradeoffs that are available with
state-of-the-art ciphers.

We need to design a mechanism and cipher API to switch individual the ciphers of
individual nuggets with low overhead. Determine useful cipher switching
"strategies" to allow tradeoff of concerns along pareto curve with low overhead

Deciding when to switch?
