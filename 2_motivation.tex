\section{Motivation}\label{sec:motivation}

\subsection{Example: Filesystem Reacts to ``Battery Saver''}

Suppose we have an ARM-based ultra-low-voltage netbook provided to us by our
employer. As this is an enterprise device, our employer requires that 1) our
drive is fully encrypted at all times and 2) our encrypted data is constantly
backed up to an offsite system. The industry standard in full drive encryption
is AES-XTS, so we initialize our drive with it. Given these requirements, three
primary concerns present themselves.

First, it is well known that FDE using AES-XTS adds significant latency and
power overhead to I/O operations, especially on mobile and battery-constrained
devices~\cite{google-engadget, android-M-mobile-motivation,
android-M-mobile-motivation-2}. To keep our drive encrypted at all times with
AES-XTS means we must accept this hit to performance and battery life. Worse, if
our device does not support hardware accelerated AES, performance can be
degraded even further; I/O latency can be as high as 3-5x~\cite{StrongBox}.
Hardware accelerated AES is hardly ubiquitous, and the existence of myriad
devices that do not support it cannot simply be ignored, hence Google's
investment in Adiantum~\cite{Adiantum}.

Second, AES-XTS was designed to mitigate threats to drive data ``at rest,''
which assumes an attacker cannot access snapshots of our encrypted data nor
manipulate our data without those manipulations being immediately obvious to us.
With access to multiple snapshots of a drive's AES-XTS-encrypted contents, an
attacker can passively glean information about the plaintext over time by
contrasting those snapshots, leading to confidentiality violations in some
situations~\cite{XEX, XTS}. Similarly, an attacker that can manipulate encrypted
bits without drawing our attention will corrupt any eventual plaintext,
violating data integrity and, in the worst case, influencing the behavior of
software. Unfortunately, in real life, data rarely remains ``at rest'' in these
ways. In our example, our employer requires we back up the contents of our drive
to some offsite backup service; this service will receive periodic snapshots of
the encrypted state of our drive, violating our ``at rest'' invariant. These
backups occur at a layer above the drive controller, meaning any encryption
happening at the FTL or below is irrelevant.

Third, our system is battery constrained, placing a cap on our energy budget
that can change at any moment. Our system should respond to these changing
requirements without violating any other concerns.

To alleviate the performance concern, we can choose a stream cipher like
ChaCha20 rather than the AES-XTS block cipher. Using StrongBox, an encryption
driver built for ChaCha20-based FDE, we can achieve on average a 1.7x speedup
and a commensurate reduction in energy use~\cite{StrongBox}.

When it comes to the security concern, StrongBox solves both the snapshot and
integrity problems by 1) never writing data encrypted with the same key to the
same location and 2) tracking drive state using a Merkle tree and monotonic
counter supported by trusted hardware to prevent rollbacks. This ensures data
manipulations cannot occur and guarantees confidentiality even when snapshots
are compared regardless of the stream cipher used. Unfortunately, restoring from
a backup necessitates a forced rollback of drive state, potentially opening us
back up to confidentiality-violating snapshot comparison
attacks~\cite{StrongBox}.

To truly address the security concern requires a cipher with an additional
security property: \emph{ciphertext randomization}. Without ciphertext
randomization, an attacker can map plaintexts to their ciphertext counterparts
during snapshot comparison, especially if they can predict what might be written
to certain drive regions. However, with ciphertext randomization, a cipher will
output a different ``random'' ciphertext even when given the same key, nonce,
and plaintext; this means, even after a forced rollback of system state and/or
legitimate restoration from a backup, comparing future writes is no longer
confidentiality violating because each snapshot will always consist of different
ciphertext regardless of the plaintext being encrypted or the state of the
drive. Using Freestyle~\cite{Freestyle}, a ChaCha20-based stream cipher that
supports ciphertext randomization, we can guarantee data confidentiality in this
way. So, we switch from StrongBox to an encryption driver that supports
Freestyle.

Unfortunately, like AES-XTS, Freestyle has significant overhead compared to the
original ChaCha20. In exchange for stronger security properties, Freestyle is up
to 1.6x slower than ChaCha20, uses more energy, has a higher initialization
cost, and expands the ciphertext which reduces total writeable drive
space~\cite{Freestyle}.

Further complicating matters is our final concern: a constrained energy budget.
Our example system is battery constrained. Even if we accepted trading off
performance, drive space, and energy for security in some situations, in other
situations we might prioritize reducing total energy use. For example, when we
trigger ``battery saver'' mode, we expect our device to conserve as much energy
as possible. It would be ideal if our device could pause backups and the
encryption driver could switch from the ciphertext-randomizing Freestyle
configuration back to our high performance energy-efficient ChaCha20
configuration when conserving energy is a top priority, and then switch back to
the Freestyle configuration when we connect to a charger and backups are
eventually resumed.

In this paper we present SwitchCrypt, a device mapper that can trade off between
these two configurations and others without compromising security or performance
or requiring the device be restarted. With prior work, the user must select a
static operating point in the energy-security-latency space at initialization
time and hope it is optimal across all workloads and cases. If they choose the
Freestyle configuration, their device will be slower and battery hungry even
when they are not backing up. If they choose the more performant ChaCha20
configuration, they risk confidentiality-violating snapshot comparison attacks.
SwitchCrypt solves this problem by encrypting data with the high performance
ChaCha20 when the battery is low and switching to Freestyle when plugged in and
syncing with the backup service resumes.

\subsection{Key Challenges}

To trade off between different cipher configurations, we must address three key
challenges. First, we must determine what cipher configurations are most
desirable in which contexts and why. This requires we \emph{quantify} the
desirable properties of these configurations. Second, we must have some way to
encrypt independent storage units with any one of these configurations. This
requires we \emph{decouple} cipher implementations from the encryption process
used in prior work. Third, we need to determine when to re-encrypt those units,
which configuration to use, and where to store the output, all with minimal
overhead. This requires we implement efficient cipher \emph{switching
strategies}.

\textbf{Quantifying the properties traded off between configurations.} To obtain
a space of configurations that we might reason about, it is necessary to compare
certain properties of stream ciphers useful in the FDE context. However,
different ciphers have a wide range of security properties, performance
profiles, and output characteristics, including those that randomize their
outputs and those with non-length-preserving outputs---\ie{the cipher outputs
more data than it takes in}. To address this, we propose a framework for
quantitative cipher comparison in the FDE context; we use this framework to
define our configurations.

\textbf{Decoupling ciphers from the encryption process.} To flexibly switch
between configurations in SwitchCrypt requires a generic cipher interface. This
is challenging given the variety of inputs required by various stream ciphers,
the existence of non-length-preserving ciphers, and other differences. We
achieve the required generality by defining independent storage units called
\emph{nuggets}; we borrow this terminology from prior work (see
\cite{StrongBox}) to easily differentiate our logical blocks (nuggets) from
physical drive and other storage blocks. And since they are independent, we can
use our interface to select any configuration to encrypt or decrypt any nugget
at any point.

\textbf{Implementing efficient switching strategies.} Finally, to determine when
to switch a nugget's cipher and to where we commit the output, we implement a
series of high-level policies we call \textit{cipher switching strategies}.
These strategies leverage our generic cipher interface and flexible drive layout
to selectively ``re-cipher'' groups of nuggets, whereby the key and the cipher
used to encrypt/decrypt a nugget are switched at runtime. These strategies allow
SwitchCrypt to move from one configuration point to another or even settle on
optimal configurations wholly unachievable with prior work. The challenge
here is to accomplish this while minimizing overhead.
