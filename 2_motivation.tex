\section{Motivation}\label{sec:motivation}

\subsection{Example: Filesystem Reacts to ``Battery Saver''}

\PUNT{\begin{figure}[ht] \textbf{Linearity Between Baseline Cipher Latency and
Energy Use}\par\medskip
   \centering
   {\begin{tikzpicture}[baseline]

    \pgfmathsetmacro{\ymax}{1} % set the maximum y value
    \pgfmathsetmacro{\ymaxbreak}{1.1} % set the y value at which overflow is drawn

    \begin{axis}[
        %axis x line*=bottom,
        only marks,
        height=6cm,
        width=\linewidth,
        tick align=outside,
        tick pos=bottom, % make sure ticks only appear at the bottom and left axes
        tick style={ black },
        y tick label style={ /pgf/number format/fixed, /pgf/number format/precision=0 },
        grid style={ dotted, gray },
        %every node near coord/.append style={font=\tiny},
        %
        % magic to make the numbers appear above the overly long bars:
        % visualization depends on={rawy \as \rawy}, % save original y values
        % restrict y to domain*={ % now clip/restrict any y value to ymax
        %     \pgfkeysvalueof{/pgfplots/ymin}:\ymaxbreak
        % },
        % after end axis/.code={ % draw squiggly line indicating break
        %     \draw [semithick, white, decoration={snake,amplitude=0.1mm,segment length=0.75mm,post length=0.375mm}, decorate] (rel axis cs:0,1.01) -- (rel axis cs:1,1.01);
        % },
        % nodes near coords={\color{.!75!black}\pgfmathprintnumber\rawy}, % print the original y values (darkened in case they are too light)...
        % nodes near coords greater equal only=\ymax, % ... but ONLY if they are >= ymax
        % clip=false, % allow clip to protrude beyond ymax
        % Custom stuff to edit per template
        %
        xlabel={\footnotesize Energy (normalized)},
        xlabel near ticks,
        xmin=0, xmax=1,
        xtick={ 0, 1 },
        %enlarge x limits=0.2, % add some breathing room along the x axis's sides
        %major x tick style=transparent,
        %
        ylabel={\footnotesize Latency (normalized)},
        ylabel near ticks,
        ylabel shift={-1mm},
        ymajorgrids=true,
        ymin=0, ymax=\ymax,
        ytick={ 0, 1 },
        restrict x to domain=0:1,
        %yticklabels={ 0, 0.5, 1.5, 2 },
        % extra y ticks={1},
        % extra y tick style={grid=major, grid style={dashed, black}},
        % extra y tick label={\empty},
        %bar width=4.5pt, % change size of bars
        %
        legend cell align=left,
        legend style={ column sep=1ex },
        legend style={
            draw=none,
            legend columns=2,
            at={(0.5,1.02)},
            anchor=south,
        },
    ]
        \addplot [blue] table [
            only marks,
            x=energy,
            y=latency,
            discard if symbol not={iop}{4k-r},
            discard if symbol not={order}{seq},
            col sep=space,
        ] {charts/tradeoff-baseline.dat};
        \addlegendentry{\scriptsize 4K I/O}
        \addplot [] table [
            only marks,
            x=energy,
            y=latency,
            discard if symbol not={iop}{512k-r},
            discard if symbol not={order}{seq},
            col sep=space
            ] {charts/tradeoff-baseline.dat};
        \addlegendentry{\scriptsize 512K/5M/40M I/O}
        \addlegendimage{mark=none, red, thick}
        \addlegendentry{\scriptsize Regression}
        \addplot [] table [
            only marks,
            x=energy,
            y=latency,
            discard if symbol not={iop}{5m-r},
            discard if symbol not={order}{seq},
            col sep=space
        ] {charts/tradeoff-baseline.dat};
        \addplot [] table [
            only marks,
            x=energy,
            y=latency,
            discard if symbol not={iop}{40m-r},
            discard if symbol not={order}{seq},
            col sep=space
        ] {charts/tradeoff-baseline.dat};
        \addplot [blue] table [
            only marks,
            x=energy,
            y=latency,
            discard if symbol not={iop}{4k-w},
            discard if symbol not={order}{rnd},
            col sep=space,
        ] {charts/tradeoff-baseline.dat};
        \addplot [] table [
            only marks,
            x=energy,
            y=latency,
            discard if symbol not={iop}{512k-w},
            discard if symbol not={order}{rnd},
            col sep=space
        ] {charts/tradeoff-baseline.dat};
        \addplot [] table [
            only marks,
            x=energy,
            y=latency,
            discard if symbol not={iop}{5m-w},
            discard if symbol not={order}{rnd},
            col sep=space
        ] {charts/tradeoff-baseline.dat};
        \addplot [] table [
            only marks,
            x=energy,
            y=latency,
            discard if symbol not={iop}{40m-w},
            discard if symbol not={order}{rnd},
            col sep=space
        ] {charts/tradeoff-baseline.dat};
        \addplot [sharp plot, mark=none, red] table [
            x=energy,
            y={create col/linear regression={y=latency}},
            col sep=space
        ] {charts/tradeoff-baseline.dat};
    \end{axis}%
\end{tikzpicture}%
} \caption{Comparison of cipher
   configuration median sequential and random latency versus median total energy
   use per I/O operation size (4KB, 512KB, 5MB, 40MB) without switching. The
   relationship between latency and total energy use is linear for the cipher
   configurations we examine in this paper.}
  \label{fig:linearity-latency-energy}
\end{figure}}

Suppose we have an ARM-based ultra-low-voltage netbook provided to us by our
employer. As this is an enterprise device, our employer requires that 1) our
drive is fully encrypted at all times and 2) our encrypted data is constantly
backed up to an offsite system. The industry standard in full drive encryption
is AES-XTS, so we initialize our drive with it. Given these requirements, three
primary concerns present themselves.

First, it is well known that FDE using AES-XTS adds significant latency and
power overhead to I/O operations, especially on mobile and battery-constrained
devices~\cite{google-engadget, android-M-mobile-motivation,
android-M-mobile-motivation-2}. To keep our drive encrypted at all times with
AES-XTS means we must accept this hit to performance and battery life. Worse, if
our device does not support hardware accelerated AES, performance can be
degraded even further; I/O latency can be as high as 3-5x~\cite{StrongBox}.
Hardware accelerated AES is hardly ubiquitous, and the existence of myriad
devices that do not support it cannot simply be ignored, hence Google's
investment in Adiantum~\cite{Adiantum}.

Second, AES-XTS was designed to mitigate threats to drive data ``at rest,''
which implies an attacker cannot access snapshots of our encrypted data nor
manipulate that data without it being immediately obvious to us. With access to
multiple snapshots of a drive's AES-XTS-encrypted contents, an attacker can
passively glean information about the plaintext over time, possibly violating
data confidentiality~\cite{XEX, XTS}. Similarly, an attacker that can flip
encrypted bits without drawing our attention will corrupt any eventual
plaintext. Unfortunately, in real life, data rarely remains ``at rest'' in this
way. In our example, our employer requires we back up the contents of our drive
to some offsite backup service. This service will receive periodic snapshots of
the encrypted state of our drive, violating our invariant.

Third, our system is battery constrained, placing a cap on our energy budget
that can change at any moment.

To alleviate the performance concern, we can choose a stream cipher like
ChaCha20 rather than the AES-XTS block cipher. Using StrongBox, an encryption
driver built for ChaCha20-based FDE, we can achieve on average a 1.7x speedup
and a commensurate reduction in energy use~\cite{StrongBox}.

When it comes to the security concern, StrongBox solves both the snapshot and
integrity problems by 1) never writing data encrypted with the same key to the
same location and 2) tracking drive state using a Merkle tree and monotonic
counter supported by trusted hardware to prevent rollbacks. This ensures data
manipulations cannot occur and guarantees confidentiality even when snapshots
are compared. Unfortunately, restoring from a backup necessitates a forced
rollback of drive state, potentially opening us back up to
confidentiality-violating snapshot comparison attacks~\cite{StrongBox}.

To truly address the security concern requires a cipher with an additional
security property: ciphertext randomization. Using Freestyle~\cite{Freestyle}, a
ChaCha20-based stream cipher that supports ciphertext randomization, we can
guarantee data confidentiality even after a rollback of drive state occurs. So,
we switch to an encryption driver that supports Freestyle.

Unfortunately, like AES-XTS, Freestyle has significant overhead compared to the
original ChaCha20. In exchange for stronger security properties, Freestyle is up
to 1.6x slower than ChaCha20, uses more energy, has a higher initialization
cost, and expands the ciphertext which reduces total writeable drive
space~\cite{Freestyle}.

Further complicating matters is our final concern: a constrained energy budget.
Our example system is battery constrained. Even if we accepted trading off
performance, drive space, and energy for security in some situations, in other
situations we might prioritize reducing total energy use. For example, when we
trigger ``battery saver'' mode, we expect our device to conserve as much energy
as possible. It would be ideal if our device could pause backups and the
encryption driver could switch from the ciphertext randomizing Freestyle
configuration back to our high performance energy-efficient ChaCha20
configuration when conserving energy is a top priority, and then switch back to
the Freestyle configuration when we connect to a charger and backups are
eventually resumed.

With SwitchCrypt, we can dynamically trade off between these two configurations
and others without compromising security or performance or requiring the device
be restarted.

\subsection{Key Challenges}

To trade off between different configurations, we must address three key
challenges. First, we must understand how to flexibly encrypt independent
storage units efficiently. This requires we \emph{decouple} cipher
implementations from the encryption process. Second, we need to determine when
to re-encrypt those units and where to store the output. We address this with
\emph{switching strategies}. Third, we must reason about which cipher
configurations are most desirable in which contexts and why. This requires we
\emph{quantify} the desirable properties of different configurations.

\textbf{Decoupling ciphers from encryption for mixed-cipher layouts.} To rapidly
switch ciphers for the drive, we require a \emph{generic cipher API} and
flexible drive layout. These requirements are challenging because, even with the
class of stream ciphers, we find vastly different input requirements, output
formats, and other barriers to presenting a single unified interface that can
work with our flexible drive layout. We achieve the required generality by
defining independent storage units we call \emph{nuggets}. We borrow this
terminology from prior work (see \cite{StrongBox}) to easily differentiate our
logical blocks (nuggets) from physical drive and other storage blocks. And since
they are independent, we can use our cipher API to select any cipher to encrypt
or decrypt any nugget at any point, answering the ``how'' of switching ciphers.

\textbf{Strategies to switch nugget ciphers with acceptable overhead.} To answer
``when'' to switch a nugget's cipher and to ``where'' we commit the output, we
implement a series of policies we call \textit{cipher switching strategies} that
leverage the generic cipher API and drive layout to selectively ``re-cipher''
groups of nuggets, whereby the key and the cipher used to encrypt/decrypt a
nugget are switched at runtime. These strategies allow us to navigate our
configuration tradeoff space and settle on optimal points unreachable with prior
work. The challenge here is to accomplish this while minimizing overhead.

\textbf{Quantifying ciphers with disparate properties.} Finally, to obtain a
configuration space that we might reason about, it is necessary to score certain
security properties of stream ciphers. This is challenging since different
ciphers have a wide range of disparate security properties, including ciphers
that are not length-preserving in their output. To address this challenge, we
propose a method for quantitative cipher comparison in the FDE context and use
it to define our configuration space. These cipher configurations, each with
different strengths, answer ``why'' we might prefer one over another.
