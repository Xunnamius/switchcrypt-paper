\section{Conclusion}\label{sec:conclusion}

This paper advocates for a more flexible approach to FDE where the storage
system can dynamically adjust the tradeoffs between security and latency/energy.
To support this vision of heterogeneous FDE, we 1) identified three switching
modes to switch ciphers with low overhead and no downtime, 2) demonstrated an
interface that allows multiple stream ciphers to be composed in a generic
manner, and 3) proposed a classification framework for determining when to use
one cipher over another. Our case studies show how different switching modes can
be used to optimize for different goals. We believe heterogeneous encryption
will become increasingly important as users demand storage systems be able to
balance conflicting operational requirements. We hope this work inspires further
research in achieving this balance. Our work is publicly available open
source~\footnoteref{ftn:foss}.

\hsg{Other notes:\\
- double check figure/table captions to be consistent to what the text says\\
- Add ``as explained in cref'' in every table/figure caption. \\
- Put figures on the same page where they are being described. - Make sure
figure captions explain the x-axis and y-axis properly. Do explicitly state
``the x-axis shows ..., the y-axis shows ...''. Do put more annotations inside
the figure/graph to help reviewers read the final conclusion.}

