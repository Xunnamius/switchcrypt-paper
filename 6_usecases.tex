\section{\SYSTEM{} Case Studies} \label{sec:usecases}

\subsection{Variable Security Regions (VSR)}

Communicating classified materials, grand jury testimony, corporate secrets,
etc. require the highest level of discretion when handled, yet sensitive
information like this often appears within a (much) larger amount of data that
we care less about in context.

In this scenario, a user wants to indicate one or more regions of a file are
more sensitive than the others. For example, perhaps banking transaction
information is littered throughout a document; perhaps passwords and other
sensitive or compromising information exists within a much larger data file.
This sensitive information would be encrypted using a less performant (sometimes
dramatically so) cipher in exchange for a stronger security guarantee.

The user will not experience a significant performance hit when perusing the
data if the bulk of it is encrypted using a high performance cipher.
Simultaneously, the more sensitive data regions are future-proofed and more
resilient to attack using a high security cipher.

A "VSR" is a region of a file that is crypted with the alternative
rather than the primary cipher that encrypts the remaining majority of the file.

Benefit: we can "future-proof" our encrypted highly sensitive data against more
powerful future attacks/less trustworthy ciphers while preserving the
performance win from using a faster less secure cipher.

\subsection{Constrained Energy Budget}

When our mobile devices enter power saving mode, it is usually because the total
energy/power budget for the device has become constrained for one reason or
another.

When a device enters this mode, all software and components are configured by
the OS to use as little of the available energy as possible. The filesystem
should be made to behave in a manner that is energy-aware as well.

Our goal is to use as little energy as possible (while reasonably preserving
filesystem performance) until the energy budget changes or the device dies.

Benefit: When constantly streaming data, e.g. using DLNA to stream a high
resolution video wirelessly to a TV on the same network, the ability to adapt to
time-varying data rates and QoS requirements while maintaining confidentiality
and integrity guarantees is paramount. This can be done by trading off a set of
security guarantees with respect to the energy spent crypting each bit. With
cipher switching, the filesystem can react dynamically to the system's total
energy budget while still aiming for the most performant (least latency)
configuration.

\subsection{SSD End-of-Life Offset}

Due to garbage collection and the append-mostly nature of SSDs and other NAND
devices, as free space becomes constrained, performance drops off a cliff. This
is a well-studied issue (see related work).

If the filesystem is made aware when the backing store is in such a state, we
can offset some of the (drastic) performance loss by swapping the ciphers of hot
nuggets to the fastest cipher available until the disk space problem is
remedied, after which the system can detect return the swapped nuggets to their
former encrypted state.

Benefit: we can mitigate the performance loss of a slowing SSD by using a faster
but less secure cipher.

\subsection{Custody Panic}

Nation-state and other adversaries have truly extensive compute resources at
their disposal, as well as knowledge of side-channels and access to technology
like q-bit computers.

Suppose one were attempting to re-enter a country through a border checkpoint
after visiting family when one is stopped. Your mobile device is confiscated and
placed in custody of the State. In such a scenario, it would be useful if the
device could swap itself into a more secure state as quickly as possible.

Benefit: greater security guarantee achieved using the highest security
encryption available versus powerful adversaries with unknown means and motive.
