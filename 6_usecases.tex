\section{SwitchCrypt Case Studies}\label{sec:usecases}

\TODO{You need some sort of overview paragraph that tells people what is to come
in this section and why they should care about it. For example, you want to say
something about these case studies cover a wide range of situations including...
They also demonstrate uses of both temporal and spatial switching... Basically,
make the argument that these case studies provide good coverage of all the
things that you discussed earlier in the paper.}

\subsection{Balancing Security Goals with a Constrained Energy Budget}

This usecase illustrates that, because latency and energy use are correlated
among the ciphers we examined, we can exploit that property using temporal
Forward switching to save our battery. We revisit the motivating example from
\secref{motivation}, demonstrating that the ability to re-cipher individual
nuggets allows us to complete our task while staying within our energy budget.

\TODO{We may need to say something about how/why these file sizes were chosen. We want to make it clear that they are not magic numbers and the results would hold with different sizes.}
We begin sequentially writing 10 40MB files using the Freestyle Balanced cipher
configuration. After 5 seconds, the device enters ``battery saver'' mode. We
simulate this event by 1) underclocking the cores to their lowest frequencies
and 2) using \texttt{taskset} to transition the SwitchCrypt processes to the
energy-efficient LITTLE cores. Afterwards, we complete the remaining workload
using the ChaCha8 cipher. We repeat this experiment three times.

\begin{figure}[ht] \textbf{Batter Saver Use Case: Energy-Security Tradeoff vs
   Strict Energy Budget}\par\medskip
   \centering
   {\begin{tikzpicture}[baseline]

    \pgfmathsetmacro{\xmax}{130} % set the maximum x value
    \pgfmathsetmacro{\ymax}{50} % set the maximum y value
    \pgfmathsetmacro{\ymaxbreak}{50.1} % set the y value at which overflow is drawn

    \begin{groupplot}[
        group style={
            group size=1 by 2,
            ylabels at=edge left,
            xlabels at=edge bottom,
            yticklabels at=edge left,
            xticklabels at=edge bottom,
            vertical sep=10pt,
        },
        %axis x line*=bottom,
        height=4cm,
        width=\linewidth,
        tick align=outside,
        tick pos=bottom, % make sure ticks only appear at the bottom and left axes
        tick style={ black },
        y tick label style={ /pgf/number format/fixed, /pgf/number format/precision=0 },
        grid style={ dotted, gray },
        every node near coord/.append style={font=\tiny},
        %
        % % magic to make the numbers appear above the overly long bars:
        % visualization depends on={rawy \as \rawy}, % save original y values
        % restrict y to domain*={ % now clip/restrict any y value to ymax
        %     \pgfkeysvalueof{/pgfplots/ymin}:\ymaxbreak
        % },
        % after end axis/.code={ % draw squiggly line indicating break
        %     \draw [semithick, white, decoration={snake,amplitude=0.1mm,segment length=0.75mm,post length=0.375mm}, decorate] (rel axis cs:0,1.01) -- (rel axis cs:1,1.01);
        % },
        % nodes near coords={\color{.!75!black}\pgfmathprintnumber\rawy}, % print the original y values (darkened in case they're too light)...
        % nodes near coords greater equal only=\ymax, % ... but ONLY if they're >= ymax
        clip=true, % allow clip to protrude beyond ymax if false
        % % Custom stuff to edit per template
        %
        xlabel={Time (s)},
        xlabel near ticks,
        xlabel shift={-4mm},
        xmin=0, xmax=\xmax,
        xtick={ 0, \xmax },
        enlargelimits=false, % add some breathing room along the x axis's sides
        % %major x tick style=transparent,
        %
        ylabel near ticks,
        ylabel shift={-5mm},
        ymajorgrids=true,
        %yticklabels={ 0, 0.5, 1.5, 2 },
        % extra y ticks={1},
        % extra y tick style={grid=major, grid style={dashed, black}},
        % extra y tick label={\empty},
        %bar width=4.5pt, % change size of bars
        %
        legend cell align=center,
        legend style={ column sep=1ex },
        legend entries={
            {\scriptsize Freestyle Balanced},
            {\scriptsize Freestyle Balanced + ChaCha8},
            {\scriptsize ChaCha8},
        },
        legend style={
            draw=none,
            legend columns=2,
            at={(0.5, 1.02)},
            anchor=south,
        },
    ]
        \nextgroupplot[ylabel={Energy Used (j)}, ymin=0, ymax=\ymax, ytick={ 0, \ymax }]
            \addplot [thick] table [
                x=time,
                y=energy,
                discard if symbol not={cipher}{fb},
                discard if symbol not={iop}{w},
                col sep=space,
                mark=none
            ] {charts/usecase-battery.dat};
            \addplot [thick, dashdotted] table [
                x=time,
                y=energy,
                discard if symbol not={cipher}{fb+c8},
                discard if symbol not={iop}{w},
                col sep=space,
                mark=none
            ] {charts/usecase-battery.dat};
            \addplot [thick, densely dashed] table [
                x=time,
                y=energy,
                discard if symbol not={cipher}{c8},
                discard if symbol not={iop}{w},
                col sep=space,
                mark=none
            ] {charts/usecase-battery.dat};
            \coordinate (c1) at (35, 24);
            \coordinate (c2) at (89, 45);
            \coordinate (c3) at (0, 45);
            \draw [dotted] (0, 34) -- (130, 34) node [above of=c1] {\tiny (energy max)};
            \draw [dotted] (120, 0) -- (120, 50) node [right of=c2] {\tiny (battery dies)};
            \draw [dotted] (5, 0) -- (5, 50) node [right of=c3] {\tiny (battery critical)};
        \nextgroupplot[legend to name={throwaway7}, ylabel={Security Score}, ymin=0, ymax=3, ytick={ 0, 3 }]
            \addplot [thick] table [
                x=time,
                y=score,
                discard if symbol not={cipher}{fb},
                discard if symbol not={iop}{w},
                col sep=space
            ] {charts/usecase-battery.dat};
            \addplot [thick, dashdotted] table [
                x=time,
                y=score,
                discard if symbol not={cipher}{fb+c8},
                discard if symbol not={iop}{w},
                col sep=space
            ] {charts/usecase-battery.dat};
            \addplot [thick, densely dashed] table [
                x=time,
                y=score,
                discard if symbol not={cipher}{c8},
                discard if symbol not={iop}{w},
                col sep=space
            ] {charts/usecase-battery.dat};
            \coordinate (c4) at (35, 1.32);
            \draw [dotted] (120, 0) -- (120, 3);
            \draw [dotted] (5, 0) -- (5, 3);
            \draw [dotted] (0, 0.55) -- (130, 0.55) node [below of=c4] {\tiny (security floor)};
    \end{groupplot}%
\end{tikzpicture}%
} \caption{Median sequential write total
   energy use with respect to time and security score with respect to time.}
  \label{fig:usecase-battery}
\end{figure}

In \figref{usecase-battery}, we see time versus energy used and average security
score of the backing store. At 0 seconds, we begin writing. At 5 seconds, the
``battery critical'' event occurs, causing the system to be underclocked. At 120
seconds, the system will die. If we blow past our energy ceiling, the system
will die.

Our goal is to finish downloading the file before the device dies. We have three
cipher configuration choices. 1) Favor security and use Freestyle Balanced
exclusively. Our results show that the device will die before completing the
download. 2) Favor low energy use with ChaCha8 exclusively. Our results show
that the device will finish writing early, but we fall below our minimum
security score constraint. Finally, we have 3) favor security and use Freestyle
Balanced except when the system enters a low power state, after which the
storage layer switches to favoring optimal energy use using ChaCha20 via the
Forward switching strategy. Our results show that, while the system uses
slightly more power in the short term, we stay within our energy budget and
finish before the devices dies. Further, when we get our device to a charger,
SwitchCrypt can converge nuggets back to Freestyle Balanced.

On average, using Forward cipher switching results in a \TODO{XXX} total energy
use reduction.

\subsection{Variable Security Regions}

This usecase illustrates utility of spatial Selective switching to achieve a
performance win over prior work, where the entire drive is encrypted with a
single cipher. We demonstrate \emph{Variable Security Regions} (VSR), where we
can choose to encrypt select files or portions of files with different keys and
ciphers below the filesystem level. 

The goal is that if only a small percentage
of the data needs the strongest encryption, then only a small percentage of the
data should have that associated overhead.  Using prior techniques, either all 
the data would be stored with high overhead, the critical data would be stored 
without sufficient security, or the data would have to be split among separate 
files and stored across partitioned stores.

Communicating classified materials, corporate secrets, etc. require the highest
level of discretion when handled, yet sensitive information like this can
appears within a (much) larger amount of data that we value less. In this
scenario, a user wants to indicate one or more regions of a file are more
sensitive than the others. For example, perhaps banking transaction information
is littered throughout a document; perhaps passwords and other sensitive
information exists within several much larger files.

We begin by writing 10 5MB and 4KB files to unique SwitchCrypt instances using
ChaCha8 and again on instances using Freestyle Balanced. We repeat this on a
SwitchCrypt instance using Selective switching with a 3:1 ratio of ChaCha8
nugget I/O operations versus Freestyle Balanced operations. We repeat this
experiment three times.

\begin{figure}[ht] \textbf{VSR Use Case: ChaCha8 vs Freestyle Secure Sequential
4KB, 5MB Performance}\par\medskip
   \centering
   {\begin{tikzpicture}[baseline]

    \pgfmathsetmacro{\ymax}{20} % set the maximum y value
    \pgfmathsetmacro{\ymaxbreak}{20.1} % set the y value at which overflow is drawn

    \begin{axis}[
        %axis x line*=bottom,
        height=4cm,
        width=\linewidth,
        tick align=outside,
        tick pos=bottom, % make sure ticks only appear at the bottom and left axes
        tick style={ black },
        y tick label style={ /pgf/number format/fixed, /pgf/number format/precision=0 },
        grid style={ dotted, gray },
        every node near coord/.append style={font=\tiny},
        %
        % magic to make the numbers appear above the overly long bars:
        visualization depends on={rawy \as \rawy}, % save original y values
        restrict y to domain*={ % now clip/restrict any y value to ymax
            \pgfkeysvalueof{/pgfplots/ymin}:\ymaxbreak
        },
        after end axis/.code={ % draw squiggly line indicating break
            \draw [semithick, white, decoration={snake,amplitude=0.1mm,segment length=0.75mm,post length=0.375mm}, decorate] (rel axis cs:0,1.01) -- (rel axis cs:1,1.01);
        },
        nodes near coords={\color{.!75!black}\pgfmathprintnumber\rawy}, % print the original y values (darkened in case they're too light)...
        nodes near coords greater equal only=\ymax, % ... but ONLY if they're >= ymax
        clip=false, % allow clip to protrude beyond ymax
        % Custom stuff to edit per template
        %
        xlabel={\footnotesize Cipher Configuration},
        xlabel near ticks,
        xmin=C8, xmax=FS,
        xtick=data,
        symbolic x coords={C8,C8+FS,FS},
        enlarge x limits=0.2, % add some breathing room along the x axis's sides
        %major x tick style=transparent,
        %
        ylabel={\footnotesize Latency (s)},
        ylabel near ticks,
        ylabel shift={-1mm},
        ymajorgrids=true,
        ymin=0, ymax=\ymax,
        ybar, % value will shift bars
        ytick={ 0, 5, ..., \ymax },
        %yticklabels={ 0, 0.5, 1.5, 2 },
        % extra y ticks={1},
        % extra y tick style={grid=major, grid style={dashed, black}},
        % extra y tick label={\empty},
        %bar width=4.5pt, % change size of bars
        %
        legend cell align=center,
        legend style={ column sep=1ex },
        legend entries={
            {\scriptsize 4K/reads},
            {\scriptsize 4K/writes},
            {\scriptsize 5M/reads},
            {\scriptsize 5M/writes},
        },
        legend style={
            draw=none,
            legend columns=2,
            at={(0.5,1.02)},
            anchor=south,
        },
    ]
        \addplot[fill=orangeDark, every node near coord/.append style={color=orangeDark}]
        table[x=conf, y=latr-4k, col sep=space] {charts/usecase-vsr-tradeoff.dat};
        \addplot[fill=orangeDark, postaction={pattern=north east lines}, every node near coord/.append style={color=purpleDark}]
        table[x=conf, y=latw-4k, col sep=space] {charts/usecase-vsr-tradeoff.dat};
        \addplot[fill=purpleDark, every node near coord/.append style={color=orangeDark}]
        table[x=conf, y=latr-5m, col sep=space] {charts/usecase-vsr-tradeoff.dat};
        \addplot[fill=purpleDark, postaction={pattern=north east lines}, every node near coord/.append style={color=purpleDark}]
        table[x=conf, y=latw-5m, col sep=space] {charts/usecase-vsr-tradeoff.dat};
    \end{axis}%
\end{tikzpicture}%
} \caption{Median sequential read and
   write performance comparison of 5MB I/O with 3-to-1 ratio of ChaCha8 nuggets
   to Freestyle Secure nuggets, respectively.}
  \label{fig:usecase-vsr-bar}
\end{figure}

In \figref{usecase-vsr-bar}, we see the sequential read and write performance of
4K and 5M workloads when nuggets are encrypted exclusively with ChaCha8 or
Freestyle Balanced. Between them, we see SwitchCrypt Selective switching 3:1
ratio I/O results.

Our goal is to use VSRs to keep our sensitive data secure while keeping the
performance and energy use benefits of using a fast cipher for the majority of
I/O operations. On average, using SwitchCrypt Selective switching versus prior
work results in a \TODO{XXX} reduction in latency.

\subsection{Responding to End-of-Life Slowdown in Solid State Drives}

This usecase illustrates using temporal Forward switching to offset the
debilitating decline in performance when SSDs reach end-of-life
(EoL)~\cite{SSDEOL1, SSDEOL2, SSDEOL3}. We demonstrate the utility of such a
system to dynamically stay within a strict latency budget while meeting minimum
security requirements, which is not possible using prior work.

Due to garbage collection and wear-leveling requirements of SSDs, as free space
becomes constrained, I/O performance drops precipitously~\cite{SSDEOL1, SSDEOL2,
SSDEOL3}. With prior work, our strict latency ceiling is violated. However, if
SwitchCrypt is made aware when the backing store is in such a state, we can
offset some of the performance loss by switching the ciphers of high traffic
nuggets to the fastest cipher available using Forward switching.

We begin by writing 10 40MB files to SwitchCrypt per each cipher as a baseline.
We then introduce a delay into SwitchCrypt I/O of $20ms$ and repeat the
experiment three times.

\begin{figure}[ht] \textbf{SSD EoL Use Case: Latency-Security Tradeoff vs
   Goals}\par\medskip
   {\begin{tikzpicture}[baseline]

    \pgfmathsetmacro{\ymax}{1.1} % set the maximum y value
    \pgfmathsetmacro{\ymaxbreak}{1.2} % set the y value at which overflow is drawn

    \begin{groupplot}[
        group style={
            group size=2 by 2,
            xlabels at=edge bottom,
            ylabels at=edge left,
            xticklabels at=edge bottom,
            yticklabels at=edge left,
            vertical sep=25pt,
            horizontal sep=15pt,
        },
        %axis x line*=bottom,
        scatter,
        point meta=explicit,
        scatter/classes={
            1={},
            2={dashed},
            3={mark=triangle*,red,mark size=2.5pt},
            4={mark=triangle*,orange,mark size=3pt},
            5={mark=square*,blue,mark size=2pt}
        },
        height=6cm,
        width=\textwidth/2,
        tick align=outside,
        %tick pos=bottom, % make sure ticks only appear at the bottom and left axes
        title style={yshift=-1.5ex},
        tick style={ black },
        y tick label style={ /pgf/number format/fixed, /pgf/number format/precision=0 },
        grid style={ dotted, gray },
        %every node near coord/.append style={font=\tiny},
        %
        % magic to make the numbers appear above the overly long bars:
        % visualization depends on={rawy \as \rawy}, % save original y values
        % restrict y to domain*={ % now clip/restrict any y value to ymax
        %     \pgfkeysvalueof{/pgfplots/ymin}:\ymaxbreak
        % },
        % after end axis/.code={ % draw squiggly line indicating break
        %     \draw [semithick, white, decoration={snake,amplitude=0.1mm,segment length=0.75mm,post length=0.375mm}, decorate] (rel axis cs:0,1.01) -- (rel axis cs:1,1.01);
        % },
        % nodes near coords={\color{.!75!black}\pgfmathprintnumber\rawy}, % print the original y values (darkened in case they're too light)...
        % nodes near coords greater equal only=\ymax, % ... but ONLY if they're >= ymax
        % clip=false, % allow clip to protrude beyond ymax
        % Custom stuff to edit per template
        %
        xlabel={\footnotesize Security Score},
        xlabel near ticks,
        %xlabel shift={-1.5mm},
        xmin=0, xmax=4,
        xtick={ 0, 1, 2, 3, 4 },
        xticklabels={ 0,,, 3, \empty },
        major x tick style=transparent,
        %enlarge x limits=0.2, % add some breathing room along the x axis's sides
        %
        ylabel={\footnotesize Latency (normalized)},
        ylabel near ticks,
        ylabel shift={-1.5mm},
        ymajorgrids=true,
        ymin=0, ymax=\ymax,
        ytick={ 0, 1, \ymax },
        yticklabels={ 0, 1, \empty },
        %yticklabels={ 0, 0.5, 1.5, 2 },
        % extra y ticks={1},
        % extra y tick style={grid=major, grid style={dashed, black}},
        % extra y tick label={\empty},
        %bar width=4.5pt, % change size of bars
        %
        legend cell align=center,
        legend style={ column sep=1ex },
        legend entries={%
            {\scriptsize Normal},
            {\scriptsize Delayed},
            {\scriptsize Choice Config (Normal)},
            {\scriptsize Bad Config (Delayed)},
            {\scriptsize Choice Config (Delayed)}
        },
        legend style={
            draw=none,
            legend columns=3,
            at={(1.0,1.2)},
            anchor=south,
        },
    ]
        \nextgroupplot[title={Sequential Reads}]
            \addplot [thick] table [
                meta=label,
                x=score,
                y=latency,
                discard if symbol not={iop}{r},
                discard if symbol not={delayed}{no},
                discard if symbol not={order}{seq},
                col sep=space,
            ] {charts/usecase-eol-tradeoff.dat};
            \addplot [thick, dashed] table [
                meta=label,
                x=score,
                y=latency,
                discard if symbol not={iop}{r},
                discard if symbol not={delayed}{yes},
                discard if symbol not={order}{seq},
                col sep=space
            ] {charts/usecase-eol-tradeoff.dat};
            \coordinate (c1) at (205, 85);
            \coordinate (c2) at (60, 9);
            \draw [dotted] (190, 0) -- (190, 110) node [left of=c1] {\tiny (security floor)};
            \draw [dotted] (0, 30) -- (400, 30) node [above of=c2] {\tiny (latency ceiling)};
        \nextgroupplot[legend to name={throwaway9}, title={Random Reads}]
            \addplot [thick] table [
                meta=label,
                x=score,
                y=latency,
                discard if symbol not={iop}{r},
                discard if symbol not={delayed}{no},
                discard if symbol not={order}{rnd},
                col sep=space
            ] {charts/usecase-eol-tradeoff.dat};
            \addplot [thick, dashed] table [
                meta=label,
                x=score,
                y=latency,
                discard if symbol not={iop}{r},
                discard if symbol not={delayed}{yes},
                discard if symbol not={order}{rnd},
                col sep=space
            ] {charts/usecase-eol-tradeoff.dat};
            \coordinate (c3) at (205, 85);
            \coordinate (c4) at (60, 9);
            \draw [dotted] (190, 0) -- (190, 110) node [left of=c3] {\tiny (security floor)};
            \draw [dotted] (0, 30) -- (400, 30) node [above of=c4] {\tiny (latency ceiling)};
        \nextgroupplot[legend to name={throwaway10}, title={Sequential Writes}]
            \addplot [thick] table [
                meta=label,
                x=score,
                y=latency,
                discard if symbol not={iop}{w},
                discard if symbol not={delayed}{no},
                discard if symbol not={order}{seq},
                col sep=space
            ] {charts/usecase-eol-tradeoff.dat};
            \addplot [thick, dashed] table [
                meta=label,
                x=score,
                y=latency,
                discard if symbol not={iop}{w},
                discard if symbol not={delayed}{yes},
                discard if symbol not={order}{seq},
                col sep=space
            ] {charts/usecase-eol-tradeoff.dat};
            \coordinate (c5) at (205, 85);
            \coordinate (c6) at (60, 9);
            \draw [dotted] (190, 0) -- (190, 110) node [left of=c5] {\tiny (security floor)};
            \draw [dotted] (0, 30) -- (400, 30) node [above of=c6] {\tiny (latency ceiling)};
        \nextgroupplot[legend to name={throwaway11}, title={Random Writes}]
            \addplot [thick] table [
                meta=label,
                x=score,
                y=latency,
                discard if symbol not={iop}{w},
                discard if symbol not={delayed}{no},
                discard if symbol not={order}{rnd},
                col sep=space
            ] {charts/usecase-eol-tradeoff.dat};
            \addplot [thick, dashed] table [
                meta=label,
                x=score,
                y=latency,
                discard if symbol not={iop}{w},
                discard if symbol not={delayed}{yes},
                discard if symbol not={order}{rnd},
                col sep=space
            ] {charts/usecase-eol-tradeoff.dat};
            \coordinate (c7) at (205, 85);
            \coordinate (c8) at (60, 9);
            \draw [dotted] (190, 0) -- (190, 110) node [left of=c7] {\tiny (security floor)};
            \draw [dotted] (0, 30) -- (400, 30) node [above of=c8] {\tiny (latency ceiling)};
    \end{groupplot}%
\end{tikzpicture}%
} \caption{Median sequential and
   random 40MB read and write performance comparison: baseline versus simulated
   faulty block device.}
  \label{fig:usecase-eol-tradeoff}
\end{figure}

In \figref{usecase-eol-tradeoff}, we see the sequential and random read and
write performance of a 40MB workload when nuggets are encrypted exclusively with
our choice ciphers. While the latency ceiling and security floor have not
changed, we see increased latency in the delayed workloads.

Our goal is to remain under the latency ceiling while remaining above the
security floor. Thanks to Forward switching, accesses to highly trafficked areas
of the drive can remain performant even during EoL.

\subsection{Custody Panic: Securing Device Data Under Duress}

This usecase illustrates the utility of spatial Mirrored switching to take
advantage of more energy-efficient high-performance ciphers while retaining the
ability to quickly converge the entire backing store to a single high-security
cipher leveraging SSD Instant Secure Erase (ISE).

Nation-state and other ``adversaries'' have extensive compute resources,
knowledge of side-channels, and access to technology like quantum computers.
Suppose a scientist were attempting to re-enter her country through a border
entry point when she is stopped. Further suppose her laptop containing sensitive
priceless research data is confiscated from her custody. Being a security
researcher, she has a chance to trigger a remote wipe, where the laptop uses
Instant Secure Erase to reset its internal storage, permanently destroying all
her data. While she certainly doesn't want her data falling into the wrong
hands, she cannot afford to lose that data either. In such a scenario, it would
be useful if, instead of destroying the data, the storage layer could switch
itself to a more secure state as quickly as possible.

\begin{figure}[ht] \textbf{Custody Panic Use Case: Security Goals vs Time
Constraint}\par\medskip
   \centering
   {\begin{tikzpicture}[baseline]
    \begin{groupplot}[
        % 6 seconds (0.3) + 3 seconds (0.15) + 11 seconds (0.55) = 1.0
        no marks,
        group style={
            group size=1 by 2,
            xlabels at=edge bottom,
            ylabels at=edge left,
            %xticklabels at=edge bottom,
            yticklabels at=edge left,
            vertical sep=35pt,
            horizontal sep=15pt,
        },
        %axis x line*=bottom,
        height=4cm,
        width=\linewidth/1.25,
        tick align=outside,
        tick pos=bottom, % make sure ticks only appear at the bottom and left axes
        title style={yshift=-1.5ex},
        tick style={ black },
        y tick label style={ /pgf/number format/fixed, /pgf/number format/precision=0 },
        grid style={ dotted, gray },
        point meta=explicit symbolic,
        scatter/classes={
            c8={mark=square*},
            c20={mark=triangle*, red},
            ff={mark=diamond*},
            fb={mark=pentagon*},
            fs={mark=otimes, red}
        },
        %every node near coord/.append style={font=\tiny},
        %
        % magic to make the numbers appear above the overly long bars:
        % visualization depends on={rawy \as \rawy}, % save original y values
        % restrict y to domain*={ % now clip/restrict any y value to ymax
        %     \pgfkeysvalueof{/pgfplots/ymin}:\ymaxbreak
        % },
        % after end axis/.code={ % draw squiggly line indicating break
        %     \draw [semithick, white, decoration={snake,amplitude=0.1mm,segment length=0.75mm,post length=0.375mm}, decorate] (rel axis cs:0,1.01) -- (rel axis cs:1,1.01);
        % },
        % nodes near coords={\color{.!75!black}\pgfmathprintnumber\rawy}, % print the original y values (darkened in case they are too light)...
        % nodes near coords greater equal only=\ymax, % ... but ONLY if they are >= ymax
        % clip=false, % allow clip to protrude beyond ymax
        % Custom stuff to edit per template
        %
        xlabel near ticks,
        xlabel shift={-0.5mm},
        xmin=0, xmax=1,
        %enlarge x limits=0.2, % add some breathing room along the x axis's sides
        %
        ylabel near ticks,
        %ylabel shift={-1.5mm},
        ymajorgrids=false,
        ymin=0, ymax=4,
        ytick={ 0, 1, 1.5, 2, 3, 4 },
        yticklabels={ 0,,1.5,, 3, \empty },
        major y tick style=transparent,
        %yticklabels={ 0, 0.5, 1.5, 2 },
        % extra y ticks={1},
        % extra y tick style={grid=major, grid style={dashed, black}},
        % extra y tick label={\empty},
        %bar width=4.5pt, % change size of bars
        %
        legend cell align=center,
        legend style={ column sep=1ex },
        legend entries={%
            {\scriptsize Mirrored Scores (Without SwitchCrypt)},
            {\scriptsize Desired Minimum Score},
            {\scriptsize Actual Minimum Score (SwitchCrypt)}
        },
        legend style={
            draw=none,
            legend columns=2,
            at={(0.5,1.2)},
            anchor=south,
        },
    ]
        \nextgroupplot[
            xlabel={\footnotesize Time (s)},
            ylabel={\footnotesize Security Score},
            xtick={ 0, 0.3, 0.45, 1 },
            xticklabels={ 0, 6, 9, \ldots },
        ]
            \addplot [thick, red] table [
                x=time,
                y=score,
                discard if number not={line}{1},
                col sep=space
            ] {charts/usecase-custody.dat};
            \addplot [thick, red, forget plot] table [
                x=time,
                y=score,
                discard if number not={line}{3},
                col sep=space
                ] {charts/usecase-custody.dat};
            \addplot [thick, densely dotted, blue] table [
                x=time,
                y=score,
                discard if number not={line}{2},
                col sep=space
                ] {charts/usecase-custody.dat};
            \addplot [thick, dashed, blue] table [
                x=time,
                y=score,
                discard if number not={line}{4},
                col sep=space,
                ] {charts/usecase-custody.dat};
            \coordinate (c1) at (0.375, 3.5);
            \coordinate (c2) at (0.415, 3.5);
            \draw [dotted, black] (0.2945, 0) -- (0.2945, 4) node [left of=c1] {\tiny (panic; ISE)};
            \draw [dotted, black] (0.455, 0) -- (0.455, 4) node [right of=c2] {\tiny (ISE completes)};
        \nextgroupplot[
            scatter,
            legend to name={throwaway8},
            xlabel={\footnotesize Latency (normalized)},
            ylabel={\footnotesize Security Score},
            xtick={ 0, 1 },
            xticklabels={ 0, 1 },
        ]
            \addplot [thick] table [
                meta=cipher,
                x=latency,
                y=score,
                discard if symbol not={iop}{40m-r},
                discard if symbol not={order}{seq},
                col sep=space
            ] {charts/tradeoff-baseline.dat};
            \coordinate (c3) at (0.27, 1.75);
            \coordinate (c4) at (0.755, 0.5);
            \draw [dotted] (0.035, 0) -- (0.035, 4) node [above of=c3] {\tiny (pre-panic latency ceiling)};
            \draw [dotted] (0.999, 0) -- (0.999, 4) node [above of=c4] {\tiny (post-panic latency ceiling)};
    \end{groupplot}%
\end{tikzpicture}%
} \caption{Actual security score vs
   security goal with respect to the time and ISE.}
  \label{fig:usecase-custody}
\end{figure}

In \figref{usecase-eol-tradeoff}, we see the system begins at 0 seconds, where
all data is mirrored across the backing store (perhaps consisting of multiple
physical drives). Both the desired and minimum security score of the drive is
1.5, a balance between performance and security. At 6 seconds, custody panic is
triggered---the desired minimum security score goes to 3, the highest possible---at which point the
system executes ISE and completely erases the drive containing the minimally
scored data. ISE is known to be much faster than TRIM and completes in as little
as 3 seconds~\cite{SeaGate,Samsung,ThatOtherOEM}. Once complete, the most secure
form of the data is all that remains. The backing store has been ``locked
down.''

Our goal is to lock down the backing store, slowing down any attacker as
much as possible such that, even if they copy and permanently store her data
off-site for later attempts at decryption with more advanced compute resources
and new technologies, our researcher's data is some degree more likely to remain
irrecoverable. We show that, given a device that supports SSD ISE, SwitchCrypt,
and the Mirrored strategy, we can quickly and practically converge the backing
store to this locked down state. With prior work, data is either too weakly
encrypted or the device becomes too slow for daily use (latency ceiling). In
exchange, we trade off half of our drive's writeable space.

\TODO{Again, need some summary of what we just saw in this section.  What are the lessons learned from these four case studies?  How do they relate to the other points in the paper?}
