\begin{table}[t]
    \begin{center}
        \scriptsize
        \centering
        \begin{verbatim}
            // Change this with a real table, use paragraph
            // style for cell content (I assume you know how)
                      Read                   Write
            ---------------------------------------------------
            Forward   Not-ciphered           Ciphered with B,
                      if 1st read            Re-ciphered on demand
                      Re-ciphered with B on  in the future when
                      2nd read               A is active again
            ---------------------------------------------------
            Mirrored  Read from the prev     Duplicated both
                      cipher's region if     in the prev and
                      during migration.      new regions until
                      Read from the new      migration completes
                      region if migration
                      completes
            ---------------------------------------------------
        \end{verbatim}
    \end{center}

    \mycaption{tab-switch}{Re-ciphering/switching modes}{The table explains for
    every mode what happens on I/Os when the switch happens from cipher A to
    cipher B. ``Read'' means read of existing data during the switch. ``Write''
    means new.}
\end{table}

% ------------- floating: \begin{floatingtable}[r]{ % note the open curly
% bracket \begin{tabular}{...} \end{tabular}....} % note the closed bracket
% here \mycaption{fig-pass}{x}{x.} \end{floatingtable}
