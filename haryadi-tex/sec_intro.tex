\section{Introduction} \label{sec:introduction}



% ======================================================== INTRO

% about: security in general
Security is a very important property of storage systems. Decades of systems
and storage research in this space have looked into security in the context of V
\cite{v1, v2, v3}, W \cite{w1, w2, w3}, X \cite{x1, x2, x3}, Y \cite{y1, y2,
y3}, Z \cite{z1, z2, z3}, and many more.

% HSG: cite as many papers here from top systems venues
% until your reference hits 60 or more



% about: FDE/AES is bad
For local storage, the state of the art for securing data at rest, such as the
contents of a laptop's SSD, is Full Drive Encryption (FDE). Unfortunately,
encryption introduces overhead that drastically impacts system performance and
energy consumption. FDE implementations such as
dm-crypt~\cite{dmcrypt,DmC-Android} for Linux and BitLocker for
Windows~\cite{bitlocker1,bitlocker2} are considered the standard available
solutions. Behind these implementations is the slow AES \emph{block cipher}
(AES-XTS)~\cite{XTS, XTSComments, NISTXTS}. In this regard, the story of
Google's Android OS is a good example. Android supported FDE with the release of
Android 3.0, yet it was not enabled by default until Android
6.0~\cite{android-M-mobile-motivation}. Two years prior, Google attempted to
roll out FDE by default on Android 5.0 but had to backtrack. In a statement to
Engadget, Google blamed ``'performance issues on some partner devices' ... for
the backtracking''~\cite{google-engadget}. At the same time, AnandTech reported
a ``62.9\% drop in random read performance, a 50.5\% drop in random write
performance, and a staggering 80.7\% drop in sequential read performance''
versus Android 5.0 unencrypted storage for various
workloads~\cite{android-M-mobile-motivation-2}.



% about: recent advancements
Fortunately, in the last two years, there have been major advancements in the
FDE technology that distances away from the slow block cipher technology and
successfully implements {\em stream} chipers such as ChaCha20 to full device
encryption. Recent works---such as Google's HBSH (hash, block cipher, stream
cipher, hash)/Adiantum~\cite{Adiantum}, and StrongBox \cite{StrongBox}---brings
stream cipher based FDE to devices that do not or cannot support hardware
accelerated AES. A key to efficiently adopting stream chiper use into the
storage layer is to pair it with a log-structured file system (LFS) such as F2FS
on flash devices. This is because LFSes naturally avoid writing to the same
location multiple times, which requires an expensive re-keying operation when
using stream cipher based FDE.


% about: the need flexibility
These advancements open up a new opportunity that: {\em can a file/storage
system support multiple or flexible switching of full drive encryption?} We
postulate that such a feature is really needed but today we find {\em no}
operating systems that can support such a feature. To motivate this feature,
let's think about this one scenario: local files in a user's mobile phone are
``forced'' to use encryption \encA because it is one of the highest secured
encryption that is safe for backing up the local files to the cloud. However
when the mobile phone runs out of battery and is doing a lot of I/Os, the heavy
encryption will drain the battery down. The user might wish a low battery mode
where the data being accessed is converted into a less powerful encryption
momentarily.


% ======================================================== OUR SOLUTION


% about: the system and the benefits
We present \sys, to the best of our knowledge, the {\em first} kernel support
(at the block level) that provides file systems with flexible switching of full
drive encryption. As FDE's impact on drive performance and energy efficiency
depends on a multitude of choices, different ciphers expose different
performance and energy efficiency characteristics. \sys allows cipher choice to
be be viewed as a key configuration parameter, as opposed to a static choice at
format or boot time. \sys allows ``encryption'' to adapt to changes that arise
while the system is running, including changes in resource availability, runtime
environment, desired security properties, and respecting changing OS energy
budgets. \sys allows users to perform cipher switching in space and time (\eg,
more secure files and temporal switching). \sys allows the software system to
navigate the tradeoff space made by balancing competing security and latency
requirements. \hsg{This is where we list all the benefits; try not to be
redundant} To achieve all these benefits, \sys comes with three important
elements, representing the three main contributions of the paper.


% about: switching strategies
First, we introduce \sysA, a kernel configuration that exposes three types of
switching models: {\em forward}, {\em selective} and {\em mirrored}, to
``re-cipher'' storage units dynamically, allowing us to tradeoff different
performance and security properties of various configurations at runtime. These
switching models define what the I/O layer should do upon the ongoing read/write
I/Os during the switching. These three switching models are motivated from real
case studies. For examples, forward switching is motivated from the battery case
study above; selective switching is motivated from cases where users desire to
have certain files (\eg, legal documents) much more secure than the others; and
mirrored switching is motivated for server-side cases that would like to perform
``encryption upgrade'' without zero downtime.

% about: crypts
Second, to support the switching models above, we implement \sysB, a
block-level module that contains encryption implementations (``crypts'') that
have been restructured to support switching. Prior works mainly implement one
cipher choice and the implementation is very much integrated with the file/block
layer \cite{StrongBox, any-other-works-like-this?}. The key challenge to support
multiple ciphers is that different chipers take different inputs and produce
different type of outputs. For examples, \encA requires nonce input, \encB
config(??\xxx) and \encC sector info and \encC outputs streams while the others
output \xxx. Thus, we introduce a novel design substantively expanding prior FDE
work by wholly decoupling cipher implementations from the encryption process. In
\sysB, we wrote hooks that manages the required input values and the output
format of different ciphers.


% about: tradeoffs
Finally, we initiate \sysC, a scheme that attempts to {\em quantify} the
tradeoffs in the the rich configuration space of stream ciphers. Using this
scheme, we define a tradeoff space of cipher configurations over competing
concerns: total energy use, desirable security properties, read and write
performance (latency), total writable space on the drive, and how quickly the
contents of the drive can converge to a single encryption configuration. \sysC
helps users in understanding their cipher choices as they come with a variety of
performance, energy efficiency, and security properties in the FDE context.


% about: evaluation 
We performed a comprehensive evaluation in several ways.
%
First, we show that \sys successfully supports a wide variety of ciphers;
specifically we have integrated {\em \numCiphers ciphers} and a total of {\em
\numConfigs cipher configurations} into \sys in \locTotal as a kernel block
module (will be open-sourced), and can act as an off-the-shelf replacement for
dm-crypt. The ciphers are ChaCha8 and ChaCha12~\cite{ChaCha20},
Freestyle~\cite{Freestyle}), SalsaX~\cite{SalsaX}, AES in counter mode
(AES-CTR)~\cite{AESCTR}, Rabbit~\cite{Rabbit}, Sosemanuk~\cite{Sosemanuk}, \xxx.
%
Second, we showcase the benefits of \sys with experiments illustrating three
real-world case studies (battery low case, file-level protections, and no
downtime) and show the performance/energy tradeoffs.
%
Finally, we perform several benchmarking to show the performance and \xxx of the
individual ciphers and the switching overhead of the three switching models we
provide \hsg{???}.


