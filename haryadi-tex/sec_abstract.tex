\begin{abstract}

\hsg{for the ``sys'' name. Keep it simple, short and easy to pronounce
and parse. e.g. FlexCrypt is easier to parse than SwitchCrypt}

\hsg{Regarding the title: We don't care it's cipher, etc. 
But the contribution of this paper is more about on the kernel
support for flexibls switching. Feel free to change the title.
Focus on systems/storage component}

\hsg{Also don't put the abstract until the paper is complete.}

\hsg{If the too-many names such as \sysA, \sysB, and \sysC 
are too much, feel free to remove it.  Sometimes if the main contributions
can be clearly divided, adding more names help reviwers
remember the contributions.}

\if 0
Recent work on Full Drive Encryption shows that stream ciphers achieve
significantly improved performance over block ciphers while offering stronger
security guarantees. However, optimizing for performance often conflicts with
other key concerns like energy usage and desired security properties. In this
paper we present SwitchCrypt, a software mechanism that navigates the tradeoff
space made by balancing competing security and latency requirements via
\emph{cipher switching} in space or time. Our key insight in achieving
low-overhead switching is to leverage the overwrite-averse, append-mostly
behavior of underlying solid-state storage to trade throughput for reduced
energy use and/or certain security properties. We implement SwitchCrypt on an
ARM big.LITTLE mobile processor and test its performance under the popular F2FS
file system. We provide empirical results demonstrating the conditions under
which different switching strategies are optimal through the exploration of
three case studies. In one study, where we require the filesystem to react to a
shrinking energy budget by switching ciphers, we find that SwitchCrypt achieves
up to a 3.3x total energy use reduction compared to a static approach using only
the Freestyle stream cipher. In another case, where we allow the user to
manually switch between ChaCha20 and Freestyle stream ciphers dynamically, we
achieve a 1.6x to 4.8x reduction in I/O latency compared to prior static
approaches.
\fi

\end{abstract}

\vten
