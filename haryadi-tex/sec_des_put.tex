

\subsection{Putting It All Together} \label{subsec:summary}

We revisit the motivating example from earlier in this work, where we're using
Freestyle to ensure secure backups in an energy-constrained environment.
Initially, I/O requests come down from the LFS and are received by the
cryptographic driver, which divides the request based on which nuggets it
touches. For each nugget, the per-nugget metadata is consulted to determine with
which cipher the nugget is encrypted. If it is encrypted with the active cipher
configuration (Freestyle), which must be true if we have not initiated a cipher
switch, the write is handled similarly to prior work: encrypted data is read in
from the drive, the merkle tree and monotonic counter are consulted to ensure
the integrity of encrypted data, the transaction journal is consulted during
write operations so that overwrites are handled and pad reuse violations are
avoided, and then the keycount store is consulted to derive the nugget's unique
encryption key from some master secret. Finally, using the Generic Stream Cipher
Interface, we call out to the Freestyle, allowing SwitchCrypt to
encrypts/decrypts the nugget's contents and commit any updates back to storage.
All the while, the drive's Freestyle-encrypted contents are being uploaded up to
our enterprise backup service every so often.

When the device enters ``battery saver'' mode, drive backups are paused, the
energy monitoring software downclocks the CPU, and the OS signals to SwitchCrypt
that a more energy-efficient cipher (ChaCha20) should be used until we return to
a non-curtailed energy budget. SwitchCrypt sets ChaCha20 as the active cipher
configuration. Now, when the cryptographic driver divides I/O requests into each
affected nugget, the per-nugget metadata shows SwitchCrypt that each nugget is
encrypted using a cipher that is not the active configuration. This triggers the
re-ciphering code path. Since we are using the Forward switching strategy in
this example, nugget data is immediately decrypted by calling out to the
inactive configuration through the Generic Stream Cipher Interface, after which
the nugget is re-ciphered by calling out to the active configuration. Finally,
the cryptographic driver manages encrypting/decrypting data and updating the
merkle tree and monotonic counter, transaction journal, and keycount store as
the I/O operation and related metadata is committed to the drive afterwards.

Now, thanks to SwitchCrypt, the system can adapt to changing requirements beyond
the capability of prior work. See \secref{usecases} for specifics.
