

\section{Implementation} \label{subsec:implementation}

Our SwitchCrypt implementation consists of 9,491 lines of C code; our test suite
consists of 6,077 lines of C code. All together, our solution is comprised of
15,568 lines of C code and is publicly available open-source\footnoteref{note1}.

SwitchCrypt uses OpenSSL version 1.1.0h and LibSodium version 1.0.12 for its
AES-XTS and AES-CTR implementations. Open source ARM NEON optimized
implementations of ChaCha are provided by Floodyberry~\cite{Floodyberry}. The
Freestyle cipher reference implementation is from the original Freestyle
paper~\cite{Freestyle}. The eSTREAM Profile 1 cipher implementations are from
the open source libestream cryptographic library~\cite{libestream} by Lucas
Clemente Vella. The Merkle Tree implementation is from the Secure Block
Device~\cite{SBD}.

We implement SwitchCrypt on top of the BUSE~\cite{BUSE} virtual block device,
using it as our mock device controller. BUSE is a thin (200 LoC) wrapper around
the standard Linux Network Block Device (NBD). BUSE allows an operating system
to transact block I/O requests to and from virtual block devices exposed via
domain socket.

We develop Generic Cipher Interface wrapper implementations for many cipher
implementations of which we select five for the purposes of this research. They
are: ChaCha8 and ChaCha20~\cite{ChaCha20} as well as Freestyle~\cite{Freestyle}
in three different configurations: a ``fast'' mode with parameters
\texttt{FFast($R_{min}$=$8$,$R_{max}$=$20$,$H_I$=$4$,$I_C$=$8$)}, a ``balanced''
mode with parameters \texttt{FBalanced($R_{min}$=$12$,
$R_{max}$=$28$,$H_I$=$2$,$I_C$=$10$)}, and a ``strong'' mode with parameters
\texttt{FStrong($R_{min}$=$20$,$R_{max}$=$36$,$H_I$=$1$,$I_C$=$12$)}.
