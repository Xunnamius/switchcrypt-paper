\section{Conclusion} \label{sec:conclusion}

This paper advocates for a more flexible approach to FDE where the storage
system can dynamically adjust the tradeoffs between security and latency/energy.
To support this vision of agile encryption, we proposed an interface that allows
multiple stream ciphers with different input and output characteristics to be
composed in a generic manner. We have identified three strategies for using this
interface to switch ciphers dynamically and with low overhead. We have also
proposed a quantification framework for determining when to use one cipher over
another. Our case studies show how different strategies can be used to optimize
for different goals in practice. We believe that agile encryption will become
increasingly important as successful systems are increasingly required to
balance conflicting operational requirements. We hope that this work inspires
further research in achieving this balance. Our work is publicly available
open-source\footnoteref{note1}.







\hsg{Other notes:\\
- double check figure/table captions to be consistent to what the text says\\
- Add ``as explained in Section \ref{x}'' in every table/figure caption. \\
- Feel free to change any wording that I have.  As mentioned before, I focus
on the logical flow, not choice of words  \\
- Keep things simple, don't introduce too many acronyms.  E.g. I remove that ``VSR''
thing because it's selective basically  \\
- Make sure figure placement is perfect. Put figures on the same page where they
are being described.  That's why I put all figure in separate fig-*.tex files,
so you can easily move them around.  Figure should always use [t].  Reviewers
prefer that than figures that breaks the flow of the text \\
- Make sure figure captions explain the x-axis and y-axis properly. 
Do exlicitly state ``the x-axis shows ..., the y-axis shows ...''. Also as
I mentioned last time, do put more annotations inside the figure/graph to help
reviewers read the final conclusion. 
}

