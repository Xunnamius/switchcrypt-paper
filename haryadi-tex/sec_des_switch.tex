

\subsection{Cipher Switching Strategies} \label{subsec:strategies}



\textbf{Implementing efficient switching strategies.} Finally, to determine when
to switch a nugget's cipher and to where we commit the output, we implement a
series of high-level policies we call \textit{cipher switching strategies}.
These strategies leverage our generic cipher interface and flexible drive layout
to selectively ``re-cipher'' groups of nuggets, whereby the key and the cipher
used to encrypt/decrypt a nugget are switched at runtime. These strategies allow
SwitchCrypt to move from one configuration point to another or even settle on
optimal configurations wholly unachievable with prior work. The challenge here
is to accomplish this while minimizing overhead.


% when and where
The Generic Stream Cipher Interface allows many differently ciphered nuggets to
co-exist on the same drive. However, at any moment, there is only a single
\emph{active cipher configuration} (henceforth \emph{active configuration}). The
active configuration is used to encrypt nugget contents. When a cipher switch is
triggered, a different configuration becomes the active configuration. At this
point, SwitchCrypt must determine \emph{when} to re-cipher a nugget and
\emph{where} to store the output on the drive. ``Re-ciphering'' here means using
an inactive configuration to decrypt a nugget's contents and using the active
configuration to re-cipher it. Depending on the use case, it may make the most
sense to re-cipher a nugget immediately, or eventually, or to maintain several
areas of differently-ciphered nuggets concurrently.

% varying cases 
A naive approach would switch every nugget in BODY to the active configuration
immediately, but the latency and energy cost would be unacceptable. Hence, a
more strategic approach is necessary. We satisfy this need with our \emph{cipher
switching strategies}. These novel strategies allow for nuggets to be
re-ciphered in a variety of cases with minimal impact on performance and battery
life and without compromising security. This is thanks to the nugget-based drive
layout, which limits the churn of cipher switching operations to relatively
small regions of ciphertext on the drive.

% temporal switching
Determining \emph{when} to target a nugget for re-ciphering we call
\emph{temporal switching}, for which we propose the \emph{Forward} switching
strategy. Determining \emph{where}---in which storage region and across which
nuggets---to output ciphertext we call \emph{spatial switching}, for which we
propose the \emph{Mirrored} and \emph{Selective} switching strategies.



\mysub{Forward Switching Strategy.} When a nugget is encountered during I/O
that was encrypted using something other than the active configuration, the
Forward strategy dictates that this nugget be re-ciphered immediately. If a
particular nugget encrypted with an inactive configuration is never encountered
during I/O, it is never re-ciphered and remains on the drive in its original
state. In this way, the Forward strategy represents a form of temporal cipher
switching.

Rather than re-cipher the entire drive every time the active configuration
changes, this strategy limits the performance impact of cipher switching to
individual nuggets. The expense of re-ciphering is paid only once, after which
the nugget is accessed normally during I/O until the active configuration is
switched again.

\PUNT{There are several forms the Forward strategy might take. The default and
most intuitive is \emph{0-forward}, in which SwitchCrypt immediately transitions
individual nuggets encountered during I/O to the active configuration if they
are not using it. Over time, if various I/O operations end up touching every
nugget in the drive, the encrypted contents of every nugget will become
decryptable with the currently active configuration.

The Forward strategy might also take the form of \emph{N-forward}, where
SwitchCrypt attempts to take advantage of spatial sequential locality to
transition whole sets of nuggets into the active configuration. We can trivially
expand the forward strategy to encompass the entire drive by selecting $N$ equal
to the total number of nuggets managed by SwitchCrypt. This would have the
overhead of re-ciphering large swaths of the drive upon every I/O operation
where a nugget encrypted with the inactive configuration is encountered. Of
course, this has the same dire implications for performance as simply
re-initializing the entire system or encrypted container with the new cipher.}

\textbf{Selective Switching Strategy.} When SwitchCrypt is initialized with the
Selective strategy, the drive is partitioned into $C$ regions where $C$
represents the total number of available ciphers in the system; each regions'
nuggets are encrypted by each of the $C$ ciphers respectively. For instance,
were SwitchCrypt initialized using two ciphers ($C = 2$), the drive would be
partitioned in half; all nuggets in the first region would be encrypted with the
first cipher while all nuggets in the second would be encrypted with the other.


When using this strategy, the active cipher determines which partition we
``select'' for I/O operations. Hence, unlike the Forward strategy, which
schedules individual nuggets to be re-ciphered at some point in time after the
active configuration is switched, the Selective strategy allows the wider system
to indicate \emph{where} on the drive a read or write operation should occur. In
this way, the Selective strategy represents a form of spatial cipher switching
where different regions of the drive can store differently-ciphered nuggets
independently and concurrently. A user could take advantage of this to, for
instance, set up regions with different security properties and performance
characteristics, managing them as distinct virtual drives or transparently
reading/writing bytes to different security regions on the same drive.

\textbf{Mirrored Switching Strategy.} Similar to the Selective strategy, when
SwitchCrypt is initialized with the Mirrored strategy, the drive is partitioned
into $C$ regions where $C$ represents the total number of available ciphers in
the system; each regions' nuggets are encrypted by each of the $C$ ciphers
respectively.

However, unlike the Selective strategy, all write operations that hit one region
are mirrored into the other regions immediately, so all regions of the drive
will always be in a consistent state and always share the same data. The active
configuration determines \emph{where} a read operation should occur. In this
way, the Mirrored strategy represents a form of spatial cipher switching because
we're switching which configuration we're using to read in data. A user could
take advantage of this along with SSD Instant Secure Erase~\cite{ISE1,ISE2,ISE3}
to delete other regions, thus quickly and securely converging the drive to a
single configuration without losing any data or suffering the egregious
performance or battery penalty that comes with re-ciphering every nugget.

\subsubsection{Comparing Cipher Switching Strategies}

\begin{table}[ht]
   \begin{tabular}{@{}|c|c|c|C{25mm}|@{}}
      \toprule
      \textbf{Strategy} & \textbf{Convergence} & \textbf{Waste} &
      \textbf{Performance} \\
      \midrule
      Forward   & Slower       & None & Faster reads and writes unless switching
      \\\hline
      Mirrored  & Nearly instant & High & Faster reads; slower writes \\
      \hline
      Selective & Slower       & High & Faster reads and writes  \\
      \hline
   \end{tabular}
   \caption{A summary comparison between the three cipher switching strategies.}
   \label{tbl:strategies-advantages}
\end{table}

\tblref{strategies-advantages} summarizes the higher level tradeoffs between the
three cipher switching strategies.

\textbf{Convergence.} Depending on the use case, the ability to quickly converge
the entire drive to a single cipher configuration without losing data is very
useful (see: \secref{usecases}). The near-instantaneous ``just forget the key''
nature of SSD Instant Secure Erase (ISE) implementations on modern
SSDs~\cite{ISE1,ISE2,ISE3} makes this a very fast process for the Mirrored
strategy. The Forward strategy is slow to converge compared to Mirrored since,
in the worse case, every nugget on the drive will require re-ciphering. The
Selective strategy is similarly slow to converge since entire regions of nuggets
must be moved and re-ciphered to prevent data loss; those regions could be
destroyed without moving data around using ISE too, which would be very fast,
but unlike Mirrored some data would be lost forever.

\textbf{Waste.} Unlike the other two strategies, using the Forward strategy does
not reduce the total usable space on the drive by the end-user, ciphertext
expansion notwithstanding. We refer to this as ``waste''. The Forward strategy
is not wasteful in this way because it allows differently-ciphered nuggets to
co-exist contiguously on the drive without special partitions. Since the
Mirrored and Selective strategies require partitioning the drive into some
number of regions---where the writeable size reported back to the OS is some
function of region size---there is a necessary reduction in usable space.

\textbf{Performance.} The Selective and Mirrored strategies can read data from
the drive with low overhead, reaching performance parity with prior work,
because they never have to deal with on-demand re-ciphering. This is because
switching ciphers using these two strategies amounts to offsetting the read
index so that it lands in the proper BODY partition on the drive, which has
little overhead. The Forward strategy also reads with low overhead except in the
case where a nugget was not encrypted with the active configuration. This
triggers re-ciphering on-demand, which can be costly if the workload constantly
touches unique nuggets and is small enough that cost is not amortized.

The Selective strategy also writes with low overhead because, like with reads,
an index offset is the only requirement. The Mirrored strategy, on the other
hand, can be up to two times slower for writes (when $C = 2$) compared to
baseline. Each additional region ($C > 2$) compounds the write penalty depending
on the workload. This is because each write is mirrored across \emph{all}
regions. As with reads, the Forward strategy writes with low overhead except in
the case where a nugget was not encrypted with the active configuration. This
triggers re-ciphering on-demand, which can be costly if the workload touches
unique nuggets and is small enough that cost is not amortized.\\

With these tradeoffs in mind: Mirrored is ideal when the drive must converge
quickly, write performance is not a primary concern, and drive space is
abundant; Selective is ideal when different data should be encrypted differently
and drive space is abundant; and Forward is ideal when some subset of nuggets
should be encrypted differently without wasting drive space. See
\secref{usecases} for specific scenarios that demonstrate these differences in
practice.

\subsubsection{Threat Model for Cipher Switching Strategies}

The primary concern facing any FDE solution is that of confidentiality. An
adversary should not be able to reveal any information about encrypted plaintext
without the proper key. As with prior work, encryption is achieved via a binary
additive approach: cipher output (keystream) is combined with plaintext nugget
contents using XOR, with metadata to track writes and ensure that pad reuse
never occurs during overwrites and that the system can recover from crashes into
a secure state. Another concern is data integrity: an adversary should not be
able to tamper with ciphertext and it go unnoticed. Nugget integrity is tracked
by an in-memory Merkle tree. See the threat model addressed by Dickens et
al.~\cite{StrongBox} for further details.

Switching strategies add an additional security concern not addressed by prior
work: even if we initiate a ``cipher switch,'' there may still be data on the
drive that was encrypted with an inactive configuration. Is this a problem? For
the Forward strategy, this implies data may at any time be encrypted using the
``least desirable cipher''. For the Mirrored and Selective strategies, the drive
is partitioned into regions where nuggets are guaranteed to be encrypted with
each cipher, including the ``least desirable cipher''. However, in terms of
confidentiality, the confidentiality guarantee of SwitchCrypt can be reduced to
the individual confidentiality guarantees of the available ciphers used to
encrypt nuggets.
