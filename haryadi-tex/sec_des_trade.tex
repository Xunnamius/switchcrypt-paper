

% ===========================================================================
\subsection{Quantifying Cipher Security Properties} \label{subsec:quantify}




% background 
To reason about when to trade off between the ciphers evaluated in this work, we
must have a way to compare ciphers' utility in the context of SwitchCrypt FDE.
To obtain a space of configurations that we might reason about, it is
necessary to compare certain properties of stream ciphers useful in the
FDE context. However, different ciphers have a wide range of security
properties, performance profiles, and output characteristics, including
those that randomize their outputs and those with non-length-preserving
outputs---\ie{the cipher outputs more data than it takes in}. To address
this, we propose a framework for quantitative cipher comparison in the FDE
context; we use this framework to define our configurations.
% combine with the first paragraph above
To address this need, we propose a novel
evaluation framework (see: \tblref{security-quant}). Our framework classifies
stream ciphers according to three quantitative features: relative round count,
ciphertext randomization, and ciphertext expansion. Taken together, these
features reveal a rich tradeoff space of cipher configurations optimizing for
different combinations of concerns.


Table \ref{tab-tracde} .........
We limit our analysis to groups of three implementations, each using a different
number of rounds. In the case of HC-128 and HC-256, we limit our analysis to a
group of two implementations. Scores range from 0 (least number of rounds
considered) to 1 (greatest number of rounds considered).




\begin{table}[t]
\center
\small
   \begin{tabular}{@{}c|cccc@{}}
   \textbf{Cipher} & \textbf{Rounds} & \textbf{Randomization} &
   \textbf{Expansion} \\
   \midrule
   ChaCha8         & 0           & 0           & 1           \\
   ChaCha12        & 0.5         & 0           & 1           \\
   ChaCha20        & 1           & 0           & 1           \\
   Salsa8          & 0           & 0           & 1           \\
   Salsa12         & 0.5         & 0           & 1           \\
   Salsa20         & 1           & 0           & 1           \\
   HC128           & 0           & 0           & 1           \\
   HC256           & 1           & 0           & 1           \\
   Freestyle (F)   & 0           & 2           & 0           \\
   Freestyle (B)   & 0.5         & 2.5         & 0           \\
   Freestyle (S)   & 1           & 3           & 0           \\
\end{tabular}
\mycaption{tab-trade}{Quantifying ciphers}{Our framework for classifying stream ciphers 
according to three
   ideal features: relative round count, ciphertext randomization, and
   ciphertext expansion, as discussed in Section \ref{x}.}
   \label{tbl:security-quant}
 \end{table}


% ----------------------------------------------------------
\mysub{Relative Rounds (Rounds)}
The ciphers we examine in this paper are all constructed around the notion of
\emph{rounds}, where a higher number of rounds (and possibly longer key) is
positively correlated with a higher resistance to brute force given no fatal
related-key or other attacks~\cite{ChaCha-Cryptanalysis}. Hence, this feature
represents how many rounds a cipher executes relative to other implementations
of the same algorithm. For instance: ChaCha8 is a reduced-round version of
ChaCha12, which is a reduced-round version of ChaCha20, all using the ChaCha
algorithm~\cite{ChaCha20,ChaCha-Cryptanalysis}.


% ----------------------------------------------------------
\mysub{Ciphertext Randomization (Randomization)}
A cipher with ciphertext randomization generates different ciphertexts
non-deterministically given the same key, nonce, and plaintext. This makes it
much more difficult to execute chosen-ciphertext attacks (CCA), key
re-installation attacks, XOR-based cryptanalysis and other comparison attacks,
and other confidentiality-violating schemes where the ciphertext is in full
control of the adversary ~\cite{Freestyle}. This property is useful in cases
where we cannot prevent the same key, nonce, and plaintext from being reused,
such as with data ``in motion'' (see the motivational example earlier in this
work). Ciphers without this property---such as ChaCha20 on which prior work is
based---are trivially broken when key-nonce-plaintext 3-tuples are reused. In
StrongBox, this is referred to as an ``overwrite condition'' or simply
``overwrite''~\cite{StrongBox}.

Though there are many ways to achieve ciphertext randomization, the ciphers
included in our analysis implement it using a random number of rounds for each
block of the message where the exact number of rounds are unknown to the
receiver a priori~\cite{Freestyle}. In determining the minimum and maximum
number of rounds used per block in this non-deterministic mode of operation, we
can customize the computational burden an attacker must bear by choosing lower
or higher minimums and maximums. Hence, this is not a binary feature; scores
range from 0 (no ciphertext randomization) to 1 (lowest minimum and maximum
rounds per block) to 3 (highest minimum and maximum rounds per block).

% ----------------------------------------------------------
\mysub{Ciphertext Expansion (Expansion)}
A cipher that exhibits ciphertext expansion is non-length-preserving: it outputs
more or less ciphertext than was originally input as plaintext. This can cause
major problems in the FDE context. For instance, cryptosystems that rely on
AES-XTS (e.g. Linux's dm-crypt, Microsoft's BitLocker, Apple's FileVault) or
ChaCha (e.g. StrongBox, Google's Adiantum) have storage layouts that hold
length-preserving output as an invariant, making ciphers that do not exhibit
this property incompatible with their implementations; yet, ciphertext expansion
is often (but not always) a necessary side-effect of ciphertext randomization.

The ciphers included in our analysis that exhibit ciphertext expansion have an
overhead of around 1.56\% per plaintext message block~\cite{Freestyle}. Even a
single byte of additional ciphertext vs plaintext would make a cipher
inappropriate for use with prior work. Hence, this is a binary feature in that a
cipher either outputs ciphertext of the same length as its plaintext input or it
does not. A cipher scores either a 0 if it \emph{is not} length-preserving in
this way or a 1 if the ciphertext is always the same length as the plaintext.



\hsg{So at the end how do we use this? what are these for?
we need to give some examples here, or perhaps already covered in case studies
in later sections??}

